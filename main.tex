\documentclass[11pt]{book} %10pt 11pt o 12pt permessi
%\documentclass{book}
\usepackage[T1]{fontenc} % codifica dei font
\usepackage[utf8]{inputenc} % lettere accentate da tastiera
\usepackage[italian]{babel} % lingua del documento
\usepackage{lipsum}

\usepackage{indentfirst} % indenta primo capoverso di paragrafo
\usepackage{microtype} % migliora riempimento righe - carica sempre
\usepackage{appendix}
\usepackage{emptypage}
\usepackage{tikz}
\usepackage{pgfplots}
\pgfplotsset{compat=1.17}
\usetikzlibrary{positioning}

%\usepackage[big]{layaureo} % con big nell'argomento bordi più larghi ancora
\usepackage[top=3cm,bottom=3cm,left=3.2cm,right=3.2cm,headsep=10pt, letterpaper]{geometry} % Page margins letterpaper

\usepackage{lipsum} % genera testo fittizio
\usepackage{comment}
\usepackage{quoting} % per le citazioni in display
\quotingsetup{font=small} % serve per mantenere lo stesso stile in tutte le citazioni

\usepackage{amsmath} % per la matematica
\usepackage{amssymb} % per la matematica
\usepackage{mathtools} % per valore assoluto e norma
\usepackage{braket} % per i comandi \Set e \Bra e simili
\usepackage{amsthm} % per teoremi e dimostrazioni

\usepackage[version=4]{mhchem} %CHIMICA

\usepackage{siunitx} % per unità di misura SI
\usepackage{textcomp} % SOLO SE si usano gradi Celsius

\usepackage{booktabs} % per tabelle
\usepackage{caption} % per tabelle
\usepackage{graphicx} % per figure
\usepackage{subfig} % per sottofigure

%%%%%%%%%%%%%% NEW THINGS %%%%%%%%%%%%%%%%%%%%%%%%%%%%%%%
\usepackage{xcolor} % Required for specifying colors by name
\definecolor{ocre}{RGB}{52,177,201} % Define the orange color used for highlighting throughout the book

% Font Settings
\usepackage{avant} % Use the Avantgarde font for headings
%\usepackage{times} % Use the Times font for headings
\usepackage{mathptmx} % Use the Adobe Times Roman as the default text font together with math symbols from the Sym­bol, Chancery and Com­puter Modern fonts

% Bibliography
%\usepackage[style=alphabetic,sorting=nyt,sortcites=true,autopunct=true,babel=hyphen,hyperref=true,abbreviate=false,backref=true,backend=biber]{biblatex}
%\addbibresource{bibliography.bib} % BibTeX bibliography file
%\defbibheading{bibempty}{}
%%%%%%%%%%%%%%%%%%%%%%%%%%%%%%%%%%%%%%%%%%%%%%%%%%%%%%%%%%%%%%

%\usepackage{hyperref} % metti alla fine
\usepackage{impostazioni} % per ultimo

\input{structure} % Insert the commands.tex file which contains the majority of the structure behind the template

\begin{document}

\title{Appunti di Astrofisica}

%----------------------------------------------------------------------------------------
%	TITLE PAGE
%----------------------------------------------------------------------------------------

\begingroup
\thispagestyle{empty}
\AddToShipoutPicture*{\put(0,0){\includegraphics[scale=1.25]{cover/esahubble}}} % Image background
\centering
\vspace*{5cm}
\par\normalfont\fontsize{35}{35}\sffamily\selectfont
\textbf{Appunti di Astrofisica}\\
{\LARGE Corso della prof.ssa B. Lanzoni~- A.A. $2022/2023$}\par % Book title
\vspace*{1cm}
{\Huge G.~Oancia, S.~Coli, N.~Cognome, N.~Cognome}\par % Author name
\endgroup

%----------------------------------------------------------------------------------------
%	COPYRIGHT PAGE
%----------------------------------------------------------------------------------------

\newpage
~\vfill
\thispagestyle{empty}

%\noindent Copyright \copyright\ 2014 Andrea Hidalgo\\ % Copyright notice

\noindent \textsc{Università di Bologna}\\

\noindent \textsc{github.com/gianchix/appunti-astrofisica}\\ % URL

\noindent Questo è del testo che si può aggiungere a caso\\ % License information

\noindent \textit{Prima pubblicazione, Gennaio 2023} % Printing/edition date

%----------------------------------------------------------------------------------------
%	TABLE OF CONTENTS
%----------------------------------------------------------------------------------------

\chapterimage{cover/head1.png} % Table of contents heading image

\pagestyle{empty} % No headers

\tableofcontents % Print the table of contents itself

%\cleardoublepage % Forces the first chapter to start on an odd page so it's on the right

\pagestyle{fancy} % Print headers again

%----------------------------------------------------------------------------------------
%	CHAPTER 1
%----------------------------------------------------------------------------------------



%\frontmatter
%\input{capitoli/cover/cover1.tex}
%\tableofcontents
%\mainmatter
\chapterimage{cover/head2.png} % Chapter heading image
\chapter{Setting the stage}
\section{Introduzione}\label{sec:introduzione}
Come per ogni ambito della fisica, anche per l'\emph{astrofisica} è necessario avere in mente gli ordini di grandezza con cui si lavora e gli oggetti di studio. Nella tab.~\ref{tab:ordini-grandezza-sole-vialattea} sono riportate alcune grandezze riferite al nostro Sole e alla Via Lattea. A causa delle scale spaziali e temporali enormi rispetto a quella umana, risulta ovvio che non è possibile riprodurre le strutture cosmiche in laboratorio, andare su di esse per prendere delle misure dirette, oppure seguirne la loro evoluzione istante per istante. Dunque, è necessario uno studio \emph{indiretto}, che avvenga analizzando la radiazione che riceviamo, che proviene necessariamente dal passato. In prima approssimazione possiamo ritenere che sia stata emessa $\Delta t = D / c$ tempo fa, con $D$ distanza della sorgente e $c$ velocità della luce nel vuoto. Un altro limite delle osservazioni astrofisiche consiste nel non avere accesso alla terza dimensione, ovvero alla profondità rispetto alla linea di vista. Inoltre, è necessario un \emph{approccio statistico} per comprendere i processi evoluti, cercando, in prima istanza, di raggruppare oggetti simili, i quali presumibilmente abbiamo la stessa età. Ovviamente l'astrofisica utilizza le \emph{leggi della fisica} per interpretare i dati osservativi e avanzare predizioni teoriche, tramite \emph{modelli} e \emph{simulazioni}.

\begin{table}
    \caption{Ordini di grandezza riferiti al Sole e alla Via Lattea.}
    \label{tab:ordini-grandezza-sole-vialattea}
    \centering
    \begin{tabular}{lll}
    \toprule
    Grandezza & Sole & Via Lattea  \\
    \midrule
    Raggio (\si{cm}) & $\sim \SI{6.7e10}{}$ & $\sim \SI{8e22}{}$ \\
    Massa (\si{g}) & $\sim \SI{2e33}{}$    & $\sim \SI{e45}{}$    \\
    Età (\si{s}) & $\sim \SI{1.4e17}{}$  & $\sim \SI{3.8e17}{}$  \\
    Distanza da noi (\si{cm}) & $\sim \SI{1.5e13}{}$  & $\sim \SI{2.5e22}{}$  \\
    \bottomrule
    \end{tabular}
\end{table}
\section{Grandezze e unità di misura}\label{sec:unita-di-misura}
Gli ordini di grandezza tipici delle strutture studiate dall'astrofisica rendono necessario introdurre delle nuove unità di misura rispetto a quelle standard.

\subsection{Lunghezze e massa}
Di seguito sono elencate le unità di misura maggiormente utilizzate per esprimere le lunghezze e le masse in astrofisica.
\begin{description}
    \item[Raggio solare] Vale $\si{\solarradius} \simeq \SI{6.7e10}{cm}$. Nella tabella~\ref{tab:ordini-grandezza-dimensioni-stelle} sono riportate le dimensioni di alcune stelle, espresse in raggi solari. Questa unità di misura viene utilizzata principalmente per esprimere la distanza di stelle e pianeti.
    \item[Unità astronomica] Esprime la distanza media tra Terra e Sole. Vale $\SI{1}{AU} \simeq \SI{1.5e13}{cm}$. Si utilizza principalmente per indicare distanze riferite al sistema solare e dintorni.
    \item[Anno luce] Rappresenta la distanza che la luce percorre nel vuoto in un anno, si utilizza molto nei libri di divulgazione scientifica ma non in ambito professionale. Vale $\SI{1}{ly} \sim \SI{9.5e17}{cm}$.
    \item[parsec] Indica la distanza alla quale $\SI{1}{AU}$ sottende un angolo di $\ang{;;1}$ (\emph{secondo d'arco}), come mostrato in figura~\ref{fig:parsec}. Vale $\SI{1}{pc} = \SI{3.1e18}{cm}$. Nella tabella~\ref{tab:ordini-grandezza-dimensioni-stelle} sono riportate le dimensione di alcune strutture cosmiche. Questa unità di misura di utilizza principalmente in astrofisica galattica ed extra--galattica.
    \item[Redshift] Viene utilizzato per indicare distanze dell'universo lontano e in cosmologia.
    \item[Massa Solare] Vale $\si{\solarmass} = \SI{2e33}{g}$. In tabella~\ref{tab:masse-solari} sono presenti alcuni valori tipici di massa.
\end{description}

\begin{table}
    \caption{Dimensioni di alcune stelle e strutture cosmiche.}
    \label{tab:ordini-grandezza-dimensioni-stelle}
    \centering
    \begin{tabular}{ll@{\qquad}ll}
        \toprule
        \multicolumn{2}{c}{Ordini di grandezza per le stelle} & \multicolumn{2}{c}{Ordini di grandezza per alcune strutture stellari} \\
        Stella & Dimensioni & Struttura & Dimensioni \\
        \midrule
        Giove & $\sim \SI{0.1}{\solarradius}$ & Galassie & $\sim \textup{qualche} \, \si{kpc}$ \\
        Giganti rosse & $\sim \SI{2000}{\solarradius}$ & Via Lattea & $\sim \SI{25}{kpc}$ \\
        Nane bianche & $\sim \SI{6000}{km}$ & Ammassi di Galassie & $\sim \textup{qualche} \, \si{Mpc}$  \\
        Stelle di neutroni & $\sim \SI{10}{km}$ & Nubi di Magellano & $\sim \SI{60}{kpc}$  \\
        \bottomrule
        \end{tabular}
\end{table}

\begin{table}
\caption{Masse di alcune strutture cosmiche.}
\label{tab:masse-solari}
\centering
\begin{tabular}{ll}
\toprule
Struttura & Massa (\si{\solarmass}) \\
\midrule
Stelle          & $0.08$--$150$ \\
Ammassi stellari           & $10^3$--$10^6$    \\
Galassie            & $10^7$--$10^{13}$  \\
Ammassi di galassie & $10^{14}$--$10^{15}$  \\
\bottomrule
\end{tabular}
\end{table}

\begin{figure}
    \centering
    \includegraphics[width=0.3\textwidth]{immagini/parsec.png}
    \caption{Definizione di parsec. $\SI{1}{pc}$ è la distanza alla quale $\SI{1}{AU}$ sottende un angolo di $\ang{;;1}$.}
    \label{fig:parsec}
\end{figure}

\subsection{Sfera celeste}\label{sec:posizione-sfera-celeste}
Come anticipato precedentemente, non siamo in grado di osservare la profondità degli oggetti cosmici, ma solo la loro proiezione sulla \emph{sfera celeste}. Questa è, per definizione, una sfera di raggio indeterminato (per convenzione si pone $R=1$) centrata nell'osservatore (come mostrato in fig.~\ref{fig:sfera-celeste}). In genere, a seconda della convenienza, si considera in centro della sfera celeste coincidente con il \emph{centro della Terra} (sfera geocentrica), con il \emph{centro del Sole} (sfera eliocentrica) oppure con il \emph{baricentro del sistema solare} (sfera baricentrica). Se la distanza del corpo che si sta studiando è grande, scegliere uno dei tre centri è del tutto equivalente. Si definisce \emph{equatore celeste} la proiezione dell'equatore terrestre sulla sfera celeste e non è altro che un cerchio massimo. Si definisce \emph{asse del mondo} la retta passante per il centro $O$ della sfera celeste e perpendicolare al piano equatoriale. L'intersezione tra questa retta e la sfera celeste sono detti \emph{poli celesti}: attualmente il polo nord celeste è circa nella direzione della stella polare, mentre il polo sud celeste è vicino alla croce del sud. 

L'\emph{eclittica} è definita come il cerchio massimo descritto dalla traiettoria (apparente) del sole attorno alla Terra, ovvero è l'intersezione tra la sfera celeste e il piano orbitale della Terra attorno al Sole. Essa è inclinata di $\ang{23;27;}$ rispetto all'equatore celeste e interseca l'equatore in due punti opposti, detti \emph{equinozi}: il \emph{punto della Bilancia} e il \emph{punto dell'Ariete} (o \emph{punto vernale} o \emph{equinozio di primavera}).

\begin{figure}
\centering
\includegraphics[width=0.5\textwidth]{immagini/sfera-celeste.png}
\caption{Sfera celeste. L'eclittica è l'intersezione tra la sfera celeste e il piano orbitale della Terra attorno al Sole. Essa interseca l'equatore terrestre in due punti opposti detti equinozi. Il punto dell'Ariete corrisponde all'equinozio di primavera (21 marzo), detto anche punto vernale.}
\label{fig:sfera-celeste}
\end{figure}

Lo \emph{Zenith} (fig.~\ref{fig:zenith}) è la congiungente tra il centro della Terra e l'osservatore, cioè la verticale dell'osservatore stesso. L'\emph{orizzonte astronomico} è il cerchio massimo formato dall'intersezione tra la sfera celeste e il piano perpendicolare alla verticale dell'osservatore. Ovviamente, l'osservatore è in grado di vedere solamente ciò che si trova sopra all'orizzonte astronomico. Si noti che lo Zenith e l'orizzonte astronomico dipendono dalla posizione dell'osservatore sulla Terra.

\begin{figure}
\centering
\includegraphics[width=0.5\textwidth]{immagini/zenit.png}
\caption{Zenith e orizzonte astronomico. L'osservatore vede solo ciò che sta sopra l'orizzonte astronomico. Dipendono da dove è posizionato l'osservatore sulla Terra.}
\label{fig:zenith}
\end{figure}

In ogni punto della Terra, l'osservatore la vede ruotare attorno all'asse del mondo, che passa per i \emph{poli celesti}, di cui solamente uno è visibile sopra all'orizzonte (cfr.~\ref{fig:sfera-celeste}). Il moto delle stelle, rappresentato in fig.~\ref{fig:movimento-stelle} avviene da est verso ovest ed è un moto rigido attorno all'asse del mondo. Le distanze relative tra le stelle appaiono, perciò, fisse, ciononostante ogni stella si muove di moto proprio, ma le variazioni di posizione sono così, data la distanza, da non essere apprezzabili. Il periodo di rotazione delle stelle definisce il così detto \emph{giorno siderale}.

\begin{figure}
\centering
\includegraphics[width=0.3\textwidth]{immagini/movimento-stelle.png}
\caption{Come ci appare il movimento delle stelle. Tutte le stelle sopra il cerchio giallo non tramontano mai, quelle sotto il cerchio rosso non sorgono mai e le altre stelle le vediamo sorgere e tramontare}
\label{fig:movimento-stelle}
\end{figure}

\subsection{Sistema equatoriale}
Come definire un sistema di coordinate sulla sfera celeste? Si ricordi che noi osserviamo delle proiezioni su un piano, quindi sono sufficienti due coordinate. In particolare, conviene utilizzare delle coordinate angolari e il sistema di coordinate più utilizzato è il \emph{sistema equatoriale} (fig.~\ref{fig:sistema-equatoriale}). Esso usa come cerchi di riferimento l'equatore celeste e il meridiano\footnote{Un meridiano è un cerchio perpendicolare all'equatore.} passante per il punto vernale ($\gamma$). L'origine del sistema di coordinate è nel punto vernale, $O \equiv \gamma$, e ha per coordinate angolari l'ascensione della retta e la declinazione:
\begin{description}
    \item[Ascensione retta (RA o $\alpha$)] distanza angolare tra il punto $\gamma$ e il meridiano dell'astro. Si misura in ore, minuti e secondi (hms), varia tra $0^h$ e $24^h$, aumentando verso E.
    \item[Declinazione (Dec o $\delta$)] distanza angolare tra l'equatore celeste e l'astro, lungo il meridiano dell'astro. Si misura in gradi, primi e secondi (gradi, arcominuti e arcosecondi), varia tra $\ang{0}$ e $+\ang{90}$ dall'equatore al polo N e tra $\ang{0}$ e $-\ang{90}$ dall'equatore al polo S.
\end{description}
Si faccia attenzione al fatto che i minuti e secondi d'orologio (RA) sono \emph{diversi} dai minuti e secondi d'arco (Dec). Utilizzando:
\[
    1^h = 60^m = 3600^s \qquad \ang{1} = 60' = 3600'' \qquad 24^h = \ang{360}
\]
si trova che:
\[
    1^h = \ang{15} \qquad 1^m = 15' \qquad 1^s = 15''
\]

\begin{figure}
\centering
\includegraphics[width=0.3\textwidth]{immagini/sistema-equatoriale.png}
\caption{$\alpha$ è l'ascensione retta, ovvero la distanza angolare tra il punto $\gamma$ e il meridiano dell'astro, $\delta$ è la declinazione, ovvero la distanza angolare tra l'equatore celeste e l'astro, lungo il meridiano dell'astro.}
\label{fig:sistema-equatoriale}
\end{figure}

Si guardi la figura~\ref{fig:esempio-sistema-equatoriale} per un esempio sull'utilizzo di tale sistema di coordinate.

\begin{figure}
\centering
\includegraphics[width=0.5\textwidth]{immagini/esempio-sistema-equatoriale.png}
\caption{Esempio di utilizzo delle coordinate del sistema equatoriale.}
\label{fig:esempio-sistema-equatoriale}
\end{figure}


%\chapterimage{}
\chapter{Meccanismi di emissione}
\section{Introduzione all'astrofisica osservativa}\label{sec:astrofisica-osservativa}
Come introdotto nel paragrafo~\ref{sec:introduzione}, l'astrofisica studia le strutture cosmiche in maniera indiretta attraverso la radiazione che raccogliamo con i telescopi e altri strumenti. In particolare, osservando a bande diverse, si possono estrapolare informazioni quantitative dalla misura della radiazione. Si tenga tuttavia a mente che la radiazione che riceviamo dipende sia dai processi fisici che l'hanno generata sia da quelli che ha subito nel tragitto tra la sorgente e l'osservatore.

Il primo problema dell'osservazione è il così detto \emph{assorbimento atmosferico}. Infatti, non tutta la radiazione emessa dagli oggetti astrofisici riesce a raggiungere la superficie terrestre. In particolare, ci giungono il vicino ultravioletto, il vicino infrarosso e le onde radio e visibili, mentre il resto della radiazione non arriva a terra perché viene assorbita dall'atmosfera. È dunque necessario osservare con telescopi spaziali in orbita. Un altro problema che si riscontra è quello del \emph{seeing}, secondo il quale le particelle di vapor acqueo dell'atmosfera creano una turbolenza che sfoca le immagini acquisite dai telescopi. Per questo motivo è conveniente effettuare le osservazioni in posti poco umidi, come il Cile o le Hawaii. Senza tecniche per correggere tale problema sarebbe inutile cercare di avere telescopi di diametro maggiore per avere una maggior qualità dell'immagine. Guardando una stella, generalmente l'immagine risulta un disco sulla lente, mentre se c'è turbolenza si ha un'immagine distorta e sbrodolata. Un'immagine affetta dal problema del seeing si dice \emph{seeing-limited}. Si chiama inoltre \emph{diffraction-limited} un immagine la cui qualità dipende solo dal potere risolutivo data la minore distanza angolare che si può risolvere:
\[
    \theta = 1.22 \frac{\lambda}{D}
\]
dove $\lambda$ è la lunghezza d'onda dell'onda incidente sulla lente di diametro $D$. Si può notare che, se fossimo in grado di sbarazzarci del problema del seeing, aumentando il diametro della lente saremmo in grado di diminuire il potere risolutivo e quindi potremmo risolvere oggetti più vicini. Fortunatamente, la tecnologia attuale offre la possibilità di correggere il problema del seeing attraverso l'utilizzo di \emph{ottiche adattive}. Nella figura~\ref{fig:ottiche-adattive} si spiega il loro funzionamento. 

\begin{figure}
\centering
\includegraphics[width=0.4\textwidth]{immagini/ottiche-adattive.png}
\caption{Principio di funzionamento delle ottiche adattive. Un fronte d’onda distorto incide su uno specchio deformabile (con pistoni idraulici che si allungano o contraggono), la luce viene mandata a un sistema (waveform camera) che controlla se l’immagine della stella è soggetta a turbolenza oppure no, se è distorta (ovvero soggetta a turbolenza), i pistoncini idraulici deformano lo specchio iniziale per annullare l'effetto.}
\label{fig:ottiche-adattive}
\end{figure}

Per calcolare la deformazione da applicare alla lente per correggere il problema del seeing, si utilizza una \emph{stella guida}, che deve essere una stella molto brillante. Siccome non sempre è presente una stella molto brillante per effettuare la calibrazione, si utilizza una \emph{stella laser}, ovvero i telescopi sparano fasci laser brillanti che simulano la presenza di una stella brillante attraverso l'eccitazione di atomi di sodio a $\SI{90}{km}$ di altitudine. Un altro modo per evitare il seeing è banalmente evitare l'atmosfera terrestre, dunque utilizzando telescopi spaziali, che sono solamente diffraction-limited.

Si vuole, infine, sottolineare la differenza tra un telescopio e i suoi strumenti a bordo: ciascun telescopio, infatti, ha diversi strumenti a bordo. Questi si dividono in due categorie particolari: le \emph{photometric cameras}, utilizzate per l'imaging o la spettroscopia, e gli \emph{spettrografi}, utilizzati per la spettrografia, ovvero per misurare lo spettro della radiazione, che può essere in assorbimento o in emissione.
\section{Equazione del trasporto radiativo}\label{sec:intro-trasporto-radiativo}
Come anticipato nel paragrafo~\ref{sec:astrofisica-osservativa}, per poter interpretare in maniera corrette le misure di radiazione, è necessario studiare i meccanismi di \emph{interazione tra la radiazione e la materia}. Le cause principali di tale interazione sono l'\emph{assorbimento}, lo \emph{scattering}\footnote{deviazione rispetto alla direzione di propagazione originale} e l'\emph{emissione}. Questi fenomeni variano a seconda della \emph{frequenza} e questo è il motivo per cui osservare a frequenze diverse porta a informazioni diverse.

L'\emph{equazione del trasporto radiativo} descrive il trasporto di radiazione da parte dei fotoni. Non è l'unico meccanismo di trasporto ma è il più comune nell'universo. In particolare, l'equazione dice come varia l'intensità della radiazione a causa dell'interazione tra la radiazione e la materia. Prima di ricavare l'equazione è necessario introdurre alcune grandezze indispensabili per indicare le energie.

\subsection{Intensità specifica o brillanza}\label{sec:intensità}
Dato un campo di radiazione, l'intensità specifica $I_\nu$ lungo la direzione $k$, in un punto qualsiasi $P$ del campo, è la quantità di energia che, in un intervallo di tempo $\ud t$ e nell'intervallo di frequenze $\ud \nu$, attraversa una superficie infinitesima perpendicolare alla direzione $k$ ($\ud A_\perp$), entro un angolo solido elementare $\ud \Omega$:
\begin{equation}\label{eq:intensità}
    I_\nu = \frac{\ud E}{\ud t \ud \nu \ud A \ud \Omega \cos{\theta}} \quad \bigl[\si{erg.s^{-1}.cm^{-2}.Hz^{-1}.sr^{-1}}\bigr]
\end{equation}
dove $\theta$ è l'angolo tra la normale a $\ud A$ e la direzione $k$, ovvero posso scrivere: $\ud A_\perp = \ud A \cos{\theta}$. Si guardi la figura~\ref{fig:intensità-specifica}.

\begin{figure}
\centering
\includegraphics[width=0.3\textwidth]{immagini/intensita-specifica.png}
\caption{Definizione di intensità specifica. $\ud A$ è un'area infinitesima sulla superficie della stella, $P$ è il punto della superficie della stella, $k$ è il versore rispetto al quale definisco l'intensità e $\ud A_\perp = \ud A \cos{\theta}$. La radiazione è portata da un fascio di fotoni, quindi devo considerare un tronco di angolo solido $\ud \Omega$, in cui $\ud A$ rappresenta la sezione del fascio.}
\label{fig:intensità-specifica}
\end{figure}

Si tratta di una proprietà \emph{intrinseca} del campo di radiazione, ovvero della sorgente e si noti come sia definita per unità di frequenza o equivalentemente di lunghezza d'onda.

Si ricordi, inoltre, che il differenziale di angolo solido si scrive:
\begin{equation}\label{eq:angolo-solido}
    \ud \Omega = \sin{\theta} \ud \theta \ud \phi
\end{equation}

Più avanti vedremo uno spettro o \emph{distribuzione spettrale di energia (SED)}. Si tratta della dipendenza dell'intensità dalla lunghezza d'onda (equivalentemente dalla frequenza). 

\subsection{Luminosità}\label{sec:luminosità}
La \emph{luminosità bolometrica} ($L$) di una sorgente è definita come la quantità di energia totale emessa dalla sorgente nell'unità di tempo. Si misura in $\si{erg.s^{-1}}$ oppure in \emph{luminosità solari}, dove $\si{\solarluminosity} = \SI{3.8e33}{erg.s^{-1}}$. Essa è una quantità \emph{intrinseca} della sorgente.

La \emph{luminosità monocromatica} ($L_\nu$) è la luminosità per unità di frequenza (o lunghezza d'onda), cioè la quantità di energia totale emessa dalla sorgente nell'unità di tempo e nell'intervallo di frequenze tra $\nu$ e $\nu + \ud \nu$. $L_\nu$ si misura in $\si{erg.s^{-1}.Hz^{-1}}$, mentre $L_\lambda$ si misura in $\si{erg.s^{-1}.cm^{-1}}$. È semplice ricavare la relazione che lega $L_\nu$ e $L_\lambda$, considerando che $\nu = c / \lambda$. Da questo segue che a un aumento di $\lambda$ corrisponde una diminuzione di $\nu$, ovvero $L_\nu \ud \nu = - L_\lambda \ud \lambda$ e anche che $\ud \nu / \ud \lambda = - c / \lambda^2$. Menttendo le due espressioni insieme si ottiene:
\begin{equation}\label{eq:luminosità-monocromatica}
    L_\lambda = \frac{c}{\lambda^2} L_\nu
\end{equation}

Inoltre è semplice ottenere la luminosità bolometrica a partire da quella monocromatica, è infatti sufficiente integrare su tutte le frequenze:
\begin{equation}\label{eq:luminosità-bolometrica}
    L = \int_0^\infty L_\nu \ud \nu = \int_0^\infty L_\lambda \ud \lambda
\end{equation}
Infatti, per il \emph{teorema di Fourier} è possibile scomporre un'onda nelle sue componenti monocromatiche e nel computo dell'intensità bolometrica considero il contributo di tutte le frequenze.

Ci si può ora chiedere, quale sia la relazione tra l'intensità specifica e la luminosità.

\subsection{Relazione tra intensità e luminosità}\label{sec:relazione-intensità-luminosità}
Consideriamo un campo di radiazione isotropo, per cui $I_\nu$ è uguale per ogni direzione $k$. Questo significa che se considero una superficie infinitesima $\ud A$, la quantità di radiazione entrante e uscente da tale superficie è la stessa per ogni superficie. La luminosità della sorgente è la quantità di radiazione \emph{emessa} dalla sorgente nel tempo $\ud t$, cioè la quantità di radiazione che \emph{esce} dalla sorgente stessa, ovvero la quantità di radiazione che \emph{esce da ciascuna superficie $\ud A$, integrata sulla superficie totale della sorgente}.

Per ottenere la quantità di radiazione che \emph{esce} da ciascuna superficie infinitesima $\ud A$ bisogna integrare sull'angolo solido dell'emisfera \emph{uscente}, ovvero per $\phi \in [0, 2\pi]$ e $\theta \in [0, \pi/2]$. Si faccia riferimento alla figura~\ref{fig:emisfera-uscente}.

\begin{figure}
\centering
\includegraphics[width=0.3\textwidth]{immagini/emisfera-uscente.png}
\caption{La figura rappresenta un elemento infinitesimo $\ud A$ sulla superficie di una stella. Voglio trovare la radiazione uscente, dunque è quella che attraversa la semisfera uscente che ha come cerchio massimo $\ud A$, poi integro in $\ud A$ per ottenere la radiazione uscente da tutta la superficie della stella}
\label{fig:emisfera-uscente}
\end{figure}

Per ottenere la luminosità totale della sorgente, bisogna integrare su tutta la superficie $4\pi R^2$:
\[
    L_\nu = I_\nu \int_0^{4\pi R^2} \ud A \int_0^{2\pi} \ud \phi \int_0^{\pi/2} \ud \theta \sin{\theta} \cos{\theta}
\]
dove $I_\nu$ è stato portato fuori dal segno di integrale perché non dipende dagli angoli poiché la sorgente è isotropa. Sviluppando i conti si ottiene un'espressione per la luminosità monocromatica
\begin{equation}\label{eq:luminosità-monocromatica-intensità}
    L_\nu = 4 \pi R^2 \pi I_\nu
\end{equation}
e integrando su tutte le frequenze si ottiene la luminosità bolometrica
\begin{equation}\label{eq:luminosità-intensità}
    L = 4 \pi R^2 \pi I
\end{equation}

\subsection{L'equazione e le sue soluzioni}\label{sec:trasporto-radiativo}\label{sec:soluzioni-trasporto-radiativo}

\begin{figure}
\centering
\includegraphics[width=0.7\textwidth]{immagini/trasporto-radiativo.png}
\caption{Equazione del trasporto radiativo. Si studia la variaione di intensità della radiazione emessa dalla sorgente a seguito dell'interazione con il materiale interposto, che avviene attraverso assorbimenti e emissioni. Lo scattering entra nell'assorbimento, poiché l'effetto è quello di deviare la radiazione, e ciò comporta il fatto che misuro una intensità minore di quella iniziale.}
\label{fig:trasporto-radiativo}
\end{figure}

Come già anticipato nel paragrafo~\ref{sec:intro-trasporto-radiativo}, l'\emph{equazione del trasporto radiativo} descrive la variazione dell'intensità della radiazione a causa dell'interazione tra la radiazione stessa e la materia. Infatti, prima di arrivare all'osservatore, la radiazione attraversa della materia, e l'interazione con tale materia provoca assorbimento ed emissione (figura~\ref{fig:trasporto-radiativo}). L'equazione si presenta così:
\begin{equation}\label{eq:trasporto-radiativo}
    \ud I_\lambda = - k_\nu \rho I_\nu \ud S + j_\nu \ud S
\end{equation}
dove appaiono i seguenti termini:
\begin{description}
    \item[$\ud I_\nu$] variazione dell'intensità specifica
    \item[$k_\nu$] coefficiente di assorbimento per unità di massa o opacità ($\si{cm^2.g^{-1}}$). L'opacità ha le dimensioni di una sezione d'urto per unità di massa ed esprime quanta radiazione viene assorbita per unità di massa e per lunghezza attraversata.
    \item[$\rho$] densità di massa del materiale attraversato
    \item[$I_\nu$] intensità in input
    \item[$\ud S$] dimensione infinitesima del materiale attraversato
    \item[$j_\nu$] coefficiente di emissione (\si{erg.s^{-1}.cm^{-3}.Hz^{-1}-sr^{-1}})
\end{description}

In genere si definisce:
\[
\alpha_\nu = k_\nu \rho
\]
e l'equazione assume la seguente forma:
\begin{equation}\label{eq:trasporto-radiativo2}
    \frac{\ud I_\nu}{\ud S} = -\alpha_\nu I_nu + j_\nu
\end{equation}

Si può risolvere l'equazione immediatamente in due casi particolarmente semplici. Nel caso si \emph{pura emissione}, corrispondente a $\alpha_\nu = 0$ si ha:
\[
    \frac{\ud I_\nu}{\ud S} = j_nu \qquad I_\nu = I_\nu (0) + \int \ud S j_\nu
\]
In questo caso l'aumento di densità è pari al coefficiente di emissione integrato lungo la linea di vista.

Nel caso di \emph{puro assorbimento}, corrispondente a $j_\nu = 0$, si ha:
\[
\frac{\ud I_\nu}{\ud S} = -\alpha_\nu I_\nu \quad 
I_\nu (S = I_\nu (0) e^{-\int \alpha_\nu \ud S}
\]
In questo caso l'intensità cala come l'esponenziale del coefficiente di assorbimento intergato lungo la linea di vista.

È conveniente introdurre la \emph{profontià ottica}:
\begin{equation}\label{eq:profondità-ottica}
        \tau_\nu = \int k_\nu \rho \ud S
\end{equation}
dove sto integrando lungo il cammino $\ud S$. La profondità ottica in generale dipende anche dalla frequenza ed è un numero puro. 

È conveniente introdurre anche la \emph{funzione sorgente}:
\begin{equation}\label{eq:funzione-sorgente}
    S_\nu = \frac{j_\nu}{k_\nu \rho} = \frac{j_\nu}{\alpha_\nu}
\end{equation}
Essa esprime il rapporto tra il coefficiente di emissione e quello di assorbimento in termini di $\alpha$ e ha le stesse unità di misura di una intensità specifica.

Con queste due nuove variabili, l'equazione assume una forma particolarmente semplice:
\begin{equation}\label{eq:trasporto-radiativo3}
    \frac{\ud I_\nu}{\ud \tau_\nu} = -I_\nu + S_\nu
\end{equation}
Sviluppando i conti, come mostrato in appendice~\ref{app:trasporto-radiativo} integrando tale equazione si ottiene una soluzione per l'equazione~\eqref{eq:trasporto-radiativo}:
\begin{equation}\label{eq:soluzione-trasporto-radiativo}
    I_\nu = I_{\nu_0} e^{-\tau_\nu} + S_\nu (1- e^{-\tau_\nu})
\end{equation}
Il primo termine è l'intensità iniziale $I_{\nu_0}$(prima dell'interazione con la materia) attenuata da un fattore esponenziale (si ricordi che $\tau \in [1,+\infty]$), mentre il secondo termine \emph{non} dipende dall'intensità iniziale (quella emessa dalla sorgente), ma riguarda solamente il materiale interposto\footnote{Il materiale interposto può anche essere l'atmosfera della stella stessa che si osserva, oppure potrebbe essere l'atmosfera della Terra e così via.}. Trascurando il primo termine, questo secondo termine, che rappresenta l'emissione e l'assorbomento, può anche rappresentare una autoemissione e un autoassorbimento nel caso in cui non ci sia una stella "dietro".

\subsection{Trascuro la sorgente di background}\label{sec:trascuro-sorgente-trasporto-radiativo}
Se nell'equazione~\eqref{eq:soluzione-trasporto-radiativo} trascuro la sorgente di background, assume la forma:
\begin{equation}\label{eq:soluzione-no-sorgente-trasporto-radiativo}
    I_\nu = S_\nu (1- e^{-\tau_\nu})
\end{equation}
Questo corrisponde, ad esempio, a osservare il mezzo interposto da una direzione perpendicolare rispetto alla stella, come mostrato in figura~\ref{fig:trasporto-radiativo-perpendicolare}. Il primo termine, $S_\nu$ rappresenta l'emissione, mentre il secondo termine, $S_\nu e^{-\tau_\nu}$ rappresenta l'auto-assorbimento. Si può studiare questo caso considerando due situazioni limite:
\begin{description}
    \item[Regime otticamente spesso] la profondità ottica è molto grande, $\tau_\nu \gg 1$ , quindi l'intensità coincide con la funzione sorgente: $I_\nu = S_\nu$
    \item[Regime otticamente sottile] la profondità ottica è molto piccola, $\tau_\nu \ll 1$, quindi l'intensità risulta: $I_\nu = S_\nu \tau_\nu$
\end{description}

\begin{figure}
\centering
\includegraphics[width=0.4\textwidth]{immagini/trasporto-radiativo-perpendicolare.png}
\caption{Trascuro la sorgente di background nella soluzione dell'equazione del trasporto radiativo.}
\label{fig:trasporto-radiativo-perpendicolare}
\end{figure}

Ci si può chiedere come sia fatta la funzione sorgente $S_\nu$, definita tramite eq.~\eqref{eq:funzione-sorgente}. Essa è diversa a seconda del tipo di radiazione, cioè del processo responsabile del fenomeno di radiazione che sto osservando. Tipicamente i processi di emissione vengono raggruppati in processi termici e processi non termici.
\begin{description}
    \item[Radiazione termica] emessa da un corpo in equilibrio termico (o termodinamico). Ovvero materia e radiazione sono accoppiate, ci sono continui processi di emissione e assorbimento di fotoni e il tasso di assorbimento è uguale al tasso di emissione. Esempi di processi termici sono:
    \begin{itemize}
        \item Radiazione di corpo nero (stelle)
        \item Bremsstrahlung (gas negli ammassi di galassie)
    \end{itemize}
    
    \item[Radiazione non termica] non sono dovuti a processi di emissione e assorbimento, ma, ad esempio, da moto di cariche elettriche oppure da scattering. Esempio sono:
    \begin{itemize}
        \item Sincrotrone (pulsar)
        \item Compton (effetto Sinyaev-Zel'dovich)
    \end{itemize}
\end{description}

\subsection{Radiazione termica}
Se, nell'approssimazione del paragrafo~\ref{sec:trascuro-sorgente-trasporto-radiativo} considero solamente processi termici, ovvero processi dovuti a continue emissioni e assorbimenti, la funzione sorgente coincide con la \emph{funzione di Planck} (eq.~\eqref{eq:corpo-nero}), che verrà introdotta nel prossimo paragrafo.
\[
    S_\nu = B_{BB} (T)
\]
In questo caso, i due regimi di approssimazione si riducono alle seguenti espressioni:
\begin{description}
    \item[regime otticamente spesso ($\tau_\nu \gg 1$)] $I_\nu = B_{BB}$
    \item[regime otticamente sottile ($\tau_\nu \ll 1$)] $I_\nu = B_{BB} \tau_\nu$
\end{description}
È evidente che, se il regime è \emph{otticamente spesso}, la radiazione termica è \emph{radiazione di corpo nero}.

\section{Corpo nero}\label{sec:corpo-nero}

\subsection{Legge di Planck}\label{sec:legge-planck}
Introdotto da Planck alla fine del 1800, un \emph{corpo nero} è un corpo idealizzato in equilibrio termodinamico che assorbe tutta la radiazione incidente ed emette uno spettro che dipende solo dalla temperatura superficiale $T$ del corpo stesso. Il tasso di assorbimento e di emissione è lo stesso e la forma dello spettro del corpo segue la legge di Planck:(fig.~\ref{fig:corpo-nero}):
\begin{equation}\label{eq:corpo-nero}
    B_{BB} (T) = \frac{2 h}{c^2} \frac{\nu^3}{e^{\frac{h \nu}{k_B T}} - 1}
\end{equation}
La planckiana dipende sia dalla \emph{frequenza} che dalla \emph{temperatura} del corpo. Le stelle in prima approssimazione si possono considerare dei corpi neri, come si può osservare da un confronto tra dati sperimentali e curve teoriche in figura~\ref{fig:stelle-corpi-neri}. Guardando come sono fatte le curve a diverse lunghezze d'onda si può inferire la \emph{legge di spostamento di Wien}.

\begin{figure}
\centering
\includegraphics[width=0.5\textwidth]{immagini/corpo-nero.png}
\caption{Legge di Planck. Curve in funzione della frequenza plottate a diverse temperature. I picchi seguono la legge di spostamento di Wien.}
\label{fig:corpo-nero}
\end{figure}

\begin{figure}
\centering
\includegraphics[width=0.4\textwidth]{immagini/stelle-corpi-neri.png}
\caption{Confronto tra i profili teorici di planckiana e dati sperimentali sulle stelle. In prima approssimazione le stelle possono essere trattate come corpi neri.}
\label{fig:stelle-corpi-neri}
\end{figure}

\subsection{Legge di spostamento di Wien}\label{sec:legge-wien}
Più elevata è la temperatura $T$ del corpo, più il picco di emissione si trova a basse lunghezze d'onda, corrispondenti ad alte frequenze (fig.~\ref{fig:corpo-nero}):
\begin{equation}\label{eq:legge-wien}
    T \lambda_\textup{max} = \SI{0.29}{cm.K}
\end{equation}
Da questa legge è immediato capire perché vediamo il nostro sole di colore giallo. Infatti, la sua temperatura sua temperatura superficiale è circa $T \sim \SI{5770}{K}$, da cui una lunghezza d'onda sul picco corrispondente a $\lambda_\textup{max} \sim \SI{0.5}{\mu m}$. Si può esprimere la legge anche in funzione della frequenza, come mostrato in figura~\ref{fig:legge-wien}, ed esprime che più elevata è la temperatura del corpo, più il picco di emissione si trova ad alte frequenze (basse lunghezze d'onda). Quindi, se suppongo che una sorgente che sto osservando sia un corpo nero, effettuando misure a lunghezze d'onda diverse, in base alla frequenza a cui osservo il picco di emissione, posso risalire alla temperatura superficiale della stella. In particolare, una \emph{stella blu} sarà \emph{calda}, mentre una \emph{stella rossa} sarà \emph{fredda}.

\begin{figure}
\centering
\includegraphics[width=0.3\textwidth]{immagini/legge-wien.png}
\caption{Legge di Wien vista in funzione della frequenza. All'aumentare della temperatura, il picco di emissione si sposta verso frequenze più alte.}
\label{fig:legge-wien}
\end{figure}

\subsection{Legge di Stefan-Boltzmann}\label{sec:legge-stefan-boltzmann}
All'aumentare della temperatura $T$, aumenta anche l'intensità del corpo nero, come si può osservare nell'equazione~\eqref{eq:corpo-nero}. Questo significa che un corpo nero più caldo emetterà più energia al secondo, su tutte le lunghezze d'onda, infatti basta integrare su tutte le lunghezze d'onda, ovvero basta guardare l'area sotto la curva a fissata temperatura nella figura~\ref{fig:corpo-nero}. Ma l'energia totale emessa integrata su tutte le frequenze (par.~\ref{sec:luminosità}) è proprio la luminosità bolometrica $L_\textup{bol}$ del corpo. Questo suggerisce che debba esistere una relazione tra $T$ e $L_\textup{bol}$.

Gli esperimenti di \emph{Josef Stegan} (1879) hanno mostrato che la luminosità bolometrica $L_\textup{bol}$ di un corpo nero di area $A$ alla temperatura $T$, misurata in Kelvin, è data da:
\[
    L_\textup{bol} = A \sigma T^4
\]
dove $\sigma$ è la \emph{costante di Stefan-Boltzmann} e vale:
\[
    \sigma = \SI{5.67e-5}{erg.s^{-1}.cm^{-2}.K^{-4}}
\]
Dunque, data una stella di raggio $R$ e \emph{temperatura superficiale} T:
\begin{equation}\label{eq:stefan-boltzmann}
    L_\textup{bol} = 4 \pi R^2 \sigma T^4
\end{equation}
Inserendo i dati del Sole
\[
    R \sim \SI{7e10}{cm} \qquad T \sim \SI{5770}{K} \qquad \sigma \sim \SI{5.67e-5}{erg.s^{-1}.cm^{-2}.K^{-4}}
\]
si ottiene
\[
    \si{\solarluminosity} \sim \SI{3.8e-33}{erg.s^{-1}}
\]
Si può utilizzare la legge~\eqref{eq:stefan-boltzmann}, ad esempio, per ricavare il raggio di una stella, conoscendone la temperatura superficiale e la luminosità.
\section{Flusso}\label{sec:flusso}
L'intensità (par-~\ref{sec:intensità}) e la luminosità (par-~\ref{sec:luminosità}) sono quantità \emph{intrinseche} della sorgente, ma \emph{non} sono osservabili. Infatti, più una sorgente è lontana da noi, più ci appare debole, ma la sua luminosità non cambia con la distanza, poiché l'energia emessa ogni secondo è sempre la stessa. Ciò che osserviamo è il \emph{flusso}, definito come la luminosità per unità di area: l'energia per unità di tempo per unità di area che arriva nel nostro strumento. Ovviamente il flusso dipende dalla distanza: diminuisce all'aumentare della distanza dalla sorgente.

Ipotizzando una emissione isotropa, il flusso misurato a una distanza $r$ da una sorgente che emette con una luminosità $L$ è dato da:
\begin{equation}\label{eq:flusso}
    F(r) = \frac{L}{4 \pi r^2}
\end{equation}
Come nei casi già visti, è possibile definire un \emph{flusso monocromatico}, per unità di frequenza, $F_\nu$, e trovare il flusso totale come:
\[
    F = \int_0^\infty F_\nu \ud \nu
\]
Il flusso e la luminosità delle sorgenti sono spesso espressi in termini di \emph{magnitudini}.
\section{Magnitudine}\label{sec:magnitudine}
\subsection{Magnitudine apparente}\label{sec:magnitudine-apparente}
La \emph{magnitudine apparente} è una misura del flusso ricevuto dallo strumento. È una grandezza introdotta per la prima volta dall'astronomo greco \emph{Ipparco} e varia tra $m=1$, che corrisponde all'oggetto più brillante del cielo, e $m=6$, che corrisponde alle stelle più deboli che si possono osservare a occhio nudo. Si faccia attenzione al fatto che \emph{maggiore è la magnitudine, più debole è la stella}. In particolare, la scala di magnitudini segue la \emph{risposta logaritmica} dell'occhio umano alla luminosità delle stelle che osserviamo sulla volta celeste.

\emph{Norman Robert Pogson} (1856) notò che una magnitudine di $m=6$ corrisponde a una stella $\sim 100$ volte più debole di una con magnitudine $m=1$. Si può quindi osservare che una differenza di $5$ magnitudini corrisponde a un rapporto tra i flussi di $100$:
\[
    m_1=1, \,  m_2=6 \implies F_1 = 100 F_2
\]
\[
    m_2 - m_1 = 5 \implies \frac{F_1}{F_2} = 100 = 100^{\frac{m_2 - m_1}{5}}
\]
Quindi risulta:
\begin{equation}\label{eq:magnitudine-apparente}
    \frac{F_1}{F_2} = 100^{\frac{m_2 - m_1}{5}}
\end{equation}
da ciò segue che una differenza di $1$ magnitudine corrisponde a un rapporto tra i flussi pari a $100^{1 / 5} \sim 2.5$. Quindi, ad esempio, una stella di magnitudine $m=1$ \emph{appare} $2.5$ volte più brillante di una stella di magnitudine $m=2$ e $2.5^5 \sim 100$ volte più brillante di una stella con magnitudine $m=6$. Si parla dunque di magnitudini \emph{apparenti}.

\subsection{Legame tra magnitudine apparente e flusso}\label{sec:relazione-magnitudine-apparente-flusso}
Prendendo il logaritmo (sia $\log$ il logaritmo in base $10$) di entrambi i membri dell'equazione~\eqref{eq:magnitudine-apparente}, si trova:
\[
    \log\frac{F_1}{F_2} = \frac{m_2 - m_1}{5} \log(100) = \frac{2}{5} (m_2 - m_1) = 0.4 (m_2 - m_1)
\]
Da cui segue:
\begin{equation}\label{eq:relazione-magnitudine-apparente-flusso}
    m_1 - m_2 = -2.5 \log\frac{F_1}{F_2}
\end{equation}
Ovvero, considerando i flussi monocromatici:
\begin{equation}\label{eq:relazione-magnitudine-apparente-flusso-monocromatico}
    m_1\nu - m_2\nu = -2.5 \log\frac{F_1\nu}{F_2\nu}
\end{equation}

Misurando il flusso di una sorgente, \emph{non} conosco la sua magnitudine, poiché, come ovvio dall'equazione~\eqref{eq:relazione-magnitudine-apparente-flusso}, posso solamente conoscere la magnitudine \emph{rispetto a un'altra sorgente}. È dunque necessario assegnare un valore di magnitudine apparente fisso attraverso un \emph{sistema fotometrico}. Un esempio è il \emph{sistema fotometrico di Vega}, secondo il quale impongo che la stella Vega abbia magnitudine $m=0$ a tutte le frequenze, sicché la relazione tra magnitudine e flusso diventeranno:
\[
    m_\nu = - 2.5 \log\frac{F_nu}{F_0\nu}
\]
dove $m_\nu$ e $F_\nu$ sono rispettivamente la magnitudine e il flusso della sorgente a una data frequenza $\nu$, mentre $F_0\nu$ è il flusso della stella Vega a quella data frequenza. Per farsi un'idea degli ordini di grandezza nel sistema di Vega si faccia riferimento alla tabella~\ref{tab:sistema-fotometrico-vega}.

\begin{table}
\caption{Magnitudini apparenti nel Sistema Fotometrico di Vega. Il Sole, essendo la stella a noi più vicina, è quella che ci \emph{appare} più luminosa, infatti ha la magnitudine assoluta più piccola possibile, ovvero più negativa possibile in questo caso. Il flusso dipende sia dalla luminosità della sorgente sia dalla distanza della sorgente dall'osservatore.}
\label{tab:sistema-fotometrico-vega}
\centering
\begin{tabular}{lS}
\toprule
Oggetto & {Magnitudine apparente} \\
\midrule
Sole         & -26.8 \\
Luna piena   & -12.6 \\
Venere       & -4.7 \\
Giove, Marte & -2.8 \\
Sirio & -1.5 \\
Vega & 0.0 \\
Uranio & 5.5 \\
Più deboli visibili a occhio & 6.0 \\
Nettuno & 7.7 \\
Plutone & 13.0 \\
\bottomrule
\end{tabular}
\end{table}

\subsection{Magnitudine assoluta}\label{sec:magnitudine-assoluta}
Si definisce \emph{magnitudine assoluta} la magnitudine apparente che avrebbe una sorgente se fosse a una distanza di $\SI{10}{parsec}$ ($\SI{1}{pc} = \SI{3.086e18}{cm}$):
\begin{equation}\label{eq:magnitudine-assoluta}
    M = m_{\SI{10}{pc}}
\end{equation}
Se la magnitudine apparente \emph{non} dà informazioni sulla brillantezza intrinseca della sorgente, essendo dipendente dalla distanza della sorgente stessa, la magnitudine è legata alla luminosità intrinseca della sorgente.

Si può ricavare la relazione che sussiste tra la magnitudine assoluta e la luminosità utilizzando l'equazione~\eqref{eq:relazione-magnitudine-apparente-flusso}
\[
    M_1 - M_2 = m_{1, \SI{10}{pc}} - m_{2, \SI{10}{pc}} = -2.5 \log\frac{F_1 (\SI{10}{pc})}{F_2 (\SI{10}{pc})}
\]
e utilizzando l'equazione~\eqref{eq:flusso} per scrivere
\[
    \frac{F_1 (\SI{10}{pc})}{F_2 (\SI{10}{pc})} = \frac{L_1}{L_2}
\]
Da ciò segue che:
\begin{equation}\label{eq:magnitudine-assoluta-luminosità}
    M_1 - M_2 = -2.5 \log\frac{L_1}{L_2}
\end{equation}

\subsection{Modulo di distanza}
Conoscendo la magnitudine apparente e la magnitudine assoluta di una sorgente, è possibili stimare la sua distanza attraverso il così detto \emph{modulo di distanza}. Troviamo la relazione tra $m$ e $M$ utilizzando l'equazione~\eqref{eq:relazione-magnitudine-apparente-flusso}
\[
    m - M = m - m_{\SI{10}{pc}} = -2.5 \log\frac{F(d)}{F(\SI{10}{pc})}
\]
e usando l'equazione~\eqref{eq:flusso} con $r=d$
\[
    \frac{F(d)}{F(\SI{10}{pc})} = \frac{10^2}{d^2}
\]
mettendo insieme si trova
\[
    m - M = -5 \log\frac{10}{d} = -5 + 5 \log d
\]
Da cui l'espressione per il \emph{modulo di distanza}:
\begin{equation}\label{eq:modulo-distanza}
    m - M = -5 + 5 \log(d_{\si{pc}})
\end{equation}
dove con il pedice si è sottolineato che le distanze sono espresse in \emph{parsec}.

Siccome siamo in grado di misurare la distanza del Sole e possiamo misurare la sua magnitudine apparente a varie lunghezze d'onda, utilizzando l'equazione~\eqref{eq:modulo-distanza} è possibile ricavare le magnitudini assolute del sole a varie lunghezze d'onda, ad esempio si ha (B, V e K sono tre differenti filtri fotometrici):
\[
    M_\ensuremath{\textup{bol}\odot} = 4.75 \quad M_\ensuremath{\textup{B}\odot} = 5.48 \quad M_\ensuremath{\textup{V}\odot} = 4.83 \quad M_\ensuremath{\textup{K}\odot} = 3.31
\]
Queste sono utilizzate come riferimento per esprimere la luminosità e la magnitudine assoluta per ogni altra sorgente. Infatti, richiamando l'equazione~\eqref{eq:magnitudine-assoluta-luminosità} e imponendo $M_2 = \si{\solarmass}$ e $L_2 = \si{\solarluminosity}$, si ottiene
\begin{equation}\label{eq:magnitudine-assoluta-luminosità-sole}
    M_1 - \si{\solarmass} = -2.5 \log\frac{L_1}{\si{\solarluminosity}}
\end{equation}
\section{Colore}\label{sec:colore}
\subsection{Definizione di colore}
L'\emph{indice di colore} è la differenza tra le magnitudini misurate in due diverse bande fotometriche (\emph{filtri}). In particolare, le immagini vengono acquisite attraverso filtri che selezionano solamente alcune lunghezze d'onda. Ciascun filtro ha la propria curva di trasmissione, centrata su una data lunghezza d'onda. D'altra parte, ciascuna sorgente ha il proprio spettro di emissione. Quindi, i diversi filtri selezionano solo una frazione dello spettro della sorgente, quella inclusa sotto la loro curva di trasmissione (fig.~\ref{fig:filtri-fotometrici}). Ne conviene che il flusso misurato durante l'osservazione in un dato filtro dipende sia dal filtro che dallo spettro della sorgente.

\begin{figure}
\centering
\includegraphics[width=0.7\textwidth]{immagini/filtri-fotometrici.png}
\caption{Osservazione di due spettri diversi con due filtri, uno blu (B) e uno rosso (R). In verde sono rappresentati gli spettri analizzati e le curve in blu e rosso sono rispettivamente le curve di trasmissione del filtro blu e rosso. Le zone evidenziate rappresentano il flusso misurato. Si nota che a sinistra vale $F_B > F_V$ mentre a destra $F_B < F_V$. Ciò evidenzia come il flusso misurato dipende sia dallo spettro della sorgente che dal filtro utilizzato.}
\label{fig:filtri-fotometrici}
\end{figure}

Ricordando l'equazione~\eqref{eq:relazione-magnitudine-apparente-flusso}, si può scrivere:
\[
    m_B = -2.5 \log F_B + \textup{const}
\]
\[
    m_V = -2.5 \log F_V + \textup{const}
\]
dove $m_B$ e $m_V$ sono le magnitudini apparenti rispettivamente in filtro blu e filtro rosso (analogo per i flussi misurati $F_B$ e $F_V$). Si definisce \emph{colore} la differenza tra la magnitudine misurata col filtro più blu e la magnitudine misurata col filtro più rosso:
\begin{equation}\label{eq:colore}
    m_B - m_V \equiv B-V
\end{equation}
Il colore, dunque, è un numero puro, e in tab.~\ref{tab:colore} sono presenti le relazioni tra il colore, la magnitudine assoluta e il flusso in filtro blu e rosso. 

\begin{table}
\caption{Relazione tra colore, magnitudine apparente e flusso.}
\label{tab:colore}
\centering
\begin{tabular}{lll}
\toprule
Colore & Magnitudine apparente & Flusso \\
\midrule
$<0$ & $m_B < m_V$ & $F_B > F_V$ \\
$=0$ & $m_B = m_V$ & $F_B = F_V$ \\
$>0$ & $m_B > m_V$ & $F_B < F_V$ \\
\bottomrule
\end{tabular}
\end{table}

Ovviamente, cambiamo i filtri, cambia anche la denominazione del colore. Si tenga inoltre in considerazione che il colore \emph{non} dipende dalla distanza, poiché usando le equazioni~\eqref{eq:relazione-magnitudine-apparente-flusso} e~\eqref{eq:flusso} si ottiene che la differenza tra le magnitudini apparenti e le magnitudini assolute è uguale:
\begin{equation}\label{eq:differenza-magnitudini-apparente-assoluta}
    m_B-m_V = M_B-M_V
\end{equation}

\subsection{Colore e temperatura}
Ci si può chiedere quale sia la ragione fisica alla base del diverso colore delle stelle. Ripensando all'esempio in fig.~\ref{fig:filtri-fotometrici} è evidente che una stella è tanto più blu quanto maggiore è il flusso misurato a basse lunghezze d'onda. Dunque, in definitiva, la ragione fisica del diverso colore è la stessa del diverso spettro. Come introdotto nel par.~\ref{sec:legge-planck} ed evidenziato in fig.~\ref{fig:stelle-corpi-neri}, le stelle in prima approssimazione sono dei corpi neri, dunque lo spettro è definito dall'equazione di Planck~\eqref{eq:corpo-nero}. Utilizzando un particolare filtro, si sta fissando la lunghezza d'onda (o frequenza) di picco a qui lavora tale filtro, dunque lo spettro osservato di una stella dipenderà solamente dalla sua \emph{temperatura superficiale}. Ricordando la relazione tra intensità e luminosità monocromatica~\eqref{eq:luminosità-monocromatica-intensità} e imponendo per una stella $I_\nu = B_\nu (T)$ si ottiene:
\[
    L_\nu = 4 \pi R^2 \pi B_\nu (T)
\]
da cui, attraverso la relazione~\eqref{eq:magnitudine-assoluta-luminosità}, usando~\eqref{eq:differenza-magnitudini-apparente-assoluta}, si ottiene
\[
    m_{\lambda 1} - m_{\lambda 2} = M_{\lambda 1} - M_{\lambda 2} = -2.5 \log\frac{L_{\lambda 1}}{L_{\lambda 2}} + \textup{const} = -2.5 \log\frac{B_{\lambda 1}(T)}{B_{\lambda 2}(T)} + \textup{const}
\]
da cui si può inferire che, nel limite di validità dell'approssimazione della stella come corpo nero, il colore dipende solamente dalla temperatura.

Per riassumere, più bassa è la temperatura superficiale, più rossa è la stella, che equivale a un grande $(B-V)$, mentre più alta è la temperatura superficiale, maggiore è il flusso a basse $\lambda$ (cfr. fig.~\ref{fig:corpo-nero}) e più blu è la stella, che equivale a un piccolo $(B-V)$.

\subsection{Estinzione}\label{sec:reddening}
In generale, a causa dei processi d'interazione tra radiazione e materia nel cammino tra la sorgente e l'osservatore, una stella tende ad apparire \emph{più debole} e \emph{più rossa} di come non sia veramente, questo fenomeno prende il nome di \emph{estinzione} (o arrossamento). È possibile esprimere tale fenomeno analiticamente, introducendo un parametro di correzione alla magnitudine "intrinseca", ovvero quella che vedremmo se non ci fosse l'interazione con il mezzo interstellare.
\begin{equation}\label{eq:estinzione}
    m_\lambda = m_{\lambda 0} + A_\lambda
\end{equation}
dove $m_\lambda$ è la magnitudine osservata, $m_{\lambda 0}$ quella intrinseca e $A_\lambda$ prende il nome di \emph{parametro di estinzione}. Esso è relativo alla direzione in cui viene osservata la stella e dipende fortemente dalla lunghezza d'onda, in particolare $A_\lambda$ cresce al diminuire di $\lambda$. La dipendenza di $A_\lambda$ da $\lambda$, detta \emph{legge di estinzione}, è tuttavia scarsamente nota: dipende infatti dalle proprietà del mezzo interstellare lungo la linea di vista, quindi sicuramente cambia al variare della linea di vista, e cambia al variare della galassia che si sta considerando.

È presente una legge di estinzione standard per la Via Lattea, derivata da \emph{Cardelli} (1989), tuttavia probabilmente non è valida ovunque nella nostra galassie e non è chiaro se debba valere anche per altre galassie. L'unica cosa chiara è che l'effetto dell'estinzione cresce al diminuire delle lunghezze d'onda.

Per spiegare l'effetto del \emph{reddening} consideriamo due filtri particolari, un B e un V, e utilizziamo l'equazione~\eqref{eq:estinzione} in combinazione con la~\eqref{eq:colore}. Possiamo scrivere:
\[
    m_\lambda - m_{\lambda 0} = A_\lambda
\]
\[
    B - B_0 = A_B \qquad V - V_0 = A_V
\]
e sottraendo membro a membro:
\[
    B_0 - V_0 \equiv (B-V)_0 = (B-V) - (A_B - A_V)
\]
Il primo termine, $(B-V)_0$, rappresenta il \emph{colore intrinseco}, $(B-V)$ rappresenta il \emph{colore osservato}, mentre l'ultimo termine, $(A_B-A_V) \equiv \EBV$ rappresenta il \emph{reddening}, ovvero l'eccesso di colore. Partendo dal reddening dei filtri B-V si può indicare l'effetto per dei filtri qualunque, moltiplicando per un opportuno fattore correttivo $R_\lambda$:
\begin{equation}\label{eq:reddening}
    m_\lambda = m_{\lambda 0} + A_\lambda = m_{\lambda 0} + R_\lambda \EBV
\end{equation}

\subsection{Modulo di distanza e reddening}
Si faccia attenzione al fatto che la magnitudine apparente che compare nell'espressione del modulo di distanza~\eqref{eq:modulo-distanza} è quella \emph{vera}, ovvero de--arrossata. Talvolta, tuttavia, in letteratura si trova il modulo di distanza \emph{osservato}, ad esempio, in banda V. In questo caso, per trovare la distanza, è necessario de--arrossare:
\[
    (m - M)_V = -5 + 5 \log(d_{\si{pc}}) + 3.12 \, \EBV
\]
Si faccia attenzione anche a un'ultima cosa: la magnitudine assoluta, per definizione, è sempre quella vera. Non ha senso parlare di magnitudine assoluta arrossata.




%\chapterimage{}
\chapter{Struttura stellare}
\section{Modelli stellari}\label{sec:modelli-stellari}
Una \emph{stella} è una sfera auto-gravitante di gas in equilibrio idrostatico, in cui, dunque, la forza di pressione del gas eguaglia la forza di gravità. I parametri principali con cui si descrive una stella sono la sua \emph{massa} $M$ e la sua \emph{composizione chimica}. Di seguito è riportato il modo usuale in cui si esprime quest'ultima:
\begin{description}\label{tab:composizione-chimica}
    \item[X] Frazione in massa dell'\emph{idrogeno}.
    \item[Y] Frazione in massa dell'\emph{elio}.
    \item[Z] Frazione in massa degli elementi più pesanti dell'elio, ovvero dei \emph{metalli}.
\end{description}
Per il Sole, ad esempio, si ha:
\[
    X=0.70 \qquad Y=0.28 \qquad Z=0.02
\]

Una stella è caratterizzata da tre regioni principali:
\begin{description}
    \item[Nucleo] Dove viene prodotta l'energia.
    \item[Inviluppo] Dove l'energia è trasportata in superficie. Vi sono zone radiative e convettive.
    \item[Atmosfera] è a sua volta suddivisa in tre zone
    \begin{description}
        \item[Fotosfera] Emette la maggior parte della luce. Sotto la fotosfera la stella è opaca.
        \item[Cromosfera] Ciò che vediamo durante un'eclissi.
        \item[Corona] Strato più esterno della stella. Può essere visto durante un'eclissi.
    \end{description}
\end{description}

Di una stella, osserviamo la magnitudine e il colore che vengono dall'atmosfera e per interpretare le misurazioni e inferire la sua struttura interna, è necessario un \emph{modello stellare}. Esso prenderà in input la massa e la composizione chimica e darà in output la luminosità e la temperatura superficiale. Queste, tuttavia, non sono direttamente osservabili, quindi, dovrò tenere conto di tutti i fenomeni fisici visti in precedenza e ricondurmi a tali grandezze attraverso la \emph{magnitudine} e il \emph{colore} per confrontare teoria e dati sperimentali. Di seguito sono esposte le sette equazioni necessarie per un modello stellare esaustivo:
\begin{enumerate}
    \item \emph{Equilibrio idrostatico}:
    \[
    \dfrac{\ud P(r)}{\ud r} = -\dfrac{G M(r)}{r^2} \rho(r)
    \]
    \item \emph{Continuità di massa}:
    \[
    \dfrac{\ud M(r)}{\ud r} = 4 \pi r^2 \rho(r)
    \]
    \item \emph{Equazione di stato}:
    \[
    P = \dfrac{aT^4}{3} + \dfrac{k \rho T}{\mu_i H} + 
    \begin{cases} 
    \frac{k \rho T}{\mu_e H} \\ 
    k_1 \rho^{5/3} \\ 
    k_2 \rho^{4/3}
    \end{cases}
    \]
    \item \emph{Bilancio energetico}:
    \[
    \dfrac{\ud L(r)}{\ud r} = 4 \pi r^2 \rho(r) \epsilon
    \]
    \item \emph{Gradiente radiativo e criterio di Schwarzschild}:
    \[
    \dfrac{\ud T}{\ud r}\Big|_\textup{rad} = - \dfrac{3 \kappa \rho}{4 \pi r^2} \dfrac{L(r)}{4 a c T^3}
    \]
    \[
    \text{se} \quad \abs*{\dfrac{\ud T}{\ud r}}_\textup{rad} > \abs*{\dfrac{\ud T}{\ud r}}_\textup{ad} \implies \text{c'è convezione.}
    \]
    \item \emph{Opacità}:
    \[
    \kappa = \kappa(\rho, T) 
    \begin{cases} 
    \kappa_{BF} \propto 10^{25} Z(1 + X) \frac{\rho}{T^{3.5}} \\ 
    \kappa_{FF} \propto 10^{22} (X+Y)(1+X) \frac{\rho}{T^{3.5}} \\ 
    \kappa_{E} \propto 0.2 (1+X) \\ 
    \end{cases}
    \]
    \item \emph{Produzione di energia tramite reazioni termonucleari}:
    \[
    \epsilon = \epsilon(X, \rho, T) 
    \begin{cases} \epsilon_{PP} = \epsilon_1 \rho X^2 T_6^\alpha \quad \alpha \in [3.5 - 6] \\ 
    \epsilon_{CN} = \epsilon_2 \rho X X_{CN} T_6^\beta \quad \beta \in [13 - 20] \\ 
    \epsilon_{3\alpha} = \epsilon_3 \rho^2 Y^3 T_8^\gamma \quad \gamma \in [20 - 30] \\
    \end{cases}
    \]
\end{enumerate}
\section{Equilibrio idrostatico}\label{sec:equilibrio-idrostatico}
L'equazione dell'\emph{equilibrio idrostatico} esprime la condizione per cui la pressione interna del gas che compone la stella è in equilibrio con la forza di gravità data dalla massa della stella stessa. Si può riassumere nella seguente maniera:
\begin{equation}\label{eq:equilibrio-idrostatico}
    \dfrac{\ud P(r)}{\ud r} = -\dfrac{G M(r)}{r^2} \rho(r)
\end{equation}

\begin{figure}
\centering
\includegraphics[width=0.3\textwidth]{immagini/equilibrio-idrostatico.png}
\caption{Volumetto infinitesimo a distanza $r$ dal centro della stella. Sulla faccia superiore agiste la pressione che spinge verso il centro. Sulla faccia inferiore agiscono la pressione, verso l'esterno, e la forza di gravità, verso il centro.}
\label{fig:equilibrio-idrostatico}
\end{figure}

Per ricavare tale equazione dividiamo la stella in gusci sferici concentrici a temperatura e densità costanti e consideriamo un volume infinitesimo di stella a una distanza $r$ dal centro. Come mostrato in figura~\ref{fig:equilibrio-idrostatico}, sulla faccia esterna del volumetto agisce una forza di pressione verso l'interno, mentre sulla faccia interna agisce una forza di pressione verso l'esterno e la forza gravitazionale, verso l'interno. Scriviamo, dunque, l'espressione per la forza di pressione $F_p$ e la forza di gravità $F_g$ e successivamente uguagliamo tali forze. Si tenga conto che l'asse radiale è rivolto verso il centro, sicché le forse che agiscono verso l'esterno avranno segno negativo.
\[
    F_p = P(r+\ud r) \ud S - P(r) \ud S = \frac{\ud P(r)}{\ud r} \ud r \ud S
\]
\[
    F_g = \frac{GM(r)}{r^2} \rho(r) \ud r \ud S
\]
dove $M(r)$ rappresenta la massa all'interno del raggio $r$, il termine $GM(r) / r^2$ rappresenta l'accelerazione locale di gravità e il termine $\rho(r) \ud r \ud S$ rappresenta la massa all'interno del volumetto. Imponendo l'equilibrio si trova:
\[
    F_p + F_g = 0 \iff \frac{\ud P(r)}{\ud r} \ud r \ud S+ \frac{GM(r)}{r^2} \rho(r) \ud r \ud S = 0
\]
da cui segue immediatamente l'eq.~\eqref{eq:equilibrio-idrostatico}. Quindi, in ogni guscio sferico della stella a fissata distanza $r$ dal centro, la gravità è bilanciata dalla pressione interna del gas. In particolare, la gravità \emph{non} è bilanciata dalla pressione, più precisamente essa è bilanciata dal \emph{gradiente di pressione}, ovvero dalla variazione di $P$ col raggio. La pressione deve decrescere all'aumentare del raggio, sicché la pressione nel centro della stella è maggiore della pressione vicino alla sua superficie. 

Quando $F_g$ e $F_p$ \emph{non} sono bilanciate, la stella si contrae se $F_g > F_p$ o espande se $F_g < F_p$ in un \emph{tempo caratteristico} pari a:
\[
    T_d = \sqrt{\frac{2r}{g}} = \sqrt{\frac{2r^3}{GM}}
\]
\section{Continuità di massa}\label{sec:continuità-massa}
L'equazione di \emph{continuità di massa} descrive come la massa interiore della stella varia con il raggio. Si può scrivere nella seguente maniera:
\begin{equation}\label{eq:continuità-massa}
    \dfrac{\ud M(r)}{\ud r} = 4 \pi r^2 \rho(r)
\end{equation}

\begin{figure}
\centering
\includegraphics[width=0.25\textwidth]{immagini/continuita-massa.jpg}
\caption{Guscio sferico infinitesimo a distanza $r$ dal centro. All'interno del guscio, a causa della simmetria sferica, ipotizziamo che la densità e la temperatura siano costanti.}
\label{fig:continuità-massa}
\end{figure}

Per ricavarla consideriamo un guscio sferico infinitesimo di stella, come in figura~\ref{fig:continuità-massa}. Siccome all'interno di tale guscio la densità di ogni elemento è la stessa, a causa della simmetria sferica della struttura, possiamo ricavare il valore di massa all'interno di tale guscio:
\[
    \ud M(r) = 4 \pi r^2 \rho(r) \ud r
\]
da cui segue immediatamente l'eq.~\eqref{eq:continuità-massa}. Consideriamo il caso semplice di densità costante, in cui $\rho(r) = \Bar{\rho}$, si può calcolare in  maniera semplice la massa presente entro il raggio $r$:
\[
    M(r) = \int_0^{M(r)} \ud m(r) = \int_0^r \ud r' 4 \pi {r'}^2 \rho(r') = \frac{4}{3} \pi r^3 \Bar{\rho}
\]

\section{Equazione di stato}\label{sec:equazione-stato}
Guardando le equazioni~\eqref{eq:equilibrio-idrostatico} e~\eqref{eq:continuità-massa}, notiamo come queste siano 2 equazioni in 3 incognite: $P(r)$, $M(r)$ e $\rho(r)$. Per risolvere completamente il sistema è necessario stabilire un'altra relazione tra grandezze indipendenti. Una relazione tra $P$ e $\rho$ è stabilita dall'\emph{equazione di stato}:
\begin{equation}\label{eq:equazione-stato}
    P = \dfrac{aT^4}{3} + \dfrac{k \rho T}{\mu_i H} + 
    \begin{cases} 
    \dfrac{k \rho T}{\mu_e H} \quad \, \, \, \,   \textup{(gas perfetto)} \\ 
    k_1 \rho^{5/3} \quad \textup{(gas degenere non relativistico)} \\ 
    k_2 \rho^{4/3} \quad \textup{(gas degenere relativistico)}
    \end{cases}
\end{equation}

Un primo aspetto da considerare è che la pressione nell'interno delle stelle è dovuta a due contributi, ossia la \emph{pressione di radiazione} e \emph{la pressione del gas}:
\[
P = P_\textup{rad} + P_\textup{gas}
\]
\subsection{Pressione di radiazione}
I fotoni esercitano una pressione perché a ogni fotone di energia $E$ è associato un impulso $p = E / c$. Una stella si può approssimare come un corpo nero, dunque è possibile calcolare la pressione di radiazione attraverso la legge di Planck~\eqref{eq:corpo-nero}.
\[
P_\textup{rad} = \frac{4\pi}{3c}\int_0^\infty B_\nu(T) \ud \nu = \frac{1}{3} a T^4
\]
dove $a$ è una costante pari a $a = 4 \sigma / c = \SI{7.6e-15}{erg.cm^{-3}.K^{-4}}$, in cui $\sigma$ è la costante di Stefan-Boltzmann. Accontentiamoci del risultato senza ulteriori specificazioni sui dettagli del conto. Si ha:
\begin{equation}\label{eq:pressione-radiazione}
    P_\textup{rad} = \frac{1}{3} a T^4
\end{equation}
Si noti come la pressione di radiazione \emph{non} dipenda dalla densità della stella e invece dipenda fortemente dalla temperatura.

\subsection{Pressione del gas ideale}\label{sec:gas-perfetto}
A causa delle temperature elevate negli interni stellari, gli atomi sono ionizzati e possiamo pensare al gas stellare come a un \emph{plasma di ioni e elettroni}. Quindi, nella maggior parte dei casi, anche a densità e pressioni elevate, il gas stellare può essere trattato come un \emph{gas ideale}, in cui si trascurano le interazioni tra le particelle del gas e la distribuzione delle velocità è rappresentata dalla distribuzione di Maxwell. La densità del gas, in questo caso, dipende sia dalla pressione che dalla temperatura e viceversa, quindi si avrà una pressione che è funzione di densità e temperatura: 
\[
P = P (\rho, T)
\]
Per un gas ideale, inoltre, vale la nota legge:
\begin{equation}\label{eq:standard-gas-ideale}
    PV = N \kb T
\end{equation}
Utilizzando l'equazione~\eqref{eq:standard-gas-ideale} e definendo $\langle m \rangle \equiv M / N$ la massa media delle particelle del gas, e $\rho \equiv M / V$ la densità, si ottiene:
\[
P = \frac{N}{V} \kb T = \frac{N}{M} \frac{M}{V} \kb T = \frac{\kb \rho T}{\langle m \rangle}
\]
In più la massa media delle particelle del gas può essere scritta anche come $\langle m \rangle = \mu H$, dove $\mu$ è il \emph{peso molecolare medio} e $H$ rappresenta la massa di un nucleo di idrogeno, ovvero la massa del protone ($H = m_P = \SI{1.6e-24}{g}$). Mettendo tutto insieme si ottiene l'equazione di stato per un gas ideale:
\begin{equation}\label{eq:gas-ideale}
    P_\textup{gas} = \frac{\kb \rho T}{\mu H}
\end{equation}
Vediamo che la costante di proporzionalità $\rho / \mu H$ dipende dalla massa del gas, che a sua volta dipende dalla composizione chimica.

La caratteristica più importante di questa equazione è la cosiddetta \emph{termo-regolazione}: se $T$ cresce $P$ cresce, ma se $P$ cresce la stella si espande e la temperatura diminuisce, quindi si ha continuamente un bilancio fra pressione e temperatura. Si faccia tuttavia attenzione al fatto che non tutti i gas si comportano come un gas ideale (par.~\ref{sec:degenerazione}).

\subsection{Peso molecolare medio}\label{sec:peso-molecolare}
Arrivati a questo punto ci interessa legare il \emph{peso molecolare medio} $\mu$ alla composizione chimica del gas, introdotta all'inizio del par.~\ref{sec:modelli-stellari}. Come già ricordato, $\mu$ rappresenta la media delle masse delle particelle che compongono il gas, espressa in termini della massa del protone $H$:
\begin{equation}\label{eq:peso-molecolare-medio-definizione}
    \mu = \frac{\langle m \rangle}{H}
\end{equation}
D'altra parte, conoscendo la massa totale del gas $M_\textup{tot}$ e il numero totale di particelle libere $N_\textup{free}$, si può calcolare la massa media delle particelle come:
\begin{equation}\label{eq:massa-media-particelle}
    \langle m \rangle = \frac{M_\textup{tot}}{N_\textup{free}}
\end{equation}
Si faccia molta attenzione al fatto che $N_\textup{free}$ rappresenta le particelle libere, dunque, in un gas ionizzato è pari alla somma del numero degli \emph{ioni} e degli \emph{elettroni}. Esso dipende da:
\begin{itemize}
    \item la \emph{composizione chimica del gas}, ovvero da $X$, $Y$ e $Z$ (vedi par.~\ref{sec:modelli-stellari}).
    \item lo \emph{stato di ionizzazione} del gas. In particolare:
    \begin{itemize}
        \item ogni atomo \emph{neutro} contribuisce con 1 particella, ovvero l'atomo stesso.
        \item ogni atomo \emph{totalmente ionizzato} contribuisce con $1+Z_a$ particelle, ovvero il nucleo e $Z_a$ elettroni ($Z_a$ è il numero atomico).
        \item ogni atomo \emph{parzialmente ionizzato} contribuisce con un numero compreso fra $1$ e $1+Z_a$ di particelle.
    \end{itemize}
\end{itemize}
Si faccia nuovamente attenzione a due dettagli. Innanzi tutto, stiamo dividendo il gas studiato in tre parti: l'idrogeno, la cui abbondanza in massa è espressa da $X$, l'elio, la cui abbondanza è espressa da $Y$ e gli elementi più pesanti, la cui abbondanza è $Z$. Dunque, non stiamo distinguendo gli elementi più pensanti dell'elio, bensì li stiamo accorpando tutti in una stessa famiglia. Questo è lecito poiché le loro abbondanze sono ridotte in confronto ai primi due elementi della tavola periodica. Inoltre, $Z_a$ rappresenta il grado di ionizzazione del gas per un gas totalmente ionizzato, che coincide dunque con il numero di protoni del gas.

Tenendo a mente che stiamo distinguendo solamente tre specie di elementi (idrogeno, elio e più pesanti), indicizziamo la specie con l'indice $j$, con $j = 1, \, 2, \, 3$; a costo di essere ridondanti, ribadiamo che $j=1$ si riferisce all'idrogeno, $j=2$ all'elio e $j=3$ a tutti gli altri elementi. A questo punto possiamo calcolare il numero di atomi della specie $j$-esima:
\begin{equation}\label{eq:numero-atomi-specie}
    N_j = \frac{M_{\textup{tot}, j}}{m_{\textup{atomo}, j}} \simeq \frac{M_{\textup{tot}, j}}{A_j H}
\end{equation}
dove $A_j$ è il \emph{numero di massa}, ovvero il numero di nucleoni, pari alla somma dei protoni e dei neutroni. In particolare, se consideriamo l'isotopo più comune dell'idrogeno, il prozio, il suo nucleo è composto solamente da un protone, per cui $A_1 = 1$. Considerando l'elio $4$, il suo nucleo è composto da $2$ protoni e $2$ neutroni, pertanto $A_2 = 4$. Invece per gli elementi più pesanti possiamo considerare che i più stabili in genere hanno un numero di protoni circa uguale al numero di protoni (la repulsione coulombiana sfavorisce l'aumento dei protoni, quindi al crescere di $A$ avremo in genere un maggior numero di neutroni; ciò nonostante la variazione non è troppo significativa, quindi possiamo assumere che il numero di protoni e quello di neutroni in genere sia lo stesso). Se consideriamo un gas \emph{totalmente ionizzato}, gli elementi più pesanti avranno $\sim Z_a$ particelle libere (con $Z_a$ il loro numero atomico) e avremo $A_3 \sim 2 Z_a$.

Inoltre possiamo scrivere l'abbondanza in massa della specie $j$-esima utilizzando la definizione stessa, ossia: 
\begin{equation}\label{eq:definizione-abbondanza}
    X_j = \frac{M_{\textup{tot}, j}}{M_\textup{tot}}
\end{equation}
dove $X_1 = X$, $X_2 = Y$ e $X_3 = Z$, come più volte ricordato. Mettendo insieme le equazioni~\eqref{eq:numero-atomi-specie} e~\eqref{eq:definizione-abbondanza} si trova:
\begin{equation}\label{eq:numero-atomi-specie2}
    N_j =\frac{X_j M_\textup{tot}}{A_j H}
\end{equation}
Pertanto, per un gas \emph{totalmente ionizzato} possiamo calcolare il numero totale di particelle libere contando le particelle come espresso nella lista sopra e utilizzando l'equazione~\eqref{eq:numero-atomi-specie2}:
\begin{equation}\label{eq:numero-atomi-liberi}
    N_\textup{free} = \sum_{j=1}^3 N_j (1 + Z_j) = \frac{M_\textup{tot}}{H} \sum_{j=1}^3 \frac{X_j}{A_j} (1+Z_j)
\end{equation}
dove $Z_j$ rappresenta il numero atomico dell'elemento considerato, dunque per l'idrogeno $Z_1 =1$, per l'elio $Z_2 = 2$ e per gli elementi più pesanti considereremo un numero atomico generico $Z_a$, per cui $Z_3 = Z_a$\footnote{non si confonda $Z_i$ con l'abbondanza degli elementi pesanti $Z$, la quale \emph{non} ha un pedice}. 

Mettendo insieme le equazioni~\eqref{eq:peso-molecolare-medio-definizione}, \eqref{eq:massa-media-particelle} e~\eqref{eq:numero-atomi-liberi}, si ottiene:
\begin{equation}
    \mu = \dfrac{1}{\sum_{j=1}^3 \frac{X_j}{A_j} (1+Z_j)}
    \label{eq:formula-generale-mu}
\end{equation}
Sostituendo con i valori che sono stati introdotti precedentemente si ha:
\[
\mu = \dfrac{1}{\frac{X_1}{A_1}(1+Z_1) + \frac{X_2}{A_2}(1+Z_2) + \frac{X_3}{A_3}(1+Z_3)} = \dfrac{1}{\frac{X}{1}(1+1) + \frac{Y}{4}(1+2) + \frac{Z}{2 Z_a}(1+Z_a)}
\]
In definitiva, per un \emph{gas totalmente ionizzato}, considerando sia il contributo degli \emph{ioni} che il contributo degli \emph{elettroni}, si ottiene un \emph{peso molecolare medio} pari a:
\begin{equation}\label{eq:peso-molecolare-gas-ionizzato}
    \mu = \dfrac{1}{2 X + \frac{3}{4} Y + \frac{1}{2} Z}
\end{equation}

Se ora consideriamo il solo contributi degli \emph{elettroni} in un gas totalmente ionizzato, otteniamo:
\begin{equation}
    \mu_e = \dfrac{1}{X + \frac{1}{2} Y + \frac{1}{2}Z}
    \label{eq:formula-mu-elettroni}
\end{equation}
infatti se consideriamo solo il contributo degli \emph{elettroni} di ogni componente del gas non stiamo considerando il contributo degli \emph{ioni}, che altro non sono che i nuclei; perciò si avrà che ogni componente contribuisce solo con $Z_j$ particelle invece che con $1+Z_j$ e se si modifica in questo modo l'equazione~\eqref{eq:formula-generale-mu} si ottiene la formula riportata.

Ma in più possiamo usare il fatto che la somma delle frazioni in massa delle varie componenti è pari a 1, quindi si ottiene che:  
\[
X+Y+Z=1 \implies X + \frac{1}{2}(Y+Z) = \frac{1}{2}X + \frac{1}{2}(X+Y+Z) =\frac{1}{2} + \frac{1}{2}X
\]
Inserendo nell'equazione~\eqref{eq:formula-mu-elettroni} otteniamo infine:
\begin{equation}\label{eq:peso-molecolare-elettroni}
    \mu_e = \dfrac{2}{X+1}
\end{equation}

\subsection{Degenerazione}\label{sec:degenerazione}
Un gas che segue l'equazione~\eqref{eq:gas-ideale} è un gas ideale e come spiegato nel paragrafo~\ref{sec:gas-perfetto} in questo caso la struttura è \emph{termoregolata}. Nel caso in cui il gas \emph{non} sia ideale, si dice che si trova in una situazione di \emph{degenerazione}, e gli effetto quanto-meccanici diventano rilevanti. Per la \emph{componente ionica} del gas (par.~\ref{sec:peso-molecolare}) ci si può sempre mettere nell'approssimazione di gas perfetto, mentre per la \emph{componente elettronica} non sempre è così; ciò è dovuto al fatto che la massa dell'elettrone è circa mille volte inferiore alla massa del protone, dunque a densità fissata è possibile una degenerazione della sola componente elettronica del gas. Possiamo quindi studiare quest'ultima, trovando un limite inferiore che varrà sicuramente anche per la componente ionica. Ma quindi in quali situazioni bisogna preoccuparsi degli effetti quanto-meccanici?

Ricordiamo che in meccanica quantistica una particella è rappresentabile attraverso una \emph{lunghezza d'onda di de Broglie} pari a:
\begin{equation}\label{eq:lunghezza-de-broglie}
    \lambda = \dfrac{h}{\sqrt{2 m \kb T}}
\end{equation}
Se la distanza media tra le particelle è molto più grande della lunghezza d'onda di de Broglie (vedi concentrazione quantistica), allora si può ignorare la natura ondulatoria e applicare le leggi della meccanica classica. Dato un certo volume $V$, la distanza media tra le particelle è tanto più grande quando più piccola è la densità. Si può dimostrare che gli effetti quantistici sono trascurabili se:
\[
T > 2.4\cdot 10^{-22} \frac{\rho^{2/3}}{m}
\]
che rappresenta la \emph{condizione di non degenerazione} per un gas di temperatura $T$, alla densità $\rho$, costituito da particelle di massa $m$. Tale condizione tende ad essere violata, ovvero il gas tende ad essere degenere, solitamente in due casi:
\begin{itemize}
    \item ad alte densità (poiché la distanza media tra le particelle è minore)
    \item  per particelle di massa minore
\end{itemize}

Come accennato precedentemente, a causa della significativa differenza di massa tra elettrone e protone, è possibile che le condizioni della materia stellare siano tali per cui gli elettroni siano in condizione di degenerazione mentre gli ioni no. D'ora in avanti ci occuperemo solamente della degenerazione elettronica. Dato che per gli elettroni $m_e \sim \SI{e-27}{g}$ si può individuare un limite di demarcazione:
\begin{itemize}
    \item $\dfrac{T}{\rho^{2/3}} > 10^5$: in questo caso si possono trascurare gli effetti quantistici e trattare il gas come gas perfetto.
    \item $\dfrac{T}{\rho^{2/3}} < 10^5$: in questo caso si \emph{devono} considerare gli effetti quantistici per la degenerazione elettronica.
\end{itemize}

In particolare, gli effetti quantistici che entrano in causa sono il \emph{principio di indeterminazione di Heisenberg} e il \emph{principio di esclusione di Pauli}.

\subsection{Principio di indeterminazione e di esclusione}\label{sec:principio-indeterminazione}
Chiamando $\Delta x$, $\Delta y$ e $\Delta z$ le incertezze nella determinazione delle posizioni e $\Delta p_x$, $\Delta p_y$ e $\Delta p_z$ le incertezze nella determinazione dei momenti nello spazio delle fasi, il \emph{principio di indeterminazione di Heisenberg} si può scrivere:
\begin{subequations}
\label{eq:principio-indeterminazione}
\begin{align}
\Delta x \cdot \Delta p_x &\sim \hbar \\
\Delta y \cdot \Delta p_y &\sim \hbar \\
\Delta z \cdot \Delta p_z &\sim \hbar 
\end{align}
\end{subequations}
Questo principio ha come importante conseguenza quella di fissare il \emph{volume limite minimo} nello spazio delle fasi (ossia quello di una singola cella) pari a:
\begin{equation}\label{eq:volume-minimo}
(\Delta x, \Delta y,\Delta z) (\Delta p_x, \Delta p_y, \Delta p_z) \sim h^3
\end{equation}

Il \emph{principio di esclusione di Pauli} afferma che in ciascuna cella dello spazio delle fasi, definita dall'equazione~\eqref{eq:volume-minimo}, non possono stare più di due fermioni, purché di spin opposto.

Questi due principi hanno un effetto diretto sulla \emph{distribuzione di velocità (momenti)} delle particelle, e quindi in definitiva sulla dipendenza della pressione dalle proprietà del gas. 

Consideriamo un gas di elettroni in condizioni di bassa densità: potremo trattarlo come gas perfetto e seguirà la distribuzione di Maxwell. Aumentando il numero di elettroni tenendo fissa la temperatura $T$, al curva di Maxwell si abbasserà, fino a un limite in cui non posso più trascurare gli effetti quantistici. In questo caso, il numero di elettroni con quantità di moto compresa tra $p$ e $p + \ud p$, in  un dato volume $V$, non può più essere infinito come nel caso della distribuzione di Maxwell, ma sarà:
\[
n(p) \ud p \leq \frac{2V}{h^3} \ud p_x \ud p_y \ud p_z
\]
Quindi, se aumentiamo il numero di elettroni, la maggior parte delle celle con i momenti più piccoli verrà via via riempita e gli elettroni in più dovranno necessariamente occupare celle con momenti maggiori, ovvero, a parità di $T$, questi ultimi avranno energie maggiori. Questo significa che la temperatura \emph{non} è più una misura dell'energia degli elettroni. 

Nel caso di \emph{degenerazione completa}, si dice che \emph{tutti}
i valori dei momenti o livelli energetici sono occupati fino a $p_0$ ($E_0$), detto momento (energia) di Fermi. Quindi $p_0$ ($E_0$) è il massimo momento (energia) possibile per le particelle di gas. 

Mentre per il gas perfetto è $T$ a stabilire il livello energetico più probabile, l'ingresso nel regime degenere di fatto cancella la dipendenza dalla temperatura:
\[
P \neq P(T)
\]

\subsection{Distribuzione di Fermi-Dirac}
Se il volume mimimo nello spazio delle fasi, come evidenziato in eq.~\eqref{eq:volume-minimo}, è $h^3$, considerando per un fermione ci sono 2 possibili stati possibili (par.~\ref{sec:principio-indeterminazione}), il peso statistico di uno stato corrisponde a $2 / h^3$ e dunque si può esprimere il numero di elettroni con quantità di moto tra $p$ e $\ud p$ in un dato volume $V$ come:
\[
n(p) \ud p = \frac{2}{h^3} \ud p_x \ud p_y \ud p_z
\]
Poiché ci interessa il numero di elettroni tra $p$ e $\ud p$, notiamo che $\ud p_x \ud p_y \ud p_z$ è equivalente al volume compreso tra due sfere concentriche di raggio $p$ e $p + \ud p$ (vedi figura~\ref{fig:volume-infinitesimo-sfera}), cioè $4 \pi p^2 \ud p$. 

\begin{figure}
\centering
\includegraphics[width=0.3\textwidth]{immagini/volume-infinitesimo-sfera.png}
\caption{$\ud p_x \ud p_y \ud p_z$ è uguale al volume compreso tra due sfere concentriche di raggio $p$ e $p + \ud p$}
\label{fig:volume-infinitesimo-sfera}
\end{figure}

Sostituendo si trova la \emph{distribuzione di Fermi-Dirac}:
\begin{equation}\label{eq:fermi-dirac}
    N_p \ud p = \frac{8 \pi}{h^3} p^2 \ud p
\end{equation}
Si noti come essa \emph{non} dipenda da T. Per evidenziare l'emergenza del comportamento quantistico oltre la soglia di Fermi, si faccia riferimento alla figura~\ref{fig:distribuzione-fermi}, in cui è rappresentato l'indice ci occupazione dei livelli energetici $\Pi(p)$ in funzione di $p$.

\begin{figure}
\centering
\includegraphics[width=0.3\textwidth]{immagini/distribuzione-fermi.png}
\caption{Indice di occupazione dei livelli energetici $\Pi(p)$ in funzione della quantità di moto $p$. $\Pi = 1$ corrisponde a un livello totalmente occupato.}
\label{fig:distribuzione-fermi}
\end{figure}

Integrando l'equazione~\eqref{eq:fermi-dirac} possiamo ottenere relazioni importanti in funzione del momento di Fermi. Si faccia attenzione al fatto che il massimo momento possibile è $p_0$, ovvero il momento di Fermi, quindi l'integrale si fermerà a questo valore:
\[
N_e = \int_0^{p_0} \frac{8 \pi}{h^3} p^3 \ud p = \frac{8\pi}{3h^3} {p_0}^3
\]
dove $N_e$ rappresenta il numero di particelle per unità di volume. Sia $N_\textup{tot}$ il numero totale di particelle, $\langle m \rangle$ la massa media delle particelle, $M$ la massa totale e $V$ il volume. Si ha:
\[
N_e = \frac{N_\textup{tot}}{V} = \frac{M}{\langle m \rangle} \frac{1}{V} = \frac{\rho}{\mu_e H}
\]
dove è stata introdotta la densità $\rho$ e si è usata l'espressione~\eqref{eq:peso-molecolare-medio-definizione}.
Mettendo insieme le ultime due espressioni si ottiene:
\begin{equation}\label{eq:relazione-momento-fermi-densità}
    p_0 = \sqrt[3]{\dfrac{3h^3}{8\pi \mu_e H}} \rho^{1/3}
\end{equation}

Arrivati a questo punto, abbiamo tutti gli strumenti necessari per il calcolo della pressione. Come noto dalla meccanica statistica, conoscendo una distribuzione dei moduli di velocità, ipotizzando isotropia dello spazio, si può sempre trovare la pressione come:
\begin{equation}\label{eq:pressione-statistica}
    P = \frac{1}{3} m \int_0^\infty N(v) v^2 \ud v = \frac{1}{3m} \int_0^\infty N(p) p^2 \ud p
\end{equation}
e in generale bisogna distinguere tre casi:
\begin{description}
    \item[Gas perfetto:] in questo caso la distribuzione di velocità da utilizzare è quella di Maxwell-Boltzmann. Integrando la~\eqref{eq:pressione-statistica} si ottiene la legge dei gas perfetti~\eqref{eq:gas-ideale}, in cui $\mu = \mu_e$.
    \item[Gas degenere non relativistico:] la distribuzione di velocità è quella di Fermi Dirac. Si utilizza l'espressione classica per l'impulso, $p=mv$.
    \item[Gas degenere relativistico:] la distribuzione di velocità è ancora quella di Fermi-Dirac, tuttavia si utilizza l'espressione relativistica per l'impulso, $p= \frac{m_e v}{\sqrt{1- (\frac{v}{c})^2}}$
\end{description}

Vediamo cosa si ottiene negli ultimi due casi.

\subsubsection{Caso non relativistico}
In questo caso $p_0 \ll m_e c$ quindi possiamo scrivere $p=mv$ e integrare la~\eqref{eq:pressione-statistica} utilizzando la distribuzione di Fermi-Dirac~\eqref{eq:fermi-dirac}. Si ottiene:
\[
P = \dfrac{8\pi}{3h^3m} \int_0^{p_0} p^4 \ud p = \dfrac{8\pi}{3h^3m}\dfrac{{p_0}^5}{5}
\]
e sostituendo l'espressione per $p_0$ trovata precedentemente~\eqref{eq:relazione-momento-fermi-densità} otteniamo
\begin{equation}\label{eq:pressione-degenerazione-non-relativistica}
    P = k_1 \rho^{5/3}
\end{equation}
dove $k_1$ è una costante pari a:
\[
k_1 = 10^{13} {\mu_e}^{-5/3}
\]
Si noti come la pressione \emph{non} dipenda dalla temperatura. Questo fa sì che \emph{non} ci sia termoregolazione. Inoltre, la dipendenza dalla densità è maggiore rispetto che al gas perfetto (equazione~\eqref{eq:gas-ideale}).

\subsubsection{Caso relativistico}
In questo caso si ha $p_0 \sim m_e c$ e bisogna utilizzare l'espressione relativistica dell'impulso:
\[
p= \dfrac{m_e v}{\sqrt{1- (\frac{v}{c})^2}}
\]
Sostituendo dentro~\eqref{eq:pressione-statistica} si ha:
\[
P = \dfrac{8\pi}{3mh^3} \int_0^{p_0} \dfrac{p^4 \ud p}{(1 + \dfrac{p^2}{m^2c^2})^{1/2}} = \dots = \dfrac{2\pi c}{3h^3}{p_0}^4
\]
dove si sono saltati i passaggi intermedi, riportati nel \emph{Cester}. Unendo l'ultima espressione con~\eqref{eq:relazione-momento-fermi-densità} si ottiene:
\begin{equation}\label{eq:pressione-degenerazione-relativistica}
    P = k_2 \rho^{4/3}
\end{equation}
con $k_2$ una costante pari a
\[
k_2 = 1.2 \cdot 10^{15} {\mu_e}^{-4/3}
\]
Nuovamente, non c'è dipendenza dalla temperatura e la dipendenza dalla densità è più forte che nel gas ideale~\eqref{eq:gas-ideale}.

\subsection{L'equazione in breve}
L'equazione di stato negli interni stellari~\eqref{eq:equazione-stato} ha tre contributi principali. Analizziamoli celermente.

\paragraph{Pressione di radiazione}
Il termine:
\[
P = \dfrac{aT^4}{3}
\]
rappresenta il contributo della pressione di radiazione per un corpo nero, secondo la eq.~\eqref{eq:pressione-radiazione} ottenuta integrando la distribuzione di Planck~\ref{eq:corpo-nero}. È il contributo dei fotoni che compongono il gas stellare, i quali hanno un impulso e dunque esercitano una pressione.

\paragraph{Pressione degli ioni}
Il termine:
\[
\dfrac{k \rho T}{\mu_i H}
\]
è il contributo degli ioni nel materiale stellare, che pensiamo come a un plasma. Questi sono sempre approssimati come a un gas perfetto, infatti seguono la legge~\eqref{eq:gas-ideale}.

\paragraph{Pressione degli elettroni}
Il contributo:
\[
\begin{cases} 
    \dfrac{k \rho T}{\mu_e H} \\ 
    k_1 \rho^{5/3} \\ 
    k_2 \rho^{4/3}
    \end{cases}
\]
dipende dalla degenerazione del gas per ciò che concerne il contributo elettronico. Se il gas di elettroni è \emph{non} degenere, ovvero è un gas perfetto, segue la legge~\eqref{eq:gas-ideale}, dove $\mu = \mu_e$. Nel caso \emph{degenere} si distinguono due situazioni: la prima in cui si trattano le particelle come \emph{non} relativistiche, con un contributo alla pressione pari a $k_1 \rho^{5/3}$, e un secondo in cui si trattano le particelle come \emph{relativistiche}, con un contributo alla pressione pari a $k_2 \rho^{4/3}$.

\subsection{Contributo dominante}
Per descrivere lo stato della materia negli interni stellari, è utile utilizzare il \emph{diagramma} $\log \rho$ -- $\log T$. In pratica, note la densità e la temperatura di un certo strato di stella, questo piano permette di sapere quale componente di pressione prevale. Infatti, eguagliando tra loro diversi contributi di pressione, si ottengono delle rette che suddividono il piano in regioni in cui domina l'uno o l'altro contributo. Analizziamo i domini di demarcazioni per casi. Nella figura~\ref{fig:diagramma-logrho-logt} sono mostrati tutti i casi. Essa è riferita al nucleo stellare, in cui le temperature sono elevatissime, in superficie le temperature sono più basse.

\begin{figure}
\centering
\includegraphics[width=0.5\textwidth]{immagini/piano-log-log.png}
\caption{Diagramma $\log \rho$ -- $\log T$. In giallo domina la pressione di radiazione, in bianco il gas non degenere (gas perfetto), in rosso il gas degenere non relativistico e in blu il gas degenere relativistico.}
\label{fig:diagramma-logrho-logt}
\end{figure}

\paragraph{Pressione di radiazione -- Pressione del gas perfetto}
Imponiamo $P_\textup{rad} = P_\textup{gas}$, uguagliando dunque l'eq.~\eqref{eq:pressione-radiazione} con l'eq.~\eqref{eq:gas-ideale}. In pratica sto considerando insieme il contributo degli ioni e degli elettroni, utilizzando $1/ \mu_i + 1 / \mu_e = 1 / \mu$
\[
\frac{1}{3} a T^4 = \frac{\kb \rho T}{\mu H}
\]
sviluppando si trova:
\[
T^3 = \dfrac{3\kb}{a\mu H} \rho
\]
da cui, passando ai logaritmi:
\[
\log T^3 = \log \rho + c
\]
e infine si può ricavare:
\begin{equation}\label{prad-pgas}
    \log T = \frac{1}{3} \log \rho + 7.57
\end{equation}
La regione in giallo del diagramma~\ref{fig:diagramma-logrho-logt} è quella in cui la pressione di radiazione domina su quella del gas perfetto, ovvero $P_\textup{rad} > P_\textup{gas}$

\paragraph{Pressione del gas degenere -- Pressione del gas perfetto}
Per sapere quando la pressione del gas degenere di elettroni prevale su quella di gas perfetto, uguagliamo la~\eqref{eq:gas-ideale} con la~\eqref{eq:pressione-degenerazione-non-relativistica}. Siccome la condizione di degenerazione riguarda solamente il contributo elettronico, nella parte del gas perfetto considero solo gli elettroni, per cui $\mu = \mu_e$:
\[
\frac{\kb \rho T}{\mu_e H} = k_1 \rho^{5/3}
\]
da cui si ottiene:
\[
T = \frac{k_1}{\kb} \rho^{2/3} \mu_e H
\]
e, passando ai logarirtmi:
\begin{equation}\label{eq:pgas-pdeg}
    \log T = \frac{2}{3} \log \rho + 4.88
\end{equation}
La regione in rosso del diagramma~\ref{fig:diagramma-logrho-logt} è quella in cui la pressione del gas degenere è maggiore della pressione del gas non degenere.

\paragraph{Pressione del gas degenere relativistico -- Pressione del gas degenere non relativistico}
Per sapere quando la pressione del gas degenere relativistico prevale su quella del gas degenere \emph{non} relativistico, uguagliamo la~\eqref{eq:pressione-degenerazione-non-relativistica} con la~\eqref{eq:pressione-degenerazione-relativistica}:
\[
k_1 \rho^{5/3} = P = k_2 \rho^{4/3}
\]
si trova:
\[
\rho^{1/3} = \frac{k_2}{k_1}
\]
da cui, passando ai logaritmi:
\begin{equation}\label{eq:pdegrel-pdegnonrel}
    \log \rho = 3 \log \frac{k_2}{k_1} = 6.6
\end{equation}
La pressione in blu del diagramma~\ref{fig:diagramma-logrho-logt} è quella in cui la pressione del gas degenere relativistico è maggiore di quella del gas degenere non relativistico.

\paragraph{Pressione del gas perfetto -- Pressione del gas degenere relativistico}
Analogamente a quanto fatto nel caso \emph{non} relativistico, ci sarà una regione del diagramma (per $\log \rho > 6.6$) in cui il gas può passare da \emph{non degenere} a \emph{degenere relativistico}. Per trovarla uguagliamo la~\eqref{eq:gas-ideale} con la~\eqref{eq:pressione-degenerazione-relativistica}:
\[
\frac{\kb \rho T}{\mu H} = k_2 \rho^{4/3}
\]
si trova:
\[
T = \frac{k_2 \mu_e H}{\kb} \rho^{1/3}
\]
e passando ai logaritmi:
\begin{equation}\label{eq:pgas-pdegrel}
    \log T = \frac{1}{3} \log \rho + 7.07
\end{equation}

\paragraph{Conclusioni}
La pressione di radiazione, siccome $p_\textup{rad} \propto T^4$, domina ad alte temperature. La pressione del gas degenere diventa dominante ad alte densità perché per avere degenerazione devo avere gli elettroni estremamente impacchettati nello spazio delle fasi, e ciò corrisponde ad elevate densità. Nei casi intermedi è sufficiente riferirsi al grafico~\ref{fig:diagramma-logrho-logt} per capire in quale situazione si è.

Riprenderemo questo utile diagramma successivamente, parlando di evoluzione stellare. Per ora sottolineamo solamente che le reazioni termonucleari sono attive solo se il gas si trova in condizioni di non degenerazione (gas perfetto).
\section{Equazione del bilancio energetico}\label{sec:bilancio-energetico}
L'\emph{equazione del bilancio energetico} esprime l'energia che emerge da ogni \emph{shell} della struttura stellare, essenzialmente dovuta alle reazioni termonucleari. Si può scrivere:
\begin{equation}\label{eq:bilancio-energetico}
    \dfrac{\ud L(r)}{\ud r} = 4 \pi r^2 \rho(r) \epsilon
\end{equation}
dove $L(r)$ è la luminosità emergente dalla sfera di raggio $r$ e $\epsilon$ l'energia prodotta da ogni shell per unità di tempo e di massa, $[\epsilon] = \si{erg.s^{-1}.g^{-1}}$.

\subsection{Ricavare l'equazione}
È semplice ricavare tale equazione. consideriamo, infatti, l'usuale guscio di gas a una distanza $r$ dal centro.. Sia $L(r)$ la luminosità emergente dalla porzione di stella delimitata dal guscio a raggio $r$ e analogamente per $L(r + \ud r)$. Possiamo scrivere:
\[
L(r + \ud r) - L(r) = \ud L(r) = 4 \pi \rho r^2 \ud r \epsilon
\]
dove $4 \pi \rho r^2 \ud r$ rappresenta la massa del guscio di sfera tra $r$ e $r + \ud r$ e $\epsilon$ è l'energia prodotta da ogni guscio per unità di tempo e di massa. Attenzione, \emph{non} si tratta dell'energia radiata dalla stella. In questo modo si riottiene la definizione di luminosità (par.~\ref{sec:luminosità}).

Consideriamo un tipico profilo di luminosità radiale per una stella (fig.~\ref{fig:profilo-luminosità-sole}): la luminosità varia con $r$ sono in una regione ristretta della stella, corrispondente al suo centro, poi verso l'esterno diventa costante. All'esterno, secondo la eq.~\eqref{eq:bilancio-energetico} vale $\epsilon = 0$ e questo ci dice che tutta l'energia viene prodotta nelle regioni interne, come ci aspettiamo dal fatto che le reazioni termonucleari hanno bisogno di una elevata temperatura per poter avvenire. Tuttavia, le reazioni termonucleari \emph{non} sono le uniche a poter contribuire all'energia, in particolare anche la \emph{contrazione gravitazionale} può produrre energia, e tale contributo non è incluso nell'eq.~\eqref{eq:bilancio-energetico}. Per capire come la contrazione gravitazionale possa contribuire all'energia, bisogna far riferimento al \emph{teorema del viriale}.

\begin{figure}
\centering
\includegraphics[width=0.4\textwidth]{immagini/profilo-luminosita-radiale-sole.png}
\caption{Profilo di luminosità radiale del Sole. Nella zona evidenziata si ha $L(r) = \textup{const}$, da cui, secondo eq.~\eqref{eq:bilancio-energetico}, $\epsilon = 0$. Questo ci dice che negli strati esterni della stella non viene prodotta energia e le reazioni termonucleari sono concentrate nell'interno stellare, dove infatti le temperature sono più elevate.}
\label{fig:profilo-luminosità-sole}
\end{figure}

\subsection{Teorema del viriale}
Per ogni sistema in equilibrio di particelle auto-gravitanti, come nel caso delle strutture stellari, vale il \emph{teorema del viriale}:
\begin{equation}\label{eq:teorema-viriale}
    2K + \Omega = 0
\end{equation}
dove $K$ è l'energia cinetica del sistema e $\Omega$ l'energia potenziale. L'eq.~\eqref{eq:teorema-viriale} dice sostanzialmente che a una contrazione ($\ud \Omega < 0$) segue un aumento dell'energia cinetica, e quindi un aumento della temperatura. Scrivendo inoltre l'energia totale del sistema come $U = K + \Omega$, si ha
\[
U = - \frac{\Omega}{2} + \Omega \implies U = \frac{\Omega}{2}
\]
ovvero a una contrazione ($\ud \Omega = 0$) segue una diminuzione dell'energia totale del sistema ($\ud U < 0$).

Le due relazioni considerate precedentemente:
\begin{equation}\label{eq:viriale-differenziale}
\ud U = \frac{\ud \Omega}{2} \qquad \ud K = - \frac{\ud \Omega}{2}
\end{equation}
mostrano che in ogni contrazione ($\ud \Omega$) \emph{metà} dell'energia è \emph{emessa} e \emph{metà} dell'energia è usata per incrementare la temperatura del caso. Possiamo riformulare quanto detto in un altro modo: ogni perdita di energia totale $\ud U$ dovuta a emissioni genera una contrazione del sistema, che produce anche un incremento della temperatura interna.

Come stimare il contributo della contrazione gravitazionale alla luminosità di una stella? Se fosse elevato andrebbe corretta l'eq.~\eqref{eq:bilancio-energetico}. Per capirlo utilizziamo la~\eqref{eq:viriale-differenziale} per scrivere
\[
L = \frac{\ud U}{\ud t} = \frac{1}{2} \abs*{\frac{\ud \Omega}{\ud t}}
\]
ovvero abbiamo scritto la luminosità in funzione della variazione di energia potenziale dovuta alla contrazione gravitazionale. Integriamo nel tempo:
\[
\int_0^t L \ud t = \frac{1}{2} \abs{\Omega}
\]
A questo punto consideriamo un tempo caratteristico $t^*$ approssimando $L$ con un suo valore medio costante $\Bar{L}$. Si può scrivere:
\[
\Bar{L} \cdot t^* = \frac{1}{2} \frac{GM^2}{R}
\]
dove abbiamo sostituito $\Omega$ con l'espressione del potenziale gravitazionale. Quindi abbiamo trovato il così detto \emph{tempo di Kelvin-Helmoltz}:
\begin{equation}\label{eq:tempo-kelvin-helmoltz}
    t^* = \dfrac{GM^2}{2LR}
\end{equation}
Esso ci dà una \emph{stima} del tempo durante il quale una stella è in grado di  mantenere costante la sua luminosità per effetto della sola contrazione gravitazionale. 

Facciamo una stima per il nostro Sole introducendo nella~\eqref{eq:tempo-kelvin-helmoltz} i parametri $G = \SI{6.67e-8}{cm^3.g^{-1}.s^{-2}}$, $\si{\solarmass} \sim \SI{2e33}{g}$, $\si{\solarradius} \sim \SI{7e10}{cm}$ e $\si{\solarluminosity} \sim \SI{4e33}{erg.s^{-1}}$. Si ottiene:
\[
{t^*}_\textup{Sole} \sim \SI{1.5e7}{anni}
\]
ovvero, per il nostro Sole, l'energia gravitazionale può mantenere costante la luminosità per circa $15$ milioni di anni (sono molto pochi). Questo farebbe pensare che la fonte principale di luminosità sia la contrazione gravitazionale per il nostro Sole, ma si rivela un'idea sbagliata perché attraverso fonti geologiche si è mostrato che l'età del Sole è dell'ordine dei miliardi di anni e che in tale tempo la luminosità del Sole è rimasta praticamente invariata. Questo dimostra che il grosso del contributo alla luminosità del Sole proviene dalle reazioni termonucleari e quindi l'equazione~\eqref{eq:bilancio-energetico} risulta soddisfacente anche se non contempla il contributo della contrazione di gravità.
\section[Gradiente radiativo]{Gradiente radiativo e criterio di Schwarzschild}\label{sec:gradiente-radiativo}
L'equazione del \emph{gradiente radiativo} fornisce il profilo radiale delle variazioni di $T$ all'interno della stella. Si può scrivere:
\begin{equation}\label{eq:gradiente-radiativo}
    \dfrac{\ud T}{\ud r}\Big|_\textup{rad} = - \dfrac{3 \kappa \rho}{4 \pi r^2} \dfrac{L(r)}{4 a c T^3}
\end{equation}
Essa mostra che l'opacità e il flusso radiativo determinano quanto rapidamente $T$ varia con $r$.

Utilizzando questa equazione è possibile determinare quando c'è convezione nella stella, utilizzando il seguente \emph{criterio di Schwarzscild}:
\begin{equation}\label{eq:criterio-schwarzscild}
    \text{se} \quad \abs*{\dfrac{\ud T}{\ud r}}_\textup{rad} > \abs*{\dfrac{\ud T}{\ud r}}_\textup{ad} \implies \text{c'è convezione.}
\end{equation}
in particolare, quando il criterio è verificato avverrà la convezione.

\subsection{Ricavare l'equazione}
Per trovare il gradiente radiativo prima di tutto ricaviamo il gradiente di pressione di radiazione, derivando rispetto a $r$ l'espressione~\eqref{eq:pressione-radiazione}. Si ottiene:
\[
\dfrac{\ud P_\textup{rad}}{\ud r} = \frac{4}{3} a T^3 \dfrac{\ud T}{\ud r}
\] 
d'altra parte, il gradiente di radiazione dipende anche dall'\emph{opacità} e dal \emph{flusso} di radiazione. Si ricorda che l'opacità $\kappa$ era stata introdotta nell'eq.~\eqref{eq:trasporto-radiativo} del par.~\ref{sec:soluzioni-trasporto-radiativo}, mentre il flusso, $F_\textup{rad}$, è definito dall'eq.~\eqref{eq:flusso}. In particolare, senza dare ulteriori specificazioni, possiamo esprimere il gradiente della pressione di radiazione come:
\[
\dfrac{\ud P_\textup{rad}}{\ud r} = -\dfrac{\kappa \rho}{c} F_\textup{rad}
\]
e mettendo insieme le utlime due relazioni si ottiene:
\[
\dfrac{\ud T}{\ud r}\Big|_\textup{rad} = - \dfrac{3}{4ac} \dfrac{\kappa \rho}{T^3} F_\textup{rad}
\]
con
\[
F_\textup{rad} = \dfrac{L_r}{4 \pi r^2}
\]
da cui segue immediatamente la~\eqref{eq:gradiente-radiativo}. Come già detto, l'equazione fornisce il profilo radiale delle variazioni di $T$ con $r$ e mostra come l'opacità $\kappa$ e il flusso radiativo $F_\textup{rad}$ influiscono sulla rapidità di variazione di $T$ con $r$.

Il gradiente di temperatura ha un impatto cruciale sul meccanismo preponderante di trasporto di energia all'interno della struttura stellare. Introduciamo brevemente i meccanismi di trasporto di energia.

\subsection{Meccanismi di trasporto di energia}
I tre meccanismi di trasporto in una stella sono il \emph{trasporto radiativo}, \emph{convettivo} e \emph{conduttivo}. Di seguito ne sono elencate le principali proprietà:
\begin{description}
    \item[trasporto conduttivo] I principali responsabili di questo meccanismo sono gli \emph{elettroni} ed è efficiente solo se il gas è \emph{degenere}. Infatti, in condizione di non degenerazione, ovvero se il gas è perfetto, il libero cammino medio è molto piccolo e un elettrone cede subito energia. Come visto, nel caso di un gas degenere (par.~\ref{sec:principio-indeterminazione}) le celle dello spazio delle fasi sono impacchettati in modo che i livelli energetici più bassi sono pieni, quindi gli elettroni percorrono una grande distanza prima di cedere energia (la quale deve essere di un ordine pari al primo livello energetico libero, che tendenzialmente sarà alto). In questo caso il trasporto conduttivo è efficiente e in questo modo è possibile trasportare l'energia dall'interno verso l'esterno. Tuttavia, questo meccanismo di trasporto non è quello preponderante.
    \item[trasporto radiativo] Esso è dovuto alla radiazione trasportata dai \emph{fotoni}.
    \item[trasporto convettivo] Con la convezione ho un rimescolamento del \emph{gas} e dunque di porzioni di gas con composizioni chimiche diversa. Siccome la struttura chimica è cruciale per stabilire la struttura stellare e la sua evoluzione, è importante capire se è in atto questo meccanismo di trasporto energetico.
\end{description}

\subsection{Criterio di Schwarzschild}
Come evidenziato nel paragrafo precedente, stabilire se sia in corso un meccanismo di convezione è importante per realizzare dei corretti modelli della struttura stellare e dunque per capire correttamente l'evoluzione di una stella. Come discriminante si può utilizzare il \emph{criterio di Schwarzschild} (eq.~\eqref{eq:criterio-schwarzscild}). Semplicemente si tratta di un confronto tra il \emph{gradiente radiativo} di temperatura e un valore di riferimento chiamato \emph{gradiente adiabatico}. In particolare, nelle regioni in cui il gradiente radiativo è maggiore del gradiente adiabatico, è in corso la convezione, come stabilito dalla~\eqref{eq:criterio-schwarzscild}. Il gradiente adiabatico dipende dai calori specifici del gas e si può scrivere:
\begin{equation}\label{eq:gradiente-adiabatico}
    \dfrac{\ud T}{\ud r}\Big|_\textup{ad} = \Bigl(1- \frac{1}{\gamma}\Bigr) \dfrac{T}{P} \dfrac{\ud P}{\ud r}
\end{equation}
dove
\[
\gamma =\frac{c_P}{c_V} = \frac{5}{3}
\]

\begin{figure}
\centering
\includegraphics[width=0.3\textwidth]{immagini/criterio-schwarzscild.png}
\caption{Raffigurazione del criterio di Schwarzscild. Se la bolla si sposta dalla posizione $1$ e $2$, può avvenire la convezione solamente se ${\rho_2}^* < \rho_2$. In caso contrario siamo in presenza di equilibrio stabile e la bolla viene respinta verso la posizione iniziale.}
\label{fig:criterio-schwarzscild}
\end{figure}

Per capire il criterio~\eqref{eq:criterio-schwarzscild} consideriamo (fig.~\ref{fig:criterio-schwarzscild}) una bolla di gas in una posizione $1$ a distanza $r$ rispetto al centro, di densità ${\rho_1}^*$ e pressione interna ${P_1}^*$. Siano $\rho_1$ e $P_1$ rispettivamente la densità e la pressione dell'ambiente circostante la bolla in quella posizione. Immagino che essa si sposti in una posizione $2$ a distanza $r + \ud r$ dal centro, avendo una nuova densità ${\rho_2}^*$ e una nuova pressione interna ${P_2}^*$. Analogamente a prima, siano $\rho_2$ e $P_2$ rispettivamente la densità e la pressione dell'ambiente circostante la bolla in quella posizione. Ho convezione nella stella solo se l'equilibrio è instabile, perché in caso di equilibrio stabile la bolla tenderebbe a tornare nella posizione originaria $1$ e non sarebbe possibile lo spostamento di porzioni di gas nella stella. La condizione di equilibrio dipende dal rapporto della densità interna della bolla nel secondo caso, ${\rho_2}^*$, e la densità del gas circostante, $\rho_2$. In particolare:
\begin{subequations}
\label{eq:relazioni-densità-schwarzscild}
\begin{align}
  \text{se} \quad {\rho_2}^* &> \rho_2 \implies \text{la bolla torna nella posizione iniziale $1$.} \\
  \text{se} \quad {\rho_2}^* &< \rho_2 \implies \text{la bolla continua ad andare verso l'alto.} 
\end{align}
\end{subequations}
ed è semplice tradurre le espressioni~\eqref{eq:relazioni-densità-schwarzscild} in termini dei gradienti di temperatura, ottenendo~\eqref{eq:criterio-schwarzscild}.

\subsection{Esempio per il gas perfetto}
\begin{figure}
\centering
\subfloat[][\emph{Caso in cui avviene la convezione: modulo del gradiente radiativo maggiore del gradiente adiabatico.} \label{fig:criterio-schwarzscild-gas-perfetto1}]
{\includegraphics[width=.35\textwidth]{immagini/criterio-schwarzscild-gas-perfetto1.png}} \qquad
\subfloat[][\emph{Caso in cui non avviene la convezione: modulo del gradiente radiativo minore del gradiente adiabatico.}\label{fig:criterio-schwarzscild-gas-perfetto2}]
{\includegraphics[width=.35\textwidth]{immagini/criterio-schwarzscild-gas-perfetto2.png}} 
\caption{Applicazione del criterio di Schwarzscild nel caso di un gas perfetto.}
\label{fig:criterio-schwarzscild-gas-perfetto}
\end{figure}

Per capire meglio le condizioni~\eqref{eq:relazioni-densità-schwarzscild} e il criterio di Schwarzscild~\eqref{eq:criterio-schwarzscild}, consideriamo il caso di un gas perfetto, in cui vale la legge~\eqref{eq:gas-ideale}. Si faccia riferimento alla figura~\ref{fig:criterio-schwarzscild-gas-perfetto}. Consideriamo un caso in cui è possibile la convezione e un caso in cui essa non è possibile

\paragraph{Convezione possibile}
Nel caso illustrato nella figura~\ref{fig:criterio-schwarzscild-gas-perfetto1} la convezione è possibile, infatti in quel caso si ha:
\[
abs*{\dfrac{\ud T}{\ud r}}_\textup{rad} > \abs*{\dfrac{\ud T}{\ud r}}_\textup{ad}
\]
si faccia attenzione al fatto che nel criterio di Schwarzscild~\eqref{eq:criterio-schwarzscild} i gradienti sono espressi in modulo, quindi bisogna guardare la pendenza in modulo delle rette. In particolare, la bolla di gas di sposta da $r_0$ fino a $r_0 + \ud r$ e nella posizione $r_0 + \ud r$ possiede una temperatura (curva rossa -- profilo adiabatico) maggiore dell'ambiente circostante (curva blu -- profilo radiativo). Per un gas perfetto, secondo la ~\eqref{eq:gas-ideale}, a fissata pressione, a una maggiore temperatura corrisponde una densità più bassa. Quindi la bolla possiede una densità più bassa dell'ambiente circostante e tende a continuare il suo moto verso l'esterno. È dunque possibile la convezione.

\paragraph{Convezione impossibile}
Nel caso illustrato nella figura~\ref{fig:criterio-schwarzscild-gas-perfetto2} la convezione \emph{non} è possibile, infatti in quel caso si ha:
\[
abs*{\dfrac{\ud T}{\ud r}}_\textup{rad} < \abs*{\dfrac{\ud T}{\ud r}}_\textup{ad}
\]
In particolare, la bolla di gas di sposta da $r_0$ fino a $r_0 + \ud r$ e nella posizione $r_0 + \ud r$ possiede una temperatura (curva rossa -- profilo adiabatico) minore dell'ambiente circostante (curva blu -- profilo radiativo). Per un gas perfetto, secondo la ~\eqref{eq:gas-ideale}, a fissata pressione, a una minore temperatura corrisponde una densità più alta. Quindi la bolla possiede una densità più alta dell'ambiente circostante e viene respinta verso la posizione iniziale a $r_0$.


Si faccia riferimento alla figura
Si faccia riferimento alla figura~\ref{fig:criterio-schwarzscild-gas-perfetto2}

\subsection{Gradiente di pressione e scala logaritmica}
Il criterio di Schwarzscild è spesso scritto usando il gradiente di temperatura riferito alla \emph{pressione} invece che al raggio e in scala logaritmica. Questo perché la temperatura è intimamente connessa con la pressione.
\begin{equation}
    \dfrac{\ud T}{\ud P} \dfrac{P}{T} = \dfrac{\ud \log T}{\ud \log P} \equiv \nabla
\end{equation}
L'impiego di $\nabla$ semplifica la formulazione del criterio. Ad esempio, possiamo riscrivere l'espressione del gradiente adiabatico~\eqref{eq:gradiente-adiabatico} nella seguente maniera:
\[
\Bigl(1- \frac{1}{\gamma}\Bigr) = \dfrac{P}{T} \dfrac{\ud r}{\ud P} \dfrac{\ud T}{\ud r}\Big|_\textup{ad} = \dfrac{P}{T} \dfrac{\ud T}{\ud P}\Big|_\textup{ad}
\]
da cui:
\begin{equation}
    \nabla_\textup{ad} = \Bigl(1- \frac{1}{\gamma}\Bigr)
\end{equation}
che è semplicemente un numero. In particolare, con $\gamma = 5/3$ si ottiene $\nabla_textup{ad} = 0.4$ ed è possibile riscrivere il criterio di Schwarzscild~\eqref{eq:criterio-schwarzscild} nella seguente materia:
\begin{equation}\label{eq:criterio-schwarzscild-nabla}
    \text{se} \quad \nabla_\textup{rad} > \nabla_\textup{ad} \implies \text{c'è convezione.}
\end{equation}
Si faccia riferimento alla figura~\ref{fig:criterio-schwarzscild-nabla} per un sunto del criterio.

\begin{figure}
\centering
\includegraphics[width=0.4\textwidth]{immagini/criterio-schwarzscild-nabla.png}
\caption{Raffigurazione del criterio di Schwarzscild utilizzando $\nabla$.}
\label{fig:criterio-schwarzscild-nabla}
\end{figure}
\section{Opacità}\label{sec:opacità}
L'opacità è una misura della resistenza della materia al flusso radiativo, ovvero alla transizione della radiazione. È una sorta di sezione d'urto per unità di massa.
\[
\kappa = \kappa(\rho, T) \qquad [\kappa] = \si{cm^2.g^{-1}}
\]
L'equazione dell'opacità in un modello stellare si può riassumere nel seguente set di equazioni, che saranno sviscerate nei paragrafi seguenti:
\begin{equation}\label{eq:opacità}
    \kappa = \kappa(\rho, T) 
    \begin{cases} 
    \kappa_{BF} \propto 10^{25} Z(1 + X) \dfrac{\rho}{T^{3.5}} \\ 
    \kappa_{FF} \propto 10^{22} (X+Y)(1+X) \dfrac{\rho}{T^{3.5}} \\ 
    \kappa_{E} \propto 0.2 (1+X) \\ 
    \end{cases}
\end{equation}
I fenomeni che possono interferire con il passaggio di radiazione in una stella sono essenzialmente dovuti alla capacità degli elettroni di assorbire e/o deviare i fotoni, e si possono riassumere nei seguenti processi:
\begin{itemize}
    \item Assorbimento bound-bound (BB)
    \item Assorbimento bound-free(BF)
    \item Assorbimento free-free (FF)
    \item Scattering elettronico (E)
\end{itemize}
i quali sono spiegati di seguito. La rilevanza di questi processi dipende principalmente da tre fattori:
\begin{itemize}
    \item La composizione chimica del gas ($X$, $Y$, $Z$).
    \item La temperatura.
    \item La densità.
\end{itemize}
In particolare, è la temperatura a stabilire il livello di ionizzazione e dell'eccitazione di ogni specie chimica. In figura~\ref{fig:assorbimenti-stella} sono riportati i diversi processi di assorbimento in funzione della distanza dal centro stellare. È difficile una modellizzazione di tali processi, pertanto si utilizzano delle relazioni approssimate, note con il nome di \emph{leggi di Kramer}, che saranno mostrate nei seguenti paragrafi.

\begin{figure}
\centering
\includegraphics[width=0.3\textwidth]{immagini/assorbimenti-stella.png}
\caption{Processi di assorbimento preponderanti in funzione della distanza dal centro. Andando verso il centro della struttura le temperature aumentano, quindi il numero di atomi completamente ionizzati e il numero di elettroni liberi aumenta. In superficie domina il BB perché le temperature sono minori e gli atomi sono ancora non ionizzati. Andando verso il centro inizia a dominare il BF perché alcuni atomi sono parzialmente ionizzati. Verso l'interno domina il FF perché quasi tutti gli atomi sono ionizzati.}
\label{fig:assorbimenti-stella}
\end{figure}

\subsection{Assorbimento bound-bound (BB)}\label{sec:bound-bound}
Un elettrone legato a un atomo, in uno stato di energia $E_1$, cattura un fotone e passa a uno stato eccitato $E_2$ \emph{rimanendo legato all'atomo}. Il fotone catturato ha energia:
\begin{equation}
    h \nu_{BB} = E_2 - E_1
\end{equation}
questo effetto \emph{non} è rilevante negli interni stellari, siccome quasi tutti gli atomi sono completamente ionizzati a causa delle elevate temperatura (fig.~\ref{fig:assorbimenti-stella}). Tuttavia, questo fenomeno è cruciale nell'\emph{atmosfera} stellare ed è responsabile della formazione delle righe di assorbimento spettrale.

In particolare, tale fenomeno avviene a una determinata lunghezza d'onda, ovvero:
\begin{equation}\label{eq:lunghezza-BB}
    \lambda_{12} = h \dfrac{c}{E_2 - E_1}
\end{equation}

Come detto, essendo questo fenomeno non rilevante negli interni stellari, non viene considerato nell'equazione dell'opacità.

\subsection{Assorbimento bound-free (BF)}\label{sec:bound-free}
Un elettrone legato a energia $E_1$ cattura un fotone e diventa libero, con energia $E_\infty$, producendo uno ione. Il fotone catturato ha energia
\begin{equation}
    h \nu_{BF} = E_\infty - E_1 > \chi_\textup{ion}
\end{equation}
dove $\chi_\textup{ion}$ è l'energia di ionizzazione dell'atomo. 

Tale fenomeno, dunque, avviene a una \emph{lunghezza d'onda} minore di una determinata soglia (di conseguenza, per la \emph{frequenza}, ho un limite inferiore ma non superiore), ovvero:
\begin{equation}\label{eq:lunghezza-BF}
    \lambda < \lambda_\textup{soglia}
\end{equation}

Rispetto all'atmosfera, in cui domina il BB, andando verso il centro inizierà progressivamente a dominare il BF a causa dell'aumento della temperatura e del fatto che  alcuni atomi iniziano a essere parzialmente ionizzati (fig~\ref{fig:assorbimenti-stella}). La corrispondente legge di Kramer, che compare nella~\eqref{eq:opacità}, si può scrivere nella seguente maniera:
\begin{equation}
    \kappa_{BF} \propto 10^{25} Z (1+X) \dfrac{\rho}{T^{3.5}}
\end{equation}
L'opacità $\kappa_{BF}$ dipende da Z, ovvero dall'abbondanza degli elementi più pesanti dell'elio, perché affinché il BF possa avvenire è necessario che siano ancora presenti degli atomi con elettroni legati. A causa della ionizzazione dell'idrogeno e dell'elio verso gli interni stellari, si possono trovare elettroni legati solamente negli elementi più pesanti.

\subsection{Assorbimento free-free (FF)}\label{sec:free-free}
Un elettrone libero, di energia $E_1$, cattura un fotone e la sua energia aumenta a $E_2$. Il fotone catturato ha energia:
\begin{equation}
    h \nu_{FF} = E_2 - E_1
\end{equation}
senza restrizioni come nei casi precedenti, perché il sistema non è legato. Questo processo è dominante negli interni stellari, perché lì la temperatura è così elevata che gli atomi sono tutti ionizzati (fig~\ref{fig:assorbimenti-stella}). Non ci sono limitazioni alla lunghezza d'onda in cui tale fenomeno può avvenire. La corrispondente legge di Kramer, che compare nella~\eqref{eq:opacità}, si può scrivere nella seguente maniera:
\begin{equation}
   \kappa_{FF} \propto 10^{22} (X+Y) (1+X) \dfrac{\rho}{T^{3.5}}
\end{equation}
L'opacità $\kappa_{FF}$ \emph{dipende da X e Y} perché il FF dipende dagli elettroni liberi e l'idrogeno e l'elio, che sono ionizzati alle alte temperature degli interni stellari, sono di gran lunga gli elementi più abbondandi, e dunque anche i principali fornitori di elettroni liberi.

\subsection{Scattering elettronico (E)}\label{sec:electron-scattering}
Un elettrone libero interagisce con un fotone, cambiando la propria traiettoria. Non si tratta di un reale assorbimento, ma agisce comunque come un effetto di opacità perché essendo il fotone deviato dal fascio, l'intensità del fascio stesso diminuisce. La corrispondente legge di Kramer, che compare nella~\eqref{eq:opacità}, si può scrivere nella seguente maniera:
\begin{equation}
    \kappa_E \propto 0.2(1+X)
\end{equation}
Si noti come \emph{non} ci sia dipendenza dalla temperatura $T$ e dalla densità $\rho$ e come esso \emph{dipende solo da X}, poiché l'idrogeno è l'elemento più abbondante. Lo scattering diventa dominante solamente ad alte temperature, perché per gli altri processi $\kappa \propto T^{-3.5}$.

\subsection{Recap}
Per riassumere l'andamento dell'opacità rappresentato dall'eq.~\eqref{eq:opacità}, facciamo riferimento alla figura~\ref{fig:opacità}. Possiamo notare che, in generale, l'opacità aumenta con la densità, infatti si ha $\kappa \propto \rho$. Inoltre si può notare che per temperature nelle regioni a $T \sim \SI{e4}{K}$ l'opacità aumenta, infatti questa è la finestra di temperatura che corrisponde alla \emph{ionizzazione dell'idrogeno}. Ovviamente, se l'idrogeno \emph{non} è ionizzato, allora i processi BF, FF ed E sono impossibili, poiché necessitano di elettroni liberi. D'altra parte, all'aumentare della temperatura, dunque al grado di ionizzazione degli elementi presenti nella struttura stellare, l'opacità decresce secondo~\eqref{eq:opacità}, con $\kappa \propto T^{-3.5}$. Ad altissime temperature, invece, domina lo scattering elettronico e la curva si appiattisce, come ci si  aspetta.

\begin{figure}
    \centering
    \includegraphics[width=0.35\textwidth]{immagini/opacita.png}
    \caption{Diagramma $\log \kappa$ -- $\log T$. Sono rappresentate curve a diverso $\log \rho$. A $T \sim \SI{e4}{K}$, curva rossa, l'opacità aumenta perché è in atto la ionizzazione dell'idrogeno. Nella curva verde si ha $\kappa \propto T^{-3.5}$ come indicato dalla eq.~\eqref{eq:opacità}. Nella curva blu domina lo scattering elettronico e l'opacità è costante come atteso.}
    \label{fig:opacità}
\end{figure}
\section[Reazioni Termonucleari]{Produzione di energia tramite reazioni termonucleari}\label{sec:reazioni-termonucleari}
Le \emph{reazioni termonucleari} sono la principale fonte di energia in una stella. La \emph{fusione} di elementi leggeri in elementi più pesanti produce non solo \emph{energia}, ma anche \emph{nuovi elementi}. Come nel caso dell'opacità i processi sono molto complessi e utilizzeremo solamente delle equazioni approssimate. Si faccia attenzione al fatto che i processi di fusione termonucleare coinvolgono solamente i \emph{nuclei} dell'atomo, perché le temperature sono così elevate ($T > \SI{e6}{K}$) che tutti gli atomi si possono considerare completamente ionizzati.

L'equazione di produzione di energia attraverso \emph{reazioni termonucleari} appare nella seguente maniera:
\begin{equation}\label{eq:reazioni-termonucleari}
    \epsilon = \epsilon(X, \rho, T) 
    \begin{cases} \epsilon_{PP} = \epsilon_1 \rho X^2 T_6^\alpha \quad \alpha \in [3.5 - 6] \\ 
    \epsilon_{CN} = \epsilon_2 \rho X X_{CN} T_6^\beta \quad \beta \in [13 - 20] \\ 
    \epsilon_{3\alpha} = \epsilon_3 \rho^2 Y^3 T_8^\gamma \quad \gamma \in [20 - 30] \\
    \end{cases}
\end{equation}
È dapprima necessario introdurre dei concetti di base sulla reazioni termonucleari.

\subsection{Ripasso di fisica nucleare}
\paragraph{Numero atomico e numero di massa}
Nel presente paragrafo si farà una veloce rassegna dei concetti di fisica nucleare utili per proseguire il discorso.

Ogni elemento chimico è univocamente identificato dal suo \emph{numero atomico} $Z$, corrispondente al numero di protoni nel nucleo. Il \emph{peso atomico} $A$ è il numero totale di nucleoni, ovvero la somma di protoni e neutroni. Gli \emph{isotopi} hanno stesso $Z$ ma diverso $A$.

\paragraph{Difetto di massa}
Nelle reazioni di fusione termonucleare, nuclei di elementi leggeri si fondono tra loro generando nuclei di elementi più pesanti ($Z$ più alto) ed energia. Pertanto, a causa della nota relazione massa-energia, la somma delle masse dei nuclei leggeri che fondono tra loro è maggiore della massa del nucleo più pesante che viene generato, e si può scrivere:
\begin{equation}\label{eq:difetto-massa}
    E = \Delta m \, c^2
\end{equation}
Guardando la carta dei nuclidi che rappresenta l'energia di legame per nucleone (fig.~\ref{fig:energia-legame}), possiamo stabilire che è grazie alle reazioni di fusione del nucleo delle stelle che sono stati prodotti tutti gli elementi più pesanti dell'elio ($Z=2$) fino al ferro ($Z=26$), come verrà riassunto successivamente.

\paragraph{Energia di legame}
All'interno del nucleo atomico, i nucleoni sono legati tra loro dalla \emph{forza forte}, la quale agisce su distanze estremamente piccole ($\sim \SI{e-12}{cm}$ -- $\SI{e-13}{cm}$). La massa totale del nucleo è sempre minore della somma di tutti i nucleoni che lo costituiscono. Questo si spiega attraverso il difetto di massa, eq.~\eqref{eq:difetto-massa}, e permette di definire l'\emph{energia di legame} come:
\begin{equation}\label{eq:energia-legame}
    E(Z,N) = \bigl[ Z m_p + N m_n - m(Z,n) \bigr] \, c^2
\end{equation}

con $Z$ il numero di protoni, $N$ il numero di neutroni, $m_p = \SI{1.672e-24}{g}$ la massa del protone, $m_n = \SI{1.675e-24}{g}$ la massa del neutrone e $m(Z,n)$ la massa del nucleo. Questo significa che quando si forma un nuovo nucleo stabile, una certa frazione di massa viene trasformata in energia secondo la~\eqref{eq:energia-legame}, ovvero, $E(Z,N)$ è l'energia che viene prodotta quando si forma un nuovo nucleo stabile, ovvero $E(Z,N)$ è l'energia che bisogna fornire ad un nucleo per spaccarlo nei singoli nucleoni che lo costituiscono.

\paragraph{Energia di legame per nucleone dei nuclidi stabili}

\begin{figure}
    \centering
    \includegraphics[width=0.7\textwidth]{immagini/energia-legame.png}
    \caption{Energia di legame media per nucleone. Si noti come \ce{^4He} sia particolarmente stabile, fino al ferro domina la fusione e dopo il ferro domina la fissione.}
    \label{fig:energia-legame}
\end{figure}

Consideriamo l'\emph{energia di legame media per nucleone}, plottata in fig.~\ref{fig:energia-legame}, e corrispondente, secondo la~\eqref{eq:energia-legame} a $E(Z,N) / A$. Dalla figura possiamo trarre alcune considerazioni generali:
\begin{itemize}
    \item ci sono configurazioni nucleari particolarmente stabili quali \ce{^4He}, \ce{^{12}C}, \ce{^{16}O} \dots
    \item a parte queste eccezioni, l’energia di legame media per nucleone, $E/A$, ha un andamento regolare. Aumenta rapidamente con il numero di nucleoni fino ad un valore dell’ordine degli $\SI{8}{Mev}$ per poi diminuire assai lentamente (\emph{proprietà di saturazione}).
    \item i nuclei più stabili sono il \ce{^{56}Fe} e il \ce{^{62}Ni}. Ciò significa che i nuclei pesanti alla sua destra possono raggiungere configurazioni più stabili ($B/A$ più elevato) diminuendo il numero di nucleoni $A$, ovvero frazionandosi in nuclei più piccoli. Mentre i nuclei leggeri alla sua sinistra possono raggiungere configurazioni più stabili ($B/A$ più elevato) aumentando $A$, ovvero aggregandosi in nuclei più grandi. Detto in altri termini ciò significa che le reazioni di fissione dei nuclei pesanti e quelle di fusione dei nuclei leggeri sono esoenergetiche ovvero producono energia qualora si sia in grado di innescarle.
    \item dal punto precedente consegue che le reazioni di \emph{fusione dei nuclei leggeri} e \emph{fissione dei nuclei pesanti} costituiscono la doppia opportunità offerta dalla fisica nucleare per la produzione di energia.
\end{itemize}

\paragraph{Barriera di potenziale ed effetto tunnel}
Da un punto di vista \emph{classico} la reazione di fusione avviene se due nuclei riescono ad avvicinarsi a meno della distanza $R_0$ necessaria per far entrare in gioco le interazioni forti, ovvero solo dopo aver superato la barriera di potenziale in fig.~\ref{fig:barriera-potenziale}

\begin{figure}
    \centering
    \includegraphics[width=0.3\textwidth]{immagini/barriera-potenziale.png}
    \caption{Barriera di potenziale. È data dalla sovrapposizione del potenziale attrattivo della forza forte, che agisce per $r < R_0$ e della forza repulsiva Coulombiana, che agisce per $r > R_0$ e vale $E_c = Z_1 Z_2 e^2 / r$.}
    \label{fig:barriera-potenziale}
\end{figure}

Facendo una stima molto rozza, possiamo trovare che la barriera di potenziale è circa $1000$ volte superiore all'energia termica media delle particelle, quindi, anche ad elevatissime temperature, è molto improbabile che le reazioni possano avvenire. Dal punto di vista quantistico questo si spiega con il così detto \emph{effetto tunnel}.

\paragraph{Pricnicpali catene di fusione termonucleare nelle stelle}
Di seguito sono elencate le principali catene di fusione nelle stelle, le quali saranno approfondite nei successivi paragrafi:
\begin{description}
    \item[Catena PP] Processo di bruciamento dell'idrogeno. È un processo di cattura protonica. Avviene per $T \simeq \SI{e7}{K}$.
    \item[Catena CNO] Processo di bruciamento dell'idrogeno. È un processo di cattura protonica. Avviene per $T \simeq \SI{1.8e7}{K}$.
    \item[Catena 3-alpha] Processo di bruciamento dell'elio. Avviene per temperature $T \simeq \SI{1.5e8}{K}$.
    \item[Cattura alpha] Il carbonio e materiali più pesanti catturano particelle $\alpha$ per produrre materiali più pesanti. Avviene per $T > \SI{5e8}{K}$.
\end{description}

\subsection{Catena protone--protone}\label{sec:catena-pp}
\paragraph{Catena PPI}
La catena \emph{protone--protone PPI} è rappresentata in fig.~\ref{fig:catena-pp1}. È una reazione di bruciamento dell'idrogeno e dà elio. Può essere scritta nella seguente maniera:
\reaction[re:pp1]{H^1 + H^1 -> H^2 + e^+ + $\nu$}
\reaction[re:pp2]{H^2 + H^1 -> He^3 + $\gamma$}
\reaction[re:pp3]{He^3 + He^3 -> He^4 + H^1 + H^1}
ovvero, in definitiva da \ce{4 H} ottengo \ce{1 He^4}, e siccome \ce{1 He^4} pesa meno di \ce{4 H}, dall'eq.~\eqref{eq:difetto-massa} possiamo stabilire che la reazione è \emph{esotermica}.

\begin{figure}
    \centering
    \includegraphics[width=0.5\textwidth]{immagini/catena-pp1.png}
    \caption{Catena protone-protone (PPI).}
    \label{fig:catena-pp1}
\end{figure}

Facciamo una stima dei contributi energetici. Il neutrino $\nu$ porta un contributo energetico \emph{negativo}, perché avendo una scarsa interazione con la materia tendono a sfuggire dalla struttura stellare. Senza aggiungere ulteriori specificazioni, si ricordi che, tenuti in considerazione tutti i contributi, l'energia totale prodotta dalla reazione PPI è:
\begin{equation}\label{eq:energia-pp1}
    E_\textup{PPI} = \SI{26.2}{MeV} = \SI{4.2e-5}{erg}
\end{equation}

Senza scendere troppo nei dettagli, evidenziamo che il \emph{tempo scala} della reazione~\ref{re:pp1} è $t_1 = \SI{1.4e9}{yr}$, il tempo scala della reazione~\ref{re:pp2} è $t_2 = \SI{6}{s}$ e il tempo scala della reazione~\ref{re:pp3} è $t_3 = \SI{e6}{yr}$. Dunque, ricordando che la \emph{probabilità} che avvenga la reazione è proporzionale all'inverso del tempo scala, possiamo stabilire che la prima reazione è molto improbabile che avvenga, mentre la seconda è molto probabile. In particolare, concentriamoci sulla prima reazione (~\ref{re:pp1}). Essa trasforma due protoni liberi (\ce{H}) in un nucleo costituito da un protone e un neutrone (\ce{H^2}). Significa che uno dei due protoni si è trasformato in neutrone. Quindi, affinché la reazione possa avvenire, è necessario che ci sia un \emph{decadimento $\beta$}, espresso dalla seguente equazione:
\reaction[re:decadimento-beta+]{p^+ -> n + e^+ + $\nu$}
Tuttavia, il decadimento $\beta^+$ per un protone libero è essenzialmente impossibile, poiché la massa del protone è minore della massa del neutrone (eq.~\eqref{eq:difetto-massa}). Nonostante ciò, nel nucleo delle stelle tale reazione può avvenire poiché ci sono tantissimi protoni, dunque anche se la probabilità è bassissima, questa è compensata dall'elevato numero. 

D'altra parte, il \emph{decadimento $\beta^-$}, secondo il quale un neutrone decade in protone, è un processo spontaneo, in quanto il suo tempo scala è dell'ordine di $\SI{800}{s}$ e tende ad eliminare tutti i neutroni liberi:
\reaction[re:decadimento-beta-]{n -> p^+ + e^- + $\nu$}
Tuttavia negli interni stellari possono comunque esserci neutroni liberi, i quali contribuiscono alla formazione di elementi più pesanti, come vedremo successivamente.

\paragraph{Catena PPII e PPIII}
Consideriamo le reazioni~\ref{re:pp1} e~\ref{re:pp2}. Con una probabilità $P_1 = 69 \%$ può avvenire, successivamente, la reazione~\ref{re:pp3}, dando origine alla catena PPI. Tuttavia, non si tratta dell'unica possibilità. Infatti, con una probabilità residua del $31\%$, può avvenire la seguente reazione:
\reaction[re:ppsecondaria]{He^3 + He^4 -> Be^7 + $\gamma$}
Se inizialmente prevale la PPI, dopo un po' che questa è attiva l'ambiente si popola di \ce{He^2} e può facilmente avvenire la~\ref{re:ppsecondaria}, che produce energia. Se è attivo tale canale, quasi sempre la catena prosegue con la PPII (fig.~\ref{fig:catena-pp2}), con una probabilità $P_2=99.7\%$, nella seguente maniera:
\reaction[re:ppII1]{Be^7 + e^- -> Li^7 + $\nu$}
\reaction[re:ppII2]{Li^7 + H^1 -> Be^8}
\reaction[re:ppII3]{Be^8 -> 2 He^4 + $\gamma$} 
In particolare, nella~\ref{re:ppII3} il \ce{Be^8} è instabile e si spacca in due nuclei di \ce{He^4}, producendo energia, a causa della sua elevata stabilità, come si nota anche in fig.~\ref{fig:energia-legame}.

\begin{figure}
    \centering
    \includegraphics[width=0.5\textwidth]{immagini/catena-pp2.png}
    \caption{Catena protone-protone (PPII).}
    \label{fig:catena-pp2}
\end{figure}

Con una probabilità residua di $P_3 = 0.3\%$, la \ref{re:ppsecondaria} può proseguire con una catena PPIII (fig.~\ref{fig:catena-pp3}), nella seguente maniera:
\reaction[re:ppIII1]{Be^7 + H^1 -> B^8 + $\gamma$}
\reaction[re:PPIII2]{B^8 -> Be^8 + e^+ + $\gamma$}
\reaction[re:PPIII3]{Be^8 -> 2 He^4 + $\gamma$}

\begin{figure}
    \centering
    \includegraphics[width=0.5\textwidth]{immagini/catena-pp3.png}
    \caption{Catena protone-protone (PPIII).}
    \label{fig:catena-pp3}
\end{figure}

In tutti i casi, tuttavia, brucio \ce{4 H} per generare \ce{1 He^4}, producendo un'energia di $\sim \SI{20}{Mev}$. In particolare si ha: $E_\textup{PPI} = \SI{26.2}{Mev}$, $E_\textup{PPII} = \SI{25.7}{Mev}$ e $E_\textup{PPIII} = \SI{19.3}{Mev}$. 

\subsection{Catena CNO}\label{sec:catena-cno}
Un'altra possibile catena di bruciamento dell'idrogeno, è la così detta \emph{catena CNO}, detta anche \emph{CN--NO}. A differenza della precedente (par.~\ref{sec:catena-pp}), richiede la presenza di carbonio, azoto e ossigeno. Si noti che questi ultimi elementi \emph{non} sono prodotti dalla catena di reazione, ma devono essere già presenti nel gas, agendo come catalizzatori. Il ciclo principale della catena è raffigurato in fig.~\ref{fig:catena-cno} e può essere sintetizzato dalle seguenti reazioni:
\reaction[re:cno1]{C^{12} + H^1 -> N^{13} + $\gamma$}
\reaction[re:cno2]{N^{13} -> C^{13} + e^+ + $\nu$}
\reaction[re:cno3]{C^{13} + H^1 -> N^{14} + $\gamma$}
\reaction[re:cno4]{N^{14} + H^1 -> O^{15} + $\gamma$}
\reaction[re:cno5]{O^{15} -> N^{15} + e^+ + $\nu$}
\reaction[re:cno6]{M^{15} + H^1 -> C^{12} + He^4}

\begin{figure}
    \centering
    \includegraphics[width=0.5\textwidth]{immagini/catena-cno.png}
    \caption{Catena CN--NO. Partendo da \ce{C^{12}}, la parte sinistra rappresenta il ramo rapido, mentre la parte destra, che parte da \ce{N^{14}} rappresenta il ramo lento.}
    \label{fig:catena-cno}
\end{figure}

In totale il ciclo utilizza \ce{4 H} e genera \ce{1 He^4}. Facendo un computo delle energie, si ha:
\begin{equation}\label{eq:energia-cno}
E_\textup{CNO} = \SI{25}{MeV} \simeq \SI{2e-5}{erg}
\end{equation}
dello stesso ordine di grandezza dell'energia delle catene PP~\eqref{eq:energia-pp1}. Senza approfondire tutti i tempi scala, si sottolinea che quello della reazione~\ref{re:cno4} è dell'ordine di $\SI{3.2e8}{yr}$, dividendo la catena in un ramo rapido (da~\ref{re:cno1} a~\ref{re:cno4}) e un ramo lento (da~\ref{re:cno4} a~\ref{re:cno6} e nuovamente~\ref{re:cno1}). Quindi, nelle stelle in cui avviene il ciclo CNO mi aspetto che, se il materiale processato va verso la superficie per convezione, deve esserci una variazione delle abbondanze chimiche. In particolare, mi aspetto che l'abbondanza del carbonio diminuisca e quella dell'azoto e dell'ossigeno aumentino. Infine, sottolineiamo che questa reazione è un prototipo di reazione di cattura protonica.

\subsection{Il problema dell'elio}
Dopo aver visto alcune catene di bruciamento dell'idrogeno, soffermiamoci sul \emph{problema dell'elio}. Esso consiste nel fatto che l'abbondanza di \ce{He} che si misura nell'Universo è troppo alta per essere spiegata solo in base al bruciamento dell'idrogeno negli interni stellari.

Considerato che nell'Universo si misura un'abbondanza di \ce{He} pari a
\begin{equation}\label{eq:abbondanza-elio-misurata}
    Y \sim 0.24 - 0.28
\end{equation}

Stimiamo, quindi, quanto \ce{He} può essere stato prodotto dalle stella da quando si è formato l'Universo, ovvero in un tempo di Hubble ($t_H \sim \SI{13}{Gyr}$), facendo un conto per la Via Lattea, della quale conosciamo la massa $M_G$ e la luminosità $L_G$
\[
M_G \sim \SI{e12}{\solarmass} \qquad M_G \sim \SI{e11}{\solarluminosity}
\]
e supponendo che tutta la luminosità della Galassia venga dal bruciamento di idrogeno in elio e che sia rimasta sempre costante. In questo modo dovrei sovrastimare il valore reale. Ricordando che la luminosità è l'energia prodotta per unità di tempo (par.~\ref{sec:luminosità}), per $L_G$, in un tempo $t_H$ si trova un'energia:
\[
E_\textup{tot} \simeq L_G t_H \sim \SI{1.6e62}{erg}
\]
che rappresenta l'energia totale prodotta dal bruciamento di \ce{H} in \ce{He} da quando si è formata la Galassia. Considerato che l'energia di legame di un nucleo di \ce{He} è
\[
E_\textup{b, He} \sim \SI{4.5e-5}{erg}
\]
e che può essere pensata come l'energia prodotta dal bruciamento di 4 nuclei di \ce{H} in un nucleo di \ce{He}, possiamo trovare il numero di atomi di \ce{He^4} che si sono formati dalla formazione della Galassia:
\[
N_\textup{He} = \frac{E_\textup{tot}}{E_\textup{b, He}} \simeq \SI{3.5e66}{}
\]
Possiamo pensare questo numero come il numero di reazioni di bruciamento di idrogeno che sono avvenute. Ora, dalla massa di un atomo di elio, $m_\textup{He} \sim \SI{6.64e-24}{g}$, possiamo ricavare la massa totale di \ce{He} prodotta in un tempo di Hubble:
\[
M_\textup{He} = N_\textup{He} m_\textup{He} \sim \SI{2.4e43}{g}
\]
Infine, usando $M_G$ possiamo trovare la frazione in massa di elio prodotta dalle Via Lattea da quando l'universo si è formato:
\begin{equation}\label{eq:abbondanza-elio-stimata}
Y = \frac{M_\textup{He}}{M_G} \sim 0.01
\end{equation}
Confrontando la~\eqref{eq:abbondanza-elio-misurata} con la~\eqref{eq:abbondanza-elio-stimata}, notiamo come l'abbondanza di elio misurata sia circa 20 volte maggiore di quella stimata. Questo significa che una frazione rilevante di \ce{He} deve essere stata prodotta da un altro processo, molto più efficiente e primordiale: si tratta del \emph{Big Bang}. Infatti, un punto rilevante che esso riesce a spiegare è la formazione di \emph{deuterio} (\ce{H^2}), processo molto difficile nelle stelle a causa della carenza di neutroni liberi, che decadono spontaneamente in protoni per il decadimento $\beta^-$ (~\ref{re:decadimento-beta-}). Tuttavia, nei primi minuti successivi al Big Banda, c'erano molti neutroni liberi disponibili, e dunque la seguente catena di reazioni spiega la formazione di una grande quantità di elio subito dopo il Big Bang:
\reaction[re:bb1]{n + p -> H^2 + $\gamma$}
\reaction[re:bb2]{H^2 + H^2 -> He^3 + n}
\reaction[re:bb3]{He^3 + n -> H^3 + p}
\reaction[re:bb4]{H^2 + H^3 -> He^4 + n}
Tale \emph{nucleosintesi primordiale} si è fermata all'elio e non ha prodotto elementi più pesanti, poiché le reazioni successive di cattura neutronica o protonica avrebbero prodotto elementi instabili. In definitiva, riteniamo che gran parte dell'elio sia stato prodotto durante il Big Bang, mentre gli elementi più pesanti siano prodotti negli interni stellari.
 
\subsection{Catena 3-alpha}
La \emph{catena 3-$\alpha$} è una catena di bruciamento dell'elio in carbonio. Si ricordi che una particella $\alpha$ non è altro che un nucleo di elio. Vediamo quando tale catena si innesca.

Quando il nucleo stellare è costituito quasi interamente da elio, le reazioni di bruciamento dell'idrogeno cessano e la struttura va fuori dall'equilibrio idrostatico. In questa situazione prevale la forza gravitazionale che fa contrarre il nucleo, provocando un conseguente aumento della temperatura. Quando le temperature raggiungono $T \sim \SI{1.5e8}{K}$, può avvenire la reazione di fusione termonucleare di \ce{He}. Le due reazioni salienti della catena 3-$\alpha$ sono le seguenti:
\reaction[re:3alpha1]{He^4 + He^4 <--> Be^8}
\reaction[re:3alpha2]{Be^8 + He^4 -> C^12 + $\gamma$}
Come si può notare, la reazione~\ref{re:3alpha1} è reversibile, e in particolare, a causa dell'elevata instabilità di \ce{Be^8}, appena questo si forma, in un tempo molto breve tende a spaccarsi nuovamente in \ce{2 He^4}. Tuttavia, in questo caso l'abbondanza di \ce{He} è così elevata che \ce{Be^8}, prima di decadere, si fonde con un altro \ce{He^4}, secondo la reazione~\ref{re:3alpha2}.

In definitiva \ce{2 He^4} vengono trasformati in un \ce{C^12}. Facendo un computo dell'energia si trova:
\begin{equation}\label{eq:energia-3alpha}
    E_{3\alpha} = \SI{7.3}{MeV} \simeq \SI{1.2e-5}{erg}
\end{equation}
e possiamo notare che viene prodotta molta meno energia dei processi di bruciamento dell'idrogeno,~\eqref{eq:energia-pp1} e~\eqref{eq:energia-cno}. In particolare, l'energia per unità di massa prodotta dal bruciamento dell'elio è $\sim 10\%$ di quella prodotta dal bruciamento dell'idrogeno. Questo ha un impatto sui \emph{tempi di evoluzione stellare}: la fase in cui viene bruciato idrogeno nel nucleo, detta \emph{sequenza principale} è molto più lunga della fase in cui viene bruciato elio nel nucleo, detta \emph{red clump} o \emph{horizontal branch}.

A questo punto, se la struttura è termoregolata (gas non degenere), esaurito l'elio, possono continuare processi di bruciamento di materiali sempre più pesanti, fino al silicio, che genera ferro. Infatti, dopo il ferro (fig.~\ref{fig:energia-legame}) i processi di fusione sono endotermici. Non ci soffermeremo sui dettagli di tali reazioni, tuttavia vogliamo richiamare l'attenzione sul fatto che durante tutte queste reazioni vengono prodotte tante particelle $\alpha$. Che fine fanno?

\subsection{Processi di cattura alpha}\label{sec:cattura-alpha}
Le particelle $\alpha$ possono essere catturate da particelle più pesanti generando elementi pesanti attraverso reazioni che necessitano di temperature più basse della fusione. Gli elementi prodotti da queste reazioni sono detti \emph{elementi-$\alpha$} e sono sostanzialmente tutti gli elementi dal carbonio al silicio. In generale, è più conveniente per un atomo pesante catturare una particella $\alpha$ che fondersi con essa, perché la temperatura di innesco è più bassa. Ovviamente, posso arrivare fino al ferro. E per gli elementi più pesanti?

\subsection{Processi di cattura neutronica}
Attraverso processi di \emph{cattura neutronica} si spiega la presenza di elementi più pesanti del ferro. Essi consistono nella cattura di un neutrone da parte di un elemento, il quale rimane lo stesso elemento chimico ($Z$ non varia), ma diventa un suo isotopo (il numero di massa $A$ aumenta di $1$). Tale isotopo spesso è instabile e decade in un elemento che ha un numero atomico maggiore. Si può riassumere nella seguente reazione:
\reaction[re:cattura-neutronica]{^A_ZX + n -> ^{A+1}_ZX -> ^{A+1}_{Z+1}X + e^- + $\Bar{\nu}$}
La catena può proseguire in questo moto, tuttavia non scendiamo nei dettagli. In generale, le catture elettroniche si dividono in catture \emph{lente}, che producono \emph{elementi s}, e catture \emph{rapide}, che producono \emph{elementi r}. Per capire in quale dei due casi si è, si compara il tempo di decadimento dell'isotopo con il tempo di acquisizione del neutrone. 

Affinché le reazioni di cattura neutronica possano avvenire, è necessario che ci siano neutroni liberi che possano essere catturati, tuttavia, per il decadimento $\beta^-$ (\ref{re:decadimento-beta-}), il quale è molto veloce e probabile, un neutrone tende a decadere in protone.  Le principali sorgenti di neutroni per i processi s sono i processi $\alpha$ (par.~\ref{sec:cattura-alpha}), i quali avvengono nel \emph{ramo asintotico delle giganti}, come vedremo in seguito. Per i processi r, invece, la fonte principale di neutroni liberi è la fotodisintegrazione del ferro, che avviene a temperature alte durante l'esplosione di una supernova di tipo II, di cui parleremo successivamente. 

\subsection{Tasso di produzione di energia}
I processi visti finora, riassunti in tab.~\ref{tab:processi-produzione-energia} e i cui prodotti sono raffigurati in fig.~\ref{fig:processi-produzione-energia}, rappresentano la fonte principale di energia in una stella. Ci proponiamo ora di calcolare il \emph{tasso di produzione di energia} $\epsilon$ per una stella, ovvero l'energia prodotta per unità di massa e unità di tempo. 

\begin{figure}
    \centering
    \includegraphics[width=0.5\textwidth]{immagini/processi-produzione-energia.png}
    \caption{Abbondanze degli elementi divise per i processi che li hanno generati in maniera preponderante.}
    \label{fig:processi-produzione-energia}
\end{figure}

\begin{table}
\caption{Principali reazioni nucleari.}
\label{tab:processi-produzione-energia}
\centering
\begin{tabular}{lll}
\toprule
Processo & Temperatura ($\si{K}$) & Tempo scala  ($\si{yr}$)\\
\midrule
Elementi leggeri & $\sim \SI{e6}{}$ & $\sim \SI{e5}{}$ \\
Catena PP & $\sim \SI{e7}{}$ & $\sim \SI{e10}{}$ \\
Catena CNO & $\sim \SI{e7}{}$ & $ \SI{e9}{}$\\
Catena 3--$\alpha$ & $> \SI{e8}{}$ & $\SI{e7}{}$ \\
Bruciamento carbonio & $\sim \SI{e9}{}$ & $\SI{e5}{}$ \\
Bruciamento Ossigeno & $> \SI{e9}{}$ & $\SI{e5}{}$ \\
Processo s & $> \SI{e8}{}$ & $\SI{e3}{}$--$\SI{e7}{}$ \\
Processo r & $> \SI{e10}{}$ & $\SI{10}{s}$--$\SI{100}{s}$ \\
\bottomrule
\end{tabular}
\end{table}

Essa può essere stimata così:
\begin{equation}
    \epsilon = \sum_{i=1}^{N_r} E_r \dfrac{N_r}{\si{g.s}} = \sum_{i=1}^{N_r} E_r \dfrac{N_r}{\si{cm^3.s}} \dfrac{\si{cm^3}}{\si{g}}
\end{equation}
dove con $i \in [N_r]$ si sta sommando sulle varie reazioni della catena in considerazione, $E_r$ è l'energia prodotta da ciascuna reazione, $N_r/\si{cm^3.s}$ è il numero di reazioni per unità di volume e di tempo e $\si{cm^3} / \si{g}$ rappresenta l'inverso della densità della materia stellare. Essa dipende da:
\begin{itemize}
    \item Carburante disponibile
    \item Temperatura (le reazioni si innescano dopo una certa soglia)
    \item Densità
\end{itemize}

In un modello stellare si esprime $\epsilon$ in maniera approssimata, in base alle proprietà macroscopica della struttura. In particolare possiamo scrivere:
\begin{subequations}
\label{eq:energia-reazioni-termonucleari}
\begin{align}
\epsilon_\textup{PP} &= \epsilon_1 \rho X^2 {T_6}^{\nu_\textup{PP}} \qquad \qquad \nu_\textup{PP} \in [3.5 - 6] \label{eq:energia-PP}\\
\epsilon_\textup{CN} &= \epsilon_2 \rho X X_\textup{CN} {T_6}^{\nu_\textup{CN}} \qquad \nu_\textup{CN} \in [13 - 20] \label{eq:energia-CN}\\
\epsilon_{3\alpha} &= \epsilon_3 \rho^2 Y^3 {T_8}^{\nu_{3\alpha}} \qquad \qquad \nu_{3 \alpha} \in [20 - 30] \label{eq:energia-3A}
\end{align}
\end{subequations}
dove $T_6$ significa che la temperatura è espressa in unità di $\SI{e6}{K}$, mentre $T_8$ significa che la temperatura è espressa in unità di $\SI{e8}{K}$. $X$ è l'abbondanza dell'idrogeno, $Y$ l'abbondanza dell'elio e $X_\textup{CN}$ l'abbondanza di carbonio e azoto. Si noti la mostruosa dipendenza dalla temperatura delle~\eqref{eq:energia-reazioni-termonucleari}, il che evidenzia che reazioni termonucleari stabili possono avvenire solamente in ambienti termonucleari, ovvero in cui il gas può essere considerato perfetto. Infatti, in ambienti degeneri, in cui la pressione non dipende dalla temperatura, siccome piccole variazioni di energia producono una variazione enorme della creazione di energia, la struttura può esplodere. Come si può notare, queste sono le espressioni che compaiono nell'equazione~\eqref{eq:reazioni-termonucleari}. 

\section{Riassunto sul modello stellare.}
Con le 7 equazioni presentate nel par.~\ref{sec:modelli-stellari} abbiamo un sistema di 7 equazioni in 7 incognite per descrivere la struttura di una stella. In particolare, le incognite sono:
\begin{itemize}
    \item La pressione $P$
    \item La massa $M$
    \item La densità $\rho$
    \item La luminosità $L$
    \item L'opacitk $\kappa$
    \item Il tasso di produzione di energia $\epsilon$
\end{itemize}
Queste equazioni descrivono l'interno stellare di una stella , dandomi in output la sua \emph{luminosità}. L'altro parametro di cui ho bisogno è la \emph{temperatura superficiale} della stella. Per ottenerla ho bisogno di un modello della sua atmosfera.

%\chapterimage{}
\chapter{Modello di atmosfera stellare}
\section{Profondità ottica}\label{sec:profondità-ottica}
Per realizzare un modello dell'atmosfera stellare si procede in maniera simile alla modellizzazione dell'interno stellare. Tuttavia, come variabile, invece della distanza dal centro, $r$, si utilizza la \emph{profondità ottica}. Infatti, le distanze dal centro sono molto elevate, pertanto è conveniente avere una misura della distanza dalla superficie della stella. 

\subsection{Profondità ottica verticale}
La profondità ottica è definita nella seguente maniera:
\begin{equation}\label{eq:profondità-ottica-atmosfera}
    \tau_\lambda = \int \kappa_\lambda \rho \ud S
\end{equation}
dove $\kappa$ è l'opacità, misurata in $\si{cm^2.g^{-1}}$, $\rho$ è la densità di massa, misurata in $\si{g.cm^{-3}}$ e $S$ rappresenta la distanza lungo una direzione radiale, misurata in $\si{cm}$.

\begin{figure}
    \centering
    \includegraphics[width=0.3\textwidth]{immagini/profondita-ottica-atmosfera.png}
    \caption{La profondità ottica non è una misura univoca della distanza dalla superficie. Se scelgo tre direzioni radiali, in blu, rosso e verde, ottengo prodondità ottiche diverse in uno stesso punto P, in particolare $\tau_{1\lambda} \neq \tau_{2\lambda} \neq \tau_{3\lambda}$. L'unica cosa uguale per le tre situazioni è che in superficie la profondità ottica è nulla.}
    \label{fig:profondità-ottica-atmosfera}
\end{figure}

In particolare, $\tau_\lambda = 0$ sulla superficie della stella, mentre $\tau_\lambda$ cresce andando verso il centro, seguendo un cammino radiale. Come si può notare in fig.~\ref{fig:profondità-ottica-atmosfera}, la profondità ottica \emph{non} è una misura \emph{unica} della distanza dalla superficie, infatti, se considero raggi con direzioni diverse, ottengo profondità ottiche diverse. Per utilizzare questa grandezza per misurare univocamente le distanze, si considera la \emph{profondità ottica verticale}, dove integro lungo una direzione radiale:
\begin{equation}\label{eq:profondità-ottica-verticale}
    \tau_{\lambda, V} = \int_z^0 \kappa_\lambda \rho \ud z
\end{equation}
dove la direzione identificata da $z$ è perpendicolare alla superficie della stella. A questo punto possiamo utilizzare la profondità ottica per determinare quale è lo strato più interno dal quale la radiazione a una data lunghezza d'onda riesce ad emergere verso l’esterno, o, in altre parole, la profondità ottica mi dice quanto è schermata la radiazione che viene dall'interno. D'ora in poi, quando si parlerà di profondità ottica si farà riferimento alla profondità ottica verticale.

\subsection{Temperatura effettiva}
Sia la profondità ottica che la temperatura aumentano andando verso il centro della struttura stellare. La loro relazione definisce univocamente la così detta \emph{temperatura effettiva} della stella, ed è un'equazione chiave per un modello di atmosfera:
\begin{equation}\label{temperatura-effettiva}
    T^4 = \frac{3}{4} {T_e}^4 \bigl( \tau_\nu + \frac{2}{3} \bigr)
\end{equation}
La temperatura effettiva $T_e$ è la temperatura alla quale la profondità ottica vale $\tau_\lambda = 2 / 3$. Posso interpretare lo strato in cui ciò è vero come lo strato più interno da cui riesce ad emergere la radiazione. Per il sole, ad esempio, si ha $T_e = \SI{5770}{K}$.


\section{Righe Spettrali}\label{sec:righe-spettrali}

\subsection{Opacità per l'atmosfera stellare}\label{sec:opacita-atmosfera}
Un parametro importante per la determinazione di un modello per l'atmosfera stellare è l'opacità, già discussa nel par.~\ref{sec:opacità}. Ripassiamo i concetti principali. I processi che determinano l'opacità di una struttura sono:
\begin{description}
    \item[BB] discusso nel par.~\ref{sec:bound-bound}. Avviene a una determinata lunghezza d'onda, secondo l'eq.~\eqref{eq:lunghezza-BB}. È responsabile delle \emph{righe di assorbimento}, poiché determinano la mancanza di flusso a una determinata lunghezza d'onda.
    \item[BF] discusso nel par.~\ref{sec:bound-free}. Contribuisce all'opacità del \emph{continuo}.
    \item[FF] discusso nel par.~\ref{sec:free-free}. Contribuisce all'opacità del \emph{continuo}.
    \item[E] discusso nel par.~\ref{sec:electron-scattering}. Contribuisce all'opacità del \emph{continuo}.
\end{description}
In particolare, nelle atmosfere le temperature sono sufficientemente basse affinché i fotoni possano essere poco energetici e possano eccitare gli elettroni degli atomi, lasciandoli in stati legati. Dunque il fenomeno del BB, che negli interni stellari trascuravamo a causa delle elevate temperatura, sarà determinante nelle atmosfere. La velocità di variazione dell'opacità con la lunghezza d'onda determina come l'opacità stessa si manifesta, ovvero in forma di righe di assorbimento o nel continuo. Vediamo degli esempi.

\paragraph{Esempio--Serie di Balmer}
\paragraph{Esempio--Balmer Jump}

\subsection{Continuo spettrale e righe spettrali di assorbimento}

\subsection{Classificazione spettrale delle stelle}
Come visto nel par.~\ref{sec:opacita-atmosfera}, l'\emph{atmosfera} stellare è dove si formano il \emph{continuo spettrale} e le \emph{righe spettrali di assorbimento}. Le righe di assorbimento, in particolare, sono degli osservabili fondamentali perché è dalla loro intensità che possiamo misurare l'\emph{abbondanza} dei diversi elementi chimici. Si faccia attenzione al seguente fatto: la presenza o meno di una data riga spettrale \emph{non} dipende dalla presenza o meno di quel dato elemento, ma è fortemente modulata dalla \emph{temperatura}. Vediamo di chiarire il fatto.

Ovviamente, se un elemento non è presente nella struttura stellare, lo spettro non potrà mostrare le sue righe. Tuttavia, l'assenza di righe di un dato elemento \emph{non necessariamente} significa che quel dato elemento è assente: potrebbe esserci, ma la \emph{temperatura} non è tale da far manifestare le sue righe. Infatti, le righe di assorbimento sono dovute a fenomeni di eccitazioni o ionizzazioni della materia e, come noto,  il numero di stati eccitati o ionizzati dipende primariamente dalla temperatura. Se, ad esempio, la temperatura non è sufficientemente alta da far sì che tutti gli atomi di un dato elemento siano ionizzati, non potrò mai vedere la riga di assorbimento di tale elemento, anche se esso è presente.

In definitiva, è la temperatura che modula il manifestarsi delle righe spettrali, ovvero è la \emph{temperatura} che determina il \emph{tipo spettrale} delle stelle. Dunque, le stelle sono classificate in base al loro tipo spettrale, ovvero in base alla presenza e all'intensità delle varie righe spettrali, il quale, a sua volta, riflette il valore della temperatura atmosferica della stella.

\begin{figure}
    \centering
    \includegraphics[width=0.7\textwidth]{immagini/classificazione-spettrale-stelle.png}
    \caption{Classi spettrali principali.}
    \label{fig:classificazione-spettrale-stelle}
\end{figure}

\begin{figure}
    \centering
    \includegraphics[width=0.7\textwidth]{immagini/classificazione-spettrale-stelle-2.png}
    \caption{Classi spettrali principali.}
    \label{fig:classificazione-spettrale-stelle-2}
    
\end{figure}

Nelle fig.~\ref{fig:classificazione-spettrale-stelle} e fig.~\ref{fig:classificazione-spettrale-stelle-2} sono rappresentate le principali \emph{classi spettrali}, elencate anche di seguito:
\begin{description}
    \item[O] $T_e > \SI{25000}{K}$
    \item[B] $ \SI{11000}{K} < T_e < \SI{25000}{K}$: ???
    \item[A] $ \SI{7500}{K} < T_e < \SI{11000}{K} $: ???
    \item[F] $ \SI{6000}{K} < T_e < \SI{7500}{K} $: ???
    \item[G] $ \SI{5000}{K} < T_e < \SI{6000}{K} $: ???
    \item[K] $ \SI{3500}{K} < T_e < \SI{3000}{K} $: ???
    \item[M] $ \SI{3500}{K} < T_e < \SI{3000}{K} $: ???
\end{description}
Si può ricordare tale lista con la seguente frase: \emph{"O,B,A, Fine Girl Kiss Me"}.

Come detto precedentemente, una data riga di assorbimento è presente o assente nello spettro a seconda del numero di atomi nei diversi stati eccitati o ionizzati di quel dato elemento. Tale numero dipende in primo luogo dalla temperatura e viene stimato attraverso:
\begin{description}
    \item[Equazione di Boltzmann]: percentuale di atomi in un dato stato di eccitazione. Risponde alla domanda: "quale frazione di atomi con elettroni legati si trova in un dato stato eccitato?"
    \item[Equazione di Saha]: percentuale di atomi in un dato stato di ionizzazione. Risponde alla domanda: "questo elemento ha ancora elettroni legati?"
\end{description}

\subsection{Equazione di Boltzmann}
Come suggerito nel precedente paragrafo, l'\emph{equazione di Boltzmann} fornisce la percenutale di atomi in un dato stato di eccitazione. In particolare, per ogni specie chimica, considera gli atomi ionizzati $j$--volte ($N_j$) e fornisce la frazioen di quelli che sono eccitati $i$--volte ($N_{ji}$):
\begin{equation}\label{eq:equazione-boltzmann}
    \dfrac{N_{ji}}{N_j} = \dfrac{g_i}{{U_j}(T)} 10^{-\theta \chi_i} 
\end{equation}
dove:
\begin{description}
    \item[$N_j$]: numero di atomi nello stato di ionizzazioni j (ovvero ionizzati $j$--volte).
    \item[$N_{ji}$]: numero di atomi nello stato di ionizzazione j, che si trovano nello stato di eccitazioni i (ovvero ionizzati $j$--volte \emph{ed} eccitati $i$--volte).
    \item[$g_i$]: peso statistico del livello energetico i.
    \item[$\theta \equiv \frac{5040}{T} \si{eV^{-1}}$], con $T$ espressa in $\si{K}$.
    \item[${U_j}(T)$]: funzione di partizione per lo stato di ionizzazione j.
    \item[$\chi_i$]: potenziale di eccitazione dal primo livello energetico disponibile al livello energetico $i$.      
\end{description}
Notiamo che l'espressione è dipendente dalla struttura dell'atomo, attraverso i termini $g_i$, ${U_j}(T)$ e $\chi_i$, e dalla temperatura. In definitiva, la percentuale è tanto maggiore quanto più alta è $T$ e quanto più basso è il potenziale di eccitazione $\chi_i$. Vediamo, ora, i vari termini con maggior dettaglio.

\paragraph{Peso statistico}
Il peso statistico $g_i$ del livello energetico $i$--esimo rappresenta il numero di livelli energetici degeneri, ovvero alla stessa energia. Per gli atomi idrogenoidi si ha:
\begin{equation*}
g_i = 2 i^2
\end{equation*}
e si ottengono i noti risultati per cui il nel livello fondamentale ($i=1$) si ha $g_1 = 2$, ovvero possono alloggiare $2$ elettroni di spin opposto, mentre nel primo livello eccitato, $i=2$, vale $g_2=8$ e possono alloggiare $2$ elettroni nell'orbitale s e $6$ elettroni nell'orbitale p, e così via.

\paragraph{Funzione di partizione}
La funzione di partizione per lo stato di ionizzazione $j$--esimo, ${U_j}(T)$, è una sommatoria dei pesi statistici ($g_i$) di tutti i livelli energetici pesati con un termine che dipende dalla temperatura, ovvero:
\begin{equation*}
    {U_j}(T) = \sum_i g_i 10^{-\theta \chi_i}
\end{equation*}

\paragraph{Potenziale di eccitazione}
nell'equazione~\eqref{eq:equazione-boltzmann}, $\chi_i$ rappresenta il potenziale di eccitazione dal primo livello energetico \emph{disponibile} al livello energetico $i$--esimo. Vediamo cosa si intende per primo livello disponibile, il quale dipenderà dal livello di ionizzazione, j. Consideriamo, ad esempio, \ce{NeI}, ovvero il neon neutro. Esso ha 10\ce{e^-} legati, in cui $2$\ce{e^-} si trovano nel livello $n=2$ e $8$\ce{e^-} si trovano nel livello $n=2$. Essendo, dunque, i primi due livelli totalmente occupati, il primo livello disponibile sarà $n=3$. Nel caso di \ce{NeII}, ovvero del neon ionizzato $1$ volta, il primo livello disponibile è $n=2$, non essendo questa volta occupato del tutto. 

Per gli atomi idrogenoidi, il potenziale di eccitazione tra due livelli energetici $a$ e $b$, con $a<b$, è:
\[
    \chi_{ab} = Z^2 \bigl( \dfrac{1}{{n_a}^2} - \dfrac{1}{{n_b}^2} \bigr) \times \SI{13.6}{eV}
\]
con $Z$ il numero atomico. Essendo il primo livello energetico disponibile il livello fondamentale, rappresentato da $a=1$, si ha:
\[
  \chi_i = Z^2 \bigl( 1-\dfrac{1}{i^2}  \bigr)  \times \SI{13.6}{eV}
\]

\subsection{Equazione di Saha}
Per ogni specie chimica, l'\emph{equazione di Saha} fornisce la percentuale di atomi ionizzati $j+1$--volte ($N_{j+1}$) rispetto al numero di atomi ionizzati $j$--volte ($N_j$):
\begin{equation}\label{eq:equazione-saha}
    \log \dfrac{N_{j+1}}{N_j} = -0.176 - \log P_e - \theta \chi_i + 2.5 \log T + \log \dfrac{{U_{j+1}}(T)}{{U_j}(T)}
\end{equation}
dove:
\begin{description}
    \item[$N_j$, $N_{j+1}$]: numero di atomi nello stato di ionizzazione $j$ e $j+1$, ovvero contigui.
    \item[$P_e$]: pressione elettronica, ovvero esercitata dalla componente elettronica del gas. Si ricordi che siamo nelle atmosfere stellari, in cui il gas è sempre approssimabile come gas perfetto.
    \item[$\theta \equiv \frac{5040}{T} \si{eV^{-1}}$], con $T$ espressa in $\si{K}$.
    \item[$\chi_i$]: potenziale di ionizzazione dell'atomo ionizzato $j$--volte. ERRORE FORSE, NON DOVREBBE ESSERE CHI J??????
    \item[${U_j}(T)$, ${U_{j+1}}(T)$]: funzioni di partizione per gli stati di ionizzazione $j$ e $j+1$.
\end{description}
Si faccia molta attenzione a cosa si riferisce l'eq.~\eqref{eq:equazione-saha}. Essa \emph{non} permette di calcolare il numero di atomi in un certo stato di ionizzazione ( $N_j$ ) rispetto al numero totale ($N$) di atomi di quella specie, che sarebbe equivalente a $N_j / N$, tuttavia essa calcola il rapporto tra il numero di atomi in due stati di ionizzaione contigui ($j+1$ e $j$), equivalente a $N_{j+1} / N_j$. Per ottenere $N_j / N$ sono necessarie applicazioni successive dell'equazione di Saha tra due stati di ionizzazione contigui.

\subsection{Frazione di atomi attivi}\label{sec:frazione-atomi-attivi}
In pratica, mettendo insieme le due equazioni, per ogni equazione chimica:
\begin{itemize}
    \item l'equazione di Saha~\eqref{eq:equazione-saha} mi dice se, a quella data temperatura $T$ esistono atomi che non sono completamente ionizzati, cioè che hanno ancora elettroni legati
    \item se ciò è vero, l'equazione di Boltzmann~\eqref{eq:equazione-boltzmann} mi dice se, a quella data temperatura $T$ gli elettorni legati sono nello stato fondamentale o in quale livello di eccitazione
\end{itemize}
Insieme, le eq.~\eqref{eq:equazione-saha} e~\eqref{eq:equazione-boltzmann} forniscono la \emph{frazione di atomi attivi} $N_a$ (che generano le righe spettrali) rispetto al totale di atomi di quella specie. È possibile misurare $N_a$ attraverso un'analisi spettroscopica delle righe spettrali, dunque, in definitiva, usando le equazioni di Saha e Boltzmann è possibile ricavare l'\emph{abbondanza} di quel dato elemento chimico. 

\subsection{Analisi delle righe spettrali}
Come discusso nel par.~\ref{sec:opacita-atmosfera}, le linee spettrali sono originate dalle transizioni elettroni che di tipo bound--bound. Nelle righe spettrali sono contenute informazioni di cruciale importanza, tra cui le \emph{abbondanze chimiche} e la \emph{velocità radiale}.

\begin{figure}
    \centering
    \includegraphics[width=0.5\textwidth]{immagini/righe-spettrali.png}
    \caption{Misura di una riga spettrale e analisi per la determinazione della velocità radiale e delle abbondanze chimiche. La riga spettrale di una transizione elettronica B--B è misurata a lunghezza d'onda $\lambda_1$, confrontata con la lunghezza d'onda di laboratorio di tale transizione $\lambda_0$.}
    \label{fig:righe-spettrali}
    
\end{figure}

Per capire come si procede sperimentalmente all'analisi delle righe spettrali, consideriamo una situazione tipo, illustrata in fig.~\ref{fig:righe-spettrali}. Immaginiamo di misurare una riga spettrale corrispondente a una data transizione elettronica a una lunghezza d'onda $\lambda_1$, diversa, in generale, dalla lunghezza d'onda di laboratorio di tale transizione, $\lambda_0$. A causa dell'\emph{effetto Doppler}, la differenza tra la lunghezza d'onda misurata e quella di laboratorio è una misura diretta della \emph{velocità radiale} della stella. Infatti si ha:
\begin{equation}\label{eq:effetto-doppler}
    \dfrac{\lambda_1 - \lambda_0}{\lambda_0} = \dfrac{\Delta \lambda}{\lambda_0} = \dfrac{v_r}{c}
\end{equation}

Per ciò che concerne le \emph{abbondanze chimiche}, è necessario misurare l'\emph{intensità} delle righe spettrali prodotte dagli atomi della specie studiata attraverso transizioni BB. In particolare, l'intensità, come mostrato in fig.~\ref{fig:righe-spettrali}, è rappresentata dall'area sotto la curva dello spettro continuo. È tuttavia comunemente misurata in termini di \emph{larghezza equivalente}. La larghezza equivalente $W$ è la larghezza (in Armstrong) di un rettangolo di altezza unitaria e area uguale all'intensità della riga spettrale. Si può calcolare nella seguente maniera:
\begin{equation}\label{eq:larghezza-equivalente}
    W = \int \dfrac{F_c - F_\lambda}{F_c} \ud \lambda
\end{equation} 
dove $F$ rappresenta il flusso. In particolare $W$ è correlato con il numero di atomi per $\si{cm^2}$ che generano la transizione, ovvero con $N_a$, il numero di atomi attivi. Come è possibile osservare in figura~\ref{fig:righe-spettrali-numero-atomi}, la riga spettrale è tanto più profonda quanto maggiore è l'abbondanza dell'elemento considerato. Siccome la profondità della riga dipende dal numero di atomi attivi, anche $W$ dipenderà da $N_a$. In particolare è possibile osservare la loro correlazione in una \emph{curva di crescita}, mostrata in fig.~\ref{fig:curva-crescita}.

\begin{figure}
    \centering
    \includegraphics[width=0.5\textwidth]{immagini/righe-spettrali-numero-atomi.png}
    \caption{Correlazione tra la larghezza equivalente $W$ e il numero di atomi attivi $N_a$. La riga spettrale è tanto più profonda quanto maggiore è l'abbondanza dell'elemento considerato}
    \label{fig:righe-spettrali-numero-atomi}
\end{figure}

\begin{figure}
    \centering
    \includegraphics[width=0.5\textwidth]{immagini/curva-crescita.png}
    \caption{}
    \label{fig:curva-crescita}
\end{figure}

\subsection{Abbondanze chimiche}


\subsubsection{Larghezza equivalente}

\subsection{DA SCRIVERE Altre cose e Recap}

%\chapterimage{}
\chapter{Evoluzione Stellare}
\section{Modello di evoluzione stellare}\label{sec:modello-evoluzione-stellare}

\subsection{Introduzione all'evoluzione stellare}
\subsection{Diagramma H--R e traccia evolutiva}

\section{Pre Main Sequence}\label{sec:pre-main-sequence}

La fase di pre-MS (pre main sequence) è quella in cui si va a formare una stella, partendo dalla contrazione di una nube di gas. Per far si che questa si formi, però, è necessario che siano presenti determinate condizioni. Sia $m$ la massa di una particella d'Idrogeno ai margini della nube gassosa (possibilmente una molecola di $H_2$) e si indichi con $R$ la distanza di questa dal centro. Al fine di osservare la formazione di una stella è necessario che l'attrazione gravitazionale agente sulla particella, dal resto della nube, sia maggiore rispetto alla pressione esercitata dal gas su di essa, imponiamo quindi l'equazione~\eqref{eq:1}
\begin{equation}
    G \frac{M m}{R} \geq k_B T
    \label{eq:1}
\end{equation}
dove $M$ è la massa della nube, $T$ è la temperatura e $k_B$ la costante di Boltzmann.
\subsection{Massa di Jeans}\label{sec:massa-jeans}

Considerando che la massa della nube può essere espressa come segue
\[
    M = \frac{4}{3} \pi R^3 \rho
\]
allora possiamo esplicitare la dipendenza del raggio da alcune caratteristiche della nube, in particolare della massa e della densità.
\begin{equation}
    R = \sqrt[3]{\frac{3}{4\pi} \frac{M}{\rho}}
    \label{eq:2}
\end{equation}
Inserendo l'equazione~\eqref{eq:2} nella~\eqref{eq:1}, si ottiene la relazione in cui compaiono le varie caratteristiche del gas, imponendo le condizioni che permettono la formazione di una stella.
\begin{equation}
    M^{\frac{2}{3}} \geq \frac{k_B T}{G m} \sqrt[3]{\frac{3}{4\pi} \frac{1}{\rho}}
    \label{eq:3}
\end{equation}
Sia $a = \sqrt{\frac{3}{4\pi}}\frac{k_B}{G m} \sim \SI{e16}{g.K^{-1}.cm^{-1}}$ una costante, allora si riscrive la relazione~\eqref{eq:3} nella forma dell'equazione~\eqref{eq:jeans}.
\begin{equation}
    M \geq a^{\frac{3}{2}} T^{\frac{3}{2}} \rho^{-\frac{1}{2}}
\label{eq:jeans}
\end{equation}

Questa è la massa necessaria che permette la contrazione, assumendo che la densità di gas sia uniforme e il valore minimo che soddisfa questa disuguaglianza viene detto massa di Jeans, la quale dipende dai valori di densità e temperatura del gas. In particolare, per temperature elevate e gas rarefatti questa risulta maggiore rispetto al caso con gas densi e relativamente freddi. Di conseguenza si deduce che l'ambiente più favorevole alla formazione stellare è proprio quest'ultimo.

Benché il modello costruito permette di trovare le condizioni iniziali necessarie a favorire la nascita di una stella, il meccanismo che attiva l'effettivo collasso gravitazionale non è ancora ben noto. Le ipotesi più probabili sono quelle dello Shock Front, cioè la compressione dovuta all'esplosione di una supernova, della collisione tra nubi o all'interazione con una galassia.
\subsection{Initial mass function}
Si consideri ora di essere in condizioni interstellari ($T = \SI{10}{K}$ e $ \bar \rho = \SI{e-23}{g.cm^{-2}} $), allora per formare una stella sarebbe necessaria una massa di circa $100 \;M_\odot $, ma noi osserviamo stelle anche decisamente meno massive. Questo significa che in realtà tale meccanismo favorisce la formazione di vari aggregati stellari, con processi frammentati riferiti alla nube iniziale. Definiamo ora con IMF (initial mass function) la distribuzione di massa della stella in formazione all'interno di una popolazione stellare. Questo è utile per studiare i passaggi che una data stella attraversa durante la sua evoluzione, in particolare permette di avere una stima delle seguenti grandezze:
\begin{description}
    \item[Destino]di una stella, cioè quale percorso una stella seguirà nella parte finale della sua vita;
    \item[Durata della vita]di una stella, indica l'ordine di grandezza del tempo necessario alla sua evoluzione;
    \item[Inquinamento]di una stella, ovvero i tipi di elementi chimici che essa rilascia nell'universo nelle ultime fasi della propria vita.
\end{description}

Essendo la IMF una funzione di natura empirica, le sue caratteristiche non sono ancora ben conosciute. Questo significa che non è ben chiaro quale sia la forma di tale funzione e nemmeno se valga in maniera universale. Ovvero se sia dipendente dal tempo, dalle caratteristiche della nube o, addirittura, dalla posizione nello spazio. Un'ipotesi del suo andamento vuole che questa prenda la forma dell'equazione~\eqref{eq:IMF},
\begin{equation}
    \Psi(M) = k M^{-s}
    \label{eq:IMF}
\end{equation}
dove $s$ è un numero positivo e $k$ una costante di proporzionalità.

Si osserva che se il modello dovesse avere questa forma, allora la percentuale di stelle massicce è decisamente minore rispetto a quella di stelle con massa relativamente piccola. Tale predizione viene confermata dal fatto che la morte di una stella molto massiva produce un'implosione sul nucleo stesso generando una supernova, ma nel nostro cielo tali avvenimenti non sono così comuni. L'andamento della IMF è mostrato nella figura~\ref{fig:IMF}.

\begin{figure}
    \centering
    \includegraphics[width = 0.3\textwidth]{immagini/IMF.png}
    \caption{La figura mostra l'andamento della IMF in funzione della massa (normalizzata a quella solare) in scala logaritmica. La presenza di più linee è dovuta al fatto che non è presente una stima unica del parametro $s$ e di conseguenza sono indicate varie possibilità per questo parametro dove in rosso si ha quella più probabile.}\label{fig:IMF}
\end{figure}
\subsection{Protostelle e Teorema di Hayashi}

Data ora una nube di gas non uniforme, questa si inizierà a frammentare verso le varie buche di potenziale gravitazionale, in cui se la massa  soddisfa il criterio di Jeans~\eqref{eq:jeans}, allora si innescherà il collasso, producendo un oggetto che verrà definito con il termine \textit{Protostella}. Quando la contrazione innesca le reazioni termonucleari, generando un equilibrio idrostatico tra pressione gravitazionale e pressione di radiazione, la stella sarà completamente convettiva. Questo stato può essere rappresentato nel diagramma H-R come un punto della \textit{traccia di Hayashi}. Si tratta di una linea nel piano H-R che, data una stella con una determinata composizione chimica, mostra al variare della massa un punto del diagramma in cui questa è in equilibrio idrostatico e completamente convettiva. Tale curva quindi mostra una famiglia di stelle con la stessa composizione chimica, ma differente massa iniziale.

Il teorema di Hayashi afferma, inoltre, che fissata la composizione chimica e la massa di una stella, esiste una regione del piano H-R dove non è possibile realizzare modelli stellari in equilibrio idrostatico, la cui traccia di Hayashi funge da bordo. In figura~\ref{fig:hay} è mostrata la zona in questione colorata in rosso, mentre in verde si ha la regione che permette di avere modelli stabili, parzialmente convettivi. La validità di questo teorema è generale,  continua quindi a valere per ogni punto dell'evoluzione stellare, in modelli convettivi e non.

\begin{figure}
    \centering
    \includegraphics[width = 0.3 \textwidth]{immagini/hayashi-1.png}
    \caption{La figura mostra un diagramma H-R dove è stata evidenziata la traccia di Hayashi in rosso. A destra e a sinistra di questa sono state evidenziate due regioni, una in rosso ed una in verde. La prima mostra la zona in cui non è possibile generare stelle in equilibrio idrostatico, mentre la seconda mostra una regione di stabilità parzialmente convettiva.}\label{fig:hay}
\end{figure}

Particolarmente importante è la posizione di questa zona d'instabilità e della conseguente traccia di Hayashi. Infatti, queste si trovano nella parte fredda del diagramma H-R e, per quanto visto precedentemente, minore sarà la temperatura, maggiore sarà l'opacità della stella. Questo implica un gradiente radiativo maggiore di quello adiabatico ($\nabla_\textup{rad} > \nabla_\textup{ad}$) e quindi la presenza di convezione.

La fase di protostella dura fino a quando il nucleo raggiunge una temperatura $T \sim \SI{e7}{K}$, necessaria ad attivare le reazioni termonucleari. La durata di questa fase della vita della stella è compresa tra i $10^4$ e i $10^7$ anni, a seconda della massa iniziale della stella (maggiore è la massa iniziale, minore sarà la durata). Si tratta di una parte relativamente breve dell'evoluzione stellare.
\section{Main sequence}\label{sec:main-sequence}
\subsection{Zero-Age Main Sequence}
Dopo aver raggiunto le condizioni adatte per innescare le reazioni termonucleari, la stella comincia la parte più lunga della sua vita, detta Main Sequence (o MS), in cui brucerà idrogeno producendo elio ed energia. Sul piano H-R il punto iniziale della MS per una stella giace sulla Zero-Age Main Sequence (in breve ZAMS), una traccia che descrive la luminosità di stelle con la stessa composizione chimica, ma massa differente.

Non tutte le protostelle riescono ad accedere a questa parte dell'evoluzione stellare, infatti, esistono dei limiti al valore che la massa può avere. Data una protostella di massa superiore a $90 \si{solarmass}$, la pressione di radiazione è così elevata che sovrasta quella gravitazionale disperdendo le parti più esterne. Per stelle con metallicità molto bassa questo limite può raggiungere anche le $300 - 500 \si{\solarmass}$ così come accadeva per le prime stelle. Nel caso in cui, invece, la massa è inferiore alle $0.08 \si{\solarmass}$ allora le temperature nel core non sono abbastanza alte da permettere l'attivazione delle reazioni termonucleari.

\begin{figure}
    \centering
    \includegraphics[width = 0.5\textwidth]{immagini/ZAMS.png}
    \caption{La figura mostra la Zero-Age Main Sequence in rosso e la traccia di Hayaschi in azzurro.}\label{fig:ZAMS}
\end{figure}

\subsection{Produzione e trasmissione di energia}\label{sec:prod-tras-energia}

La natura dei meccanismi di fusione influisce anche sull'andamento della luminosità sulla ZAMS, in particolare per stelle molto massive, si ha un andamento che va come $L \propto M^{3.6}$, per quelle relativamente più leggere si ha che $L \propto M^4$.  Stelle più piccole hanno infatti temperature del core relativamente più basse rispetto alle stelle più pesanti privilegiando le reazioni a catena PP, che avvengono a temperature di circa $T_e = \SI{1.4e7}{K}$, piuttosto che la CNO, che avvengono nei corpi più pesanti, con temperature del nucleo di circa $T_e = \SI{1.8e7}{K}$. Il punto di passaggio tra i due tipi di reazione è molto stretto e si trova a luminosità di circa $L = 1.2 \si{\solarmass}$.

Il metodo di bruciare il proprio combustibile ha anche delle conseguenze sulla struttura interna della stella, in particolare implica la presenza o meno di convezione negli strati della struttura stellare. Distinguiamo tre possibilità a seconda della posizione che una stella assume sulla ZAMS\: se la massa è inferiore a $0.3 \si{\solarmass}$ allora la struttura sarà completamente convettiva, se la massa è compresa tra le $1.2 \si{\solarmass}$ e le $0.3 \si{\solarmass}$ allora il core avrà una natura radiativa e gli strati più esterni convettivi, mentre per stelle più pesanti di $1.2 \si{\solarmass}$ la situazione è invertita con il core convettivo e gli strati esterni radiativi. Il motivo di queste differenze risiede nel fatto che, nonostante le temperature variano poco, i meccanismi per generare energia sono profondamente diversi. Come visto nella sezione-\ref{sec:catena-pp}

\[
    \epsilon_{\mbox{\footnotesize{pp}}} \propto \rho X^2 T^5
\]
\[
    \epsilon_{\mbox{\footnotesize{CNO}}} \propto \rho X X_{\mbox{\footnotesize{CNO}}} T^{15}
\]

Per cui si ha che, per masse superiori a $1.2 \si{\solarmass}$, $\epsilon_{\mbox{\footnotesize{pp}}} \ll \epsilon_{\mbox{\footnotesize{CNO}}}$ se la temperatura aumenta, lo farà anche la produzione di energia, assieme alla luminosità. Da quanto visto precedentemente l'aumento di luminosità implica un flusso più intenso. Diventa per cui necessario dissipare questa energia più velocemente, attivando i processi convettivi. Se infatti il flusso aumenta allora aumenta anche il gradiente radiativo e nel momento in cui $\nabla_{rad} > \nabla_{ad}$ si attiverà la convezione nel nucleo.

Per stelle con massa inferiore alle $0.3 \si{\solarmass}$, invece, la totale natura convettiva deriva dal fatto che, per luminosità così basse, la ZAMS coincide con la traccia di Hayashi.

Nel caso intermedio, infine, la natura radiativa del core è dovuta al fatto che le temperature non sono abbastanza elevate ad innescare le \textit{catene CNO}, per cui la produzione di energia non è intensa come per le loro controparti più massive. La cosa peculiare per queste stelle, però, è la presenza di convezione nella zona esterna della sua struttura, sicuramente non dovuta alle elevate temperature. La spiegazione arriva dall'alta opacità di questi strati, infatti per

\[
    \kappa \propto T^{-3.5}
\]

al diminuire della temperatura l'opacità aumenta e di conseguenza lo fa anche il gradiente radiativo, fino a che $\nabla_{rad} > \nabla_{ad}$ condizione per cui si attiva la convezione in quegli strati.

\subsection{Main Sequence}\label{sec:main-sequence-sub}

Dalla ZAMS l'evoluzione stellare entra nella vera e propria main sequence, nella quale fonderà l'idrogeno in elio, modificando molto lentamente la propria composizione chimica. In questa parte della propria vita la stella si trova in uno stato di estrema stabilità, dovuta al fatto che la fusione d'idrogeno è il processo termonucleare più efficiente che esista, motivo per cui questa fase è anche quella più lunga. Ciò nonostante la effettiva durata è variabile della massa iniziale, in particolare il tempo che questa impiega nella MS è inversamente proporzionale alla sua massa iniziale, secondo la relazione 
\[
    t_{MS} \cong 10^{10}M^{-3}\mbox{ \footnotesize{(yr)}}
\]
Nella tabella~\ref{tab:MS} è mostrata la durata in anni della main sequence insieme alla massa iniziale, normalizzata a quella solare.
\begin{table}
    \centering
    \caption{Nella tabella sono riportati i valori della durata della main sequence in funzione dei valori della massa stellare, per un campione di stelle.}\label{tab:MS}
    \begin{tabular}{c|ccccccc}
        \toprule
        M [$\si{\solarmass}$] & 15.0 & 9.00 & 5.00 & 3.00 & 2.25 & 1.50 & 1.00\\
        \midrule
        $t_{MS}$ [yr] & $1 \times 10^7$ & $ 2.2 \times 10^7$ & $7 \times 10^7$ & $2\times 10^8$ & $5\times 10^8$ & $1.7 \times 10^9$ & $9\times 10^9$\\
        \bottomrule
    \end{tabular}
\end{table}

Si osserva che data una stella di circa $0.8 \si{\solarmass}$, la durata della sua MS ha lo stesso ordine di grandezza dell'età dell'universo ($t_{MS} \sim 13.9\times 10^9 \mbox{ yr}$). Questo vuol dire che stelle con massa inferiore ad essa e nate poco tempo dopo l'origine dell'universo, saranno ancora nel pieno della propria main sequence.
\section{Post Main Sequence}\label{sec:post-main-sequence}
Quando la stella finisce l'idrogeno all'interno del core di passa alla fase finale dell'evoluzione stellare, chiamata \textit{post-MS}. 

\subsection{Sub e Red Giant Branch}\label{sec:SGB-RGB}
In una prima parte della post main sequence il nucleo si contrae e la temperatura aumenta, assieme alla pressione, bruciando gli strati di idrogeno esterni al nucleo in una fase chiamata \textit{Herztsprung Gap} per stelle pesanti o \textit{SGB} per stelle più leggere. Nel frattempo, mentre il core, ora pieno di elio, continua a contrarsi aumentando di massa e aumentando la propria temperatura, l'envelope si espande raffreddandosi rendendola appunto una \textit{Gigante Rossa}. Nell figura~\ref{fig:evo} i punti 3, 4 e 5 del diagramma H-R rappresentano queste transizioni, osservando che, benché la temperatura decresca repentinamente, la luminosità della stella rimane pressoché costante. Dall'equazione~\ref{eq:stefan-boltzmann}, sappiamo infatti che:
\[
    L = 4\pi \sigma R^2 T^4
\]
Perciò la diminuzione di temperatura ed il relativo aumento di volume si compensano perfettamente così che la luminosità del corpo ci appare invariata.

Per stelle poco massive la diminuzione di temperatura implica l'avvicinamento alla traccia si Hayashi, per questo motivo la struttura stellare risponde aumentando il proprio raggio e conseguentemente la sua luminosità. Per stelle più pesanti, invece, la variazione di luminosità non è così netta. 
\subsection{Flash dell'Elio}\label{sec:flash-He e Horizontal Branch}

A questo punto nei copri più leggeri, si arriva ad un punto limite, detto limite di degenerazione elettronica, superato il quale il gas non può più essere considerato perfetto ed il nucleo entra in uno stato di degenerazione. L'aumento di temperatura non riesce a contrastare quello di densità, rendendo il gas sempre meno comprimibile e ritardando l'innesco dell'elio. Nei corpi più massivi la struttura è termoregolata, in cui l'aumento di temperatura è seguito dall'aumento di pressione e del volume della stella, in modo tale da bilanciare la pressione di radiazione. In queste strutture il collasso del core è notevolmente più facile, innescando le reazioni termonucleari per la fusione dell'elio non appena la temperatura raggiunge $T \sim \SI{e8}{K}$.

La differenza tra le due possibilità si osserva anche nel diagramma H-R della figura~\ref{fig:evo}, dove per le ultime l'attivazione della fusione ad elio (passaggio dal punto 5 al punto 6) è molto rapida e non comporta una variazione significativa della luminosità. Per le stelle più leggere, invece, l'accensione della fusione ad elio avviene in condizioni di semi-degenerazione e ritardata (di circa $\sim 10^6 \mbox{ yr}$) a causa dell'assenza di termoregolazione, motivo per cui l'aumento di temperatura non coincide con l'aumento di pressione. Si raggiunge, quindi, uno stato d'instabilità termonucleare fino a che $\rho_c$ non aumenta bruscamente, facendo implodere il core, rimuovendo la degenerazione e provocando un'esplosione, alla quale ci si riferisce con il termine \textit{Flash dell'Elio}. Questa comporta una rapida espansione del volume del nucleo e quindi della stella intera a causa di moti convettivi.
\begin{figure}
    \centering
    \includegraphics[width = 0.2\textwidth]{immagini/he_flash.png}
    \caption{La figura mostra la porzione del piano H-R in cui avviene il flash dell'elio e poi l'}\label{fig:flash-he}
\end{figure}

Dopo il flash dell'elio la produzione di energia diminuisce e la luminosità della stella diminuisce, fino a raggiungere la fase di \textit{Horizontal Branch} in cui questa rimane pressoché invariata. In questa parte della post-MS, raggiunta solo da stelle con massa inferiore alle $2.2 \si{\solarmass}$, non è presente degenerazione nel nucleo e si attiva la fusione di idrogeno negli strati più esterni della struttura, ancora pieni di idrogeno. Infatti, benché nel core l'idrogeno è stato consumato tutto durante la main sequence, negli strati più esterni, dove avveniva la trasmissione della radiazione, questo è ancora presente.

Nel piano H-R si osserva che l'esplosione dovuta all'accensione delle reazioni termonucleari ad elio avviene sempre quando la stella ha una massa intorno alle $0.5 \si{\solarmass}$ (nella figura~\ref{fig:evo} il punto 6 dell'evoluazione). Questo implica che la luminosità al momento del flesh dell'elio è uguale per tutte le stelle e che quindi tale fenomeno può essere utilizzato come indicatore di distanza dell'ammasso stellare, analizzando la magnitudine osservata e comparandola con quella assoluta.

Per stelle con una massa inferiore $M = 0.5\si{\solarmass}$ non è possibile attivare la reazione di fusione di elio nel core e per questo non sono in grado di attivare alcuna reazione termonucleare. Raggiungono quindi lo stato di \textit{Nana Bianca di elio (He Nana Bianca)}, una delle possibili morti stellari, in cui il nucleo rimane elettro degenere raffreddandosi lentamente, mentre gli strati più esterni rimangono liberi nella forma di nebulosa stellare.

Per quanto discusso nella sezione~\ref{sec:main-sequence}, una stella con massa inferiore a $M = 0.8 \si{\solarmass}$ passerà nella main sequence un periodo ti tempo dell'ordine di grandezza dell'età del nostro universo. Inoltre, si è appena detto che le nane bianche si formano solo per stelle di massa inferiore a $M = 0.5 \si{\solarmass}$, questo vorrebbe dire che non è ancora possibile osservare questi tipi di corpi. Ciò nonostante questi corpi sono già presenti nei nostri cieli, com'è possibile? Quello che accade è che stelle nella fase SGB, con nuclei di elio, perdono i propri strati più esterni, per vari motivi (risucchio da parte di un'altra stella in sistemi binari, esplosione di una Supernova, ...).
\subsection{Asymptotic Giant Branch}\label{sec:asymptotic-giant-branch}

Si è arrivati al punto che le stelle con massa superiore a $M = 0.5 \si{\solarmass}$ sono in grado di riaccendere le reazioni termonucleari necessarie a fondere l'elio. Quando, però, finisce anche questo all'interno del core si arriva ad un bivio dettato ancora una volta dalla iniziale. 

Corpi con $M < 8\si{\solarmass}$ tornano in uno stato di nucleo degenere, ora composto da carbonio ed ossigeno, per cui non più in grado di attivare reazioni termonucleari, raggiungendo quello che viene chiamato il \textit{Ramo Asintottico delle Giganti (Asymptotic Giant Branch, AGB)}. La struttura di questa fase dell'evoluzione stellare è caratterizzata dall'attivazione della fusione negli strati più esterni, seguita poi dall'accensione della fusione dell'elio in quelli più interni (come mostrato in figura~\ref{fig:AGB}) in condizione di semi-degenerazione.
\begin{figure}
    \centering
    \includegraphics[width = 0.3 \textwidth]{immagini/AGB.png}
    \caption{}\label{fig:AGB}
\end{figure}
Quando anche l'elio si spegne, la stella si contrae e si attivano nuovamente le reazioni di fusione dell'idrogeno, osservando periodiche espansioni e contrazioni in un fenomeno chiamato \textit{Pulsazioni Termiche delle AGB (Thermal Pulse AGB)} dovute alla fusione periodica d'idrogeno ed elio. Nel piano H-R queste trasformazioni seguono la traccia di Hayashi parallelamente. Durante questi processi l'attrazione gravitazionale, per unità di superficie, sugli strati più esterni diminuisce a causa delle espansioni, comportando una perdita di materia rilasciandola nell'universo ad un rate di $\dot{M} \sim 10^{-4} \si{\solarmass} \mbox{ yr}^{-1}$. Si tratta quasi una massa solare ogni $10000 \mbox{ yr}$, raggiungendo lo stato di \textit{Nebula planetaria} in cui non è più possibile attivare alcuna reazione termonucleare.

In questa ultima fase gli strati di elio ed idrogeno non sono abbastanza massivi da poter osservare alcuna attività termonucleare. La stella è definitivamente spenta. Il core si contrae aumentando la sua densità fino a raggiungere uno stato di degenerazione, mentre l'envelope si espande diventando un ammasso polveroso e di molecole fredde, rimanendo negli intorni di dove una volta era presente la stella. Quando la superficie di ciò che rimane raggiunge una temperatura di $T \sim \SI{e4}{K}$ parte dei gas e polveri presenti cominciano a ionizzarsi formando una nebulosa planetaria, che inizialmente non permette di osservare cosa resta del nucleo, a causa dell'elevata opacità. Queste strutture sono composte dagli elementi chimici generati dalle varie fusioni nucleari, in particolare elementi di tipo \textit{s}, carbonio, ossigeno, litio ed elementi pesanti dovuti al CNO.

%\chapterimage{}
\chapter{Galassie}
\input{capitoli/galassie/classificazione-di-Hubble.tex}
\section{Spettro di emissione delle galassie}\label{sec:spettro-di-emissione-delle-galassie}
\subsection{Caratteristiche generali}
Per osservare le galassie viene ovviamente raccolta la radiazione proveniente da loro, ma essendo corpi celesti molto lontani otteniamo delle immagini spesso poco soddisfacenti, perché non è possibile distinguere le singole stelle che compongono la galassia (in quel caso si dice che stiamo osservando una popolazione stellare risolta). Per le galassie più lontane, per le quali la magnitudine non è risolvibile, la magnitudine è la cosiddetta \emph{magnitudine integrata} di tutte le stelle al suo interno: quello che osserviamo quindi è la cosiddetta luce integrata, che viene dalla somma di sorgenti non risolte per la maggior parte delle galassie (solo le stelle delle galassie del Gruppo Locale sono risolte, per le altre no). Allo stesso modo della luce, possiamo osservare che anche il colore di una galassia lontana è dato dall'integrale dei colori delle singole stelle e allo stesso modo possiamo applicare il medesimo ragionamento anche per uno spettro di risoluzione $\Delta\lambda$ finita. Quindi quando osserviamo la luce proveniente da una galassia in realtà stiamo osservando la luce che viene da diverse fonti: fluso o magnitudine integrata, colore integrato e spettro integrato con risoluzione $\Delta\lambda$ finita. Le normali tecniche di risoluzione stellare NON sono applicabili a galassie lontane.

Data una galassia, la luce che osserviamo provenire da essa sarà sicuramente emessa dalle stelle che la compongono ma non solo, abbiamo diverse origini:
\begin{itemize}
    \item Stelle: emissione fotosferica (UV - infrarosso intermedio), vento stellare (linee di emissione, IR da involucri di polvere), fenomeni di accelerazione (raggo X binari).
    \item Gas: freddo (idrogeno non ionizzato, nubi molecolari), tiepido (T $\sim$ $10^4$ K, linee di emissione, idrogeno ionizzato), più tiepido (T $\sim$ $2/3 10^4$ K, resti di supernovae, compressioni rapide del gas), caldo ( T $\sim$ $10^7$ K, raggi X, linee di emissione continua).
    \item Polvere: emissione termica (luce stellare riprocessata dalle interazioni, shock termico), caratteristiche di emissione/assorbimento (PAHs, silicati e grafiti), scattering (UV - luce infrarossa).
    \item Nucleo Galattico Attivo (AGN): non è presemte in tutte le galassie, si tratta di un buco nero supermassiccio al centro di una galassia. Ci può essere emissione termica e non termica (spettro continuo e discreto). Nell'emissione termica lo spettro principale è quello dalle onde radio ai raggi gamma. La radiazione di un AGN può dominare l'intero spettro integrato.
\end{itemize}

In figura~\ref{fig:spettro+galassie} è riportato lo spettro di emissione (per lunghezza d'onda) delle galassie divise per morfologia: possiamo notare che determinate emissioni con una lunghezza d'onda precisa sono proprie della formazione stellare (osserviamo dei picchi in corrispondenza di quelle $\lambda$, come per OII e H$\alpha$) e infatti non appaiono in spettri di galassie in cui non si ha generazione stellare (passive, come quelle ellittiche o lenticolari). Lo spettro prende tutte le galassie; si potrebbe anche osservare la distribuzione spettrale di energia (SED) di una galassia in cui si verifica formazione stellare, che riporta solo un campionamento. Il confronto tra i due grafici (l'altro è sulle slide) è però in accordo con quanto riportato, si osservano quei picchi in entrambi i casi.

\begin{figure}
    \centering
    \includegraphics[width = 0.5 \textwidth]{immagini/spettro-galassie.png}
    \caption{Lo spettro della radiazione diviso per morfologia mostra come alcune lunghezze d'onda siano specifiche per galassie in cui si verifica formazione stellare}
    \label{fig:spettro+galassie}
\end{figure}

\subsection{Gas freddo, gas tiepido e polvere nelle galassie a spirale}
Come abbiamo detto una parte della radiazione proveniente dalle galassie proviene da gas freddo, in particolare da idrogeno neutro (H I); questo all'interno di una galassia è caratterizzato da una densità superficiale $ \sim 1 -\SI{10}{\solarmass.pc^{-2}}$, con un disco di dimensioni circa 5 volte maggiori dei dischi stellari (arriva fino a $\sim 100 Kpc$). L'idrogeno neutro si osserva in banda radio, dal momento che l'emissione è di radiazione a lunghezza d’onda pari a $21$ cm. Perché avviene l'emissione? Si tratta di una transizione dovuta alla struttura iperfine dell’atomo di idrogeno. Il livello fondamentale dell’atomo di idrogeno è diviso in due sottolivelli: nel livello di più bassa energia il neutrone e l’elettrone hanno spin opposto, mentre il livello in cui gli spin sono uguali ha energia lievemente superiore (differenza di $6*10^{-6}eV$, corrispondente alla lunghezza d’onda di $21$ cm). Questa transizione è impossibile sulla Terra, avendo probabilità di emissione molto bassa, che corrisponde ad un'emivita lunghissima ($\tau \sim 10^7$ anni). Tuttavia, osserviamo tale transizione nell’universo a causa dell’elevata abbondanza di idrogeno, infatti la linea di emissione a $21$ cm è ben visibile sia per la Via Lattea, sia per altre galassie più esterne. La distribuzione di idrogeno neutro, inoltre, non è uniforme nel disco galattico: si ha infatti una zona di maggior densità (uguale per idrogeno ionizzato) lungo i bracci della spirale della galassia. 

L'idrogeno può comparire anche a temperature più alte e in quel caso si tratta di idrogeno ionizzato (H II): ci sono alcune regioni che si osservano in luce ultravioletta perché la radiazione viene emessa da stelle OB (stelle più calde – più giovani, hanno un'emissione così energetica che ionizza gas circostante, emettendo in banda UV). Stelle così calde si trovano in una zona di formazione stellare; osservare gas ionizzato è quindi indice della presenza di formazione stellare. Dal momento che questo fenomeno si osserva principalmente lungo i bracci della galassie, se ne deduce che i bracci delle galassie a spirale sono luogo di formazione stellare. 

Per quanto riguarda invece il gas freddo e la polvere, questi sono osservabili in luce infrarossa, perché per assorbimento (reddening) della luce, questa viene riemessa in banda infrarossa. 

\section{Struttura delle galassie a spirale}\label{sec:struttura-delle-galassie-a-spirale}
Analizziamo ora la struttura delle galassie a spirale: i dischi di spirale sono sempre in rotazione attorno al centro galattico. In quasi tutti i casi i bracci ruotano con bracci “trailing”, solo in pochissime galassie si hanno “leading arms” (fare riferimento alla figura~\ref{fig:trailing-leading arms}): la forma a spirale quindi data dal materiale che viene trascinato dai bracci ma che si muove con velocità diverse. Infatti NON si ha rotazione rigida: la velocità angolare NON è la stessa in tutti i punti del disco. Si ha invece rotazione differenziale, quindi le stelle più vicine al centro ruotano con velocità più elevate; in genere la velocità tangenziale è la stessa indipendentemente dalla distanza dal centro, quindi le regioni più esterne hanno velocità angolare $\omega = V/r$ minore rispetto a quelle più vicine al centro. Questo significa che le stelle delle regioni più lontane dal centro impiegheranno decisamente più tempo per compiere un'orbita rispetto a quelle più vicine al centro.

\begin{figure}
    \centering
    \includegraphics{immagini/trailing-leading-arms.png}
    \caption[width =0.5\textwidth]{Differenza tra "trailing arms" e "leading arms": nel primo caso i bracci sono trascinati dalla rotazione, mentre nel secondo guidano e anticipano la rotazione.}
    \label{fig:trailing-leading arms}
\end{figure}

In seguito a questa osservazione nasce il cosiddetto "dilemma del winding", mostrato in figura~\ref{fig:winding-dilemma}: se i bracci fossero fatti da materia, allora dovrebbero avvolgersi attorno al centro galattico in tempi scala più corti dell’età della galassia stessa (dopo poche decine di rotazioni). Questo però comporterebbe la perdita dei bracci, che risulterebbero tutti avvolti su se stessi, ma non è quello che noi osserviamo; infatti siamo in grado di osservare bracci a spirale anche per galassie molto antiche. La risposta quindi è che questo “arrotolamento” non c’è anche se è presente rotazione differenziale. Ma perché non c'è?

\begin{figure}
    \centering
    \includegraphics{immagini/winding-dilemma.png}
    \caption[width =\textwidth]{Dilemma del winding: se i bracci delle galassie fossero di materia dovrebbero pian piano arrotolarsi sul centro della galassia stessa.}
    \label{fig:winding-dilemma}
\end{figure}

La soluzione a questo quesito è basata sulla teoria di onda di densità di Lin: il fatto è che i bracci di spirale non sono materiali (quindi non sono composti da un insieme di gas e stelle che si muovono sempre insieme), ma sono generati da un’onda di densità, un’onda quasi-statica e molto duratura nel tempo. La galassia può essere pensata come un disco di materiale che ruota, attraversato da un’onda in compressione che ruota a sua volta ma a una velocità inferiore del disco materiale, creando così dei picchi di densità che sono quelli che chiamiamo bracci della spirale. I bracci di spirale sono perciò i luoghi di densità più elevata (e quindi la buca di potenziale è più profonda), ma NON sono composti sempre dalla stesse stelle! Si può fare un'analogia al caso delle macchine bloccate nel traffico: le zone dove c'è traffico sono quelle con maggiore densità di macchine e dove queste saranno costrette a trascorrere più tempo. Una volta superata la zona però la macchina se ne va e al suo posto ce ne saranno altre. Allo stesso modo le stelle e il gas, durante la loro orbita attraversano queste zone di più alta densità restandoci più a lungo ma pian piano escono e vengono sostituite da altre. Di conseguenza lo spazio fra i diversi bracci delle galassie non è vuoto e i bracci sono solo i luoghi dove le stelle della galassia trascorrono la maggior parte del tempo della loro orbita.

Questo modello spiega anche perché si ha formazione stellare lungo i bracci: quando le stelle e i gas entrano nell’onda si crea una regione di compressione che, per il criterio di instabilità di Jeans TEMP (che dice che le stelle si formano più facilmente in regioni di alta densità), fa sì che la formazione stellare sia favorita.

Un'altra cosa che a questo punto è naturale chiedersi è da cosa siano generate queste onde di densità? La risposta non è ancora chiara, ma si ipotizza che ci siano mancanze di simmetria iniziali nel disco, ad esempio legate al processo di formazione delle galassie, che prevedere l’interazione con altre galassie (quindi potrebbe esserci una galassia molto grande che interagisce con una piccolina a formare la galassia finale che in quella zona presenterà un zona a più alta densità).

\section{Brillanza superficiale}\label{sec:brillanza-superficiale}
Un altro parametro importante della galassie è la loro brillanza superficiale (surface brightness, SB): questa altro non è che la distribuzione bidimensionale della materia luminosa sul piano del cielo, quindi la magnitudine per unità di area. Come facciamo a calcolare i profili di brillanza superficiale? Per farlo si costruiscono le isofote (ossia anelli di profilo caratterizzata dalla stessa brillantezza), si calcola la brillanza in ogni anello e infine si plottano i profili di brillanza superficiale (magnitudine - raggio), rappresentati in figura~\ref{fig:profili-brillanza}.

\begin{figure}
    \centering
    \includegraphics[width = 0.5 \textwidth]{immagini/profili-di-brillanza.png}
    \caption{Profili di brillanza, costruiti plottando la magnitudine a raggi diversi (quindi studiando la magnitudine delle isofite).}
    \label{fig:profili-brillanza}
\end{figure}

Si è osservato che quasi tutti i profili di brillanza delle galassie (sia ellittiche sia a spirale) seguono il profilo di Sersic, descritto dalla seguente equazione:

\begin{equation*}
    I(r) = I_0 \; e^{-b_n \big(\frac{r}{r_e}\big)^\frac{1}{n}}
\end{equation*}

in cui $I(r)$ è la brillantezza superficiale, $I_0$ è la brillantezza centrale, $r_e$ è il raggio effettivo (ossia il raggio contenente il 50\% della luce proiettata), $b_n$ una costante e $n$ è l'indice di concentrazione (compreso fra $1$ e $10$).

Le galassie ellittiche sono caratterizzate da un cosiddetto "profilo Vaucoulerurs", corrispondente a n=4, mentre le galassie a spirale hanno un profilo di tipo esponenziale, corrispondente a n=1. 
\section{Dinamica interna}\label{sec:dinamica-interna}
Arrivati a questo punto è interessante studiare la dinamica interna di queste galassie, ossia come si muovono le stelle al loro interno. Si nota infatti che le galassie a spirale e quelle ellittiche hanno una dinamica interna molto differente:

\begin{itemize}
    \item \emph{Galassie a spirale:} sono "rotation supported", ossia le stelle si muovono su orbite circolari (o quasi) attorno al centro (in modo molto ordinato). Le misure di cinematica interna si esplicano nella realizzazione di una curva di rotazione, che descrive come la rotazione delle stelle varia con la distanza dal centro galattico.
    \item \emph{Galassie ellittiche:} sono “pressure supported”, ossia le stelle si muovono di moti randomici attorno al centro (moto non ordinato). Una misura di cinematica si traduce nella realizzazione di un profilo di dispersione di velocità. Infatti noi dobbiamo pensare che il sistema si muove con un moto di insieme (il cosidetto "bulk motion") verso di me o lontano da me, ma ciascuna stella ha proprio moto. La dispersione di velocità misura quanto la velocità delle singole stelle sono disperse attorno alla velocità media del bulk motion, anche detta velocità sistemica.
\end{itemize} 

\subsection{Curve di rotazione delle galassie a spirale}

\begin{figure}
    \centering
    \includegraphics[width = 0.4 \textwidth]{immagini/effetto-doppler.png}
    \caption{La riga proveniente dal punto B (centro della galassia) ci dà la velocità sistemica, mentre le linee che provengono dai punti A e C ci dicono come si stanno muovendo i bracci della galassia: se si muovono verso di noi avranno emissione più spostata sul blu, mentre se si allontanano sul rosso.}
    \label{fig:effetto-doppler-galassie}
\end{figure}

Per analizzare la dinamica interna delle galassie a spirale ci serviamo di due strumenti principali:
\begin{itemize}
    \item Split spettroscopy: mettiamo una fenditura quando osserviamo una galassia, in modo da ricevere solo la luce caduta sulla fenditura: al centro della fenditura ci sarà la luce che proviene dal centro della galassia e la luce ai bordi della fenditura verrà invece dal bordo della galassia. Per costruire uno spettro a questo punto misuriamo una determinata riga spettrale, prima al centro e poi agli estremi della fenditura: quello che si ottiene è che la riga è centrata a lunghezze d’onda diverse per effetto Doppler. La riga della parte di galassia che si avvicina a noi è blue shifted e quella della parte di galassia che si allontana da noi è red shifted e attraverso una misura della differenza delle lunghezze d’onda delle parti estremali trovo la velocità di rotazione della galassia. La riga misurata al centro mi dice invece la velocità sistemica della galassia (tramite un confronto della lunghezza d’onda con quella di laboratorio), come viene riassunto in figura~\ref{fig:effetto-doppler-galassie}. La velocità di rotazione si misura utilizzando forti righe di emissione in banda otica e dalla riga 21cm in banda radio (viene da idrogeno neutro/gas freddo). 

    \item Integral Field spettroscopy: combina imaging (fotometria) e spettroscopia. Dopo aver ottenuto un’immagine della galassia, da ogni pixel del CCD otteniamo uno spettro. Abbiamo quindi un piano su cui si ha la luce che viene raccolta e una terza dimensione perché per ogni pixel c'è uno spettro. Questo ci permette di avere informazione sulla dinamica non solo sulla fenditura, come prima, ma su tutta l’area della galassia, quindi riusciamo ad ottenere delle mappe bidimensionali di rotazione. Mentre con la "split spettroscopy" abbiamo un valore di velocità per ogni distanza fissata dal centro, in questo modo otteniamo invece una mappa bidimensionale, quindi in ogni punto dell’area possiamo vedere quanto velocemente ruotano le stelle.
\end{itemize}

Le stelle in una galassia a spirale sono punti che ruotano attorno al centro, quindi  ci aspettiamo un potenziale kepleriano; la galassia però non è puntiforme. Dobbiamo quindi pensarla come a una sfera di gas e una stella che si muove in quella sfera. Al centro della galassia si ha rotazione nulla (infatti ci troviamo sull'asse di rotazione), poi ci sarà un andamento lineare della velocità fino a un certo punto, in cui possiamo ritornare al caso kepleriano, con un andamento riassunto in figura~\ref{fig:potenziale-galassie-a-spirale}.

\begin{figure}[htb]
    \centering
    \includegraphics[width = 0.4 \textwidth]{immagini/potenziale-galassie-a-spirale.png}
    \caption{Andamento del potenziale nelle galassie a spirale.}
    \label{fig:potenziale-galassie-a-spirale}
\end{figure}

\begin{figure}[htb]
    \centering
    \includegraphics[width= 0.5\textwidth]{immagini/confronto-curve-di-rotazione.png}
    \caption{Confronto fra la curva di rotazione aspettata che segue un decadimento kepleriano e quella invece osservata, che non presenta questo decadimento ma piuttosto un plateu: questo fenomeno è spiegabile grazie alla materia oscura.}
    \label{fig:confronto-curve-di-rotazione}
\end{figure}

Dal punto 3 di figura~\ref{fig:potenziale-galassie-a-spirale}, siamo fuori dalla zona in cui c’è tanta materia, la densità è così bassa che posso approssimare la situazione come se fosse una massa puntiforme e quindi poi ci aspettiamo un andamento kepleriano. Dal profilo di brillanza ci aspetteremo quindi un andamento dapprima con aumento lineare della velocità (fino a quando mi trovo abbastanza vicino alla galassia e più forza d'interazione c'è) e poi decrescita kepleriana. Guardando ai dati però non si osserva il “calo” kepleriano che ci aspetteremmo da come osserviamo la materia visibile. C'è una spiegazione a questo: l'unico modo per avere un andamento di questo tipo è infatti avere la presenza di una massa invisibile che continua ad essere presente anche quando ci aspettiamo che sia finita la materia della galassia. Questa materia, che contribuisce alla buca di potenziale ma non è rilevabile attraverso emissione elettromagnetica, viene detta \emph{materia oscura}. In figura~\ref{fig:confronto-curve-di-rotazione} possiamo vedere le curve di rotazione relative alla nostra galassia, Via Lattea, con indicazione del Sole come sua stella. Come emerge dal confronto c'è una profonda discrepanza fra il comportamento aspettato e quello invece osservato, e questo è spiegabile solo ammettendo la presenza di questo alone di materia oscura che circonda anche la nostra galassia (addirittura si stima che la nostra galassia sia fatta al 75\% di materia oscura).

\subsection{Dispersione delle velocità per galassie ellittiche}

\begin{figure}[htb]
    \centering
    \includegraphics[width =0.7\textwidth]{immagini/spettroscopia-fessura-campo-integrale.png}
    \caption{Confronto fra i due tipi di spettroscopia: quella a fessura mi da una riga spettrale radiale, quella a campo integrale mi dà invece una mappa bidimensionale.}
    \label{fig:spettroscopia-fessura-campo-integrale}
\end{figure}

\begin{figure}[htb]
    \centering
    \includegraphics[width = 0.5\textwidth]{immagini/schiacciamento-galassie.png}
    \caption{La curva in figura divide le galassie che hanno velocità di rotazione sufficiente per causare in questo modo lo schiacciamento e quelle che invece sono schiacciate per anisotropia della dispersione di velocità. Possiamo vedere che la maggior parte delle galassie ellittiche (pallini) è schiacciata proprio per questo secondo motivo.}
    \label{fig:schiacciamento-galassie}
\end{figure}

Per quanto riguarda le galassie ellittiche, queste sono caratterizzate da una velocità media di movimento della galassia intera e poi in realtà ogni stella si muove random. Questi moti randomici hanno velocità distribuite con una distribuzione simile ad una gaussiana intorno alla velocità sistemica; si dice che le galassie ellittiche sono "pressure supported" perché più grande è la buca di potenziale e maggiore è la dispersione di velocità (quindi galassie più massicce hanno dispersione di velocità maggiore). Ci immaginiamo un allargamento della riga spettrale: ciascuna stella si muove con una velocità maggiore o minore rispetto alla velocità sistemica, quindi, avrà uno shift verso il blu o verso il rosso rispetto alla sistemica. Osserviamo una sovrapposizione di righe spettrali, alcune spostate verso il rosso e altre verso il blu; se la dispersione fosse zero avrei una distribuzione più stretta (gaussiana più stretta), quanto maggiore è la dispersione di velocità, tanto più larga è la distribuzione. A questo punto si misura l’allargamento al centro e a varie distanze dal centro e si ottiene un profilo di dispersione di velocità (sempre usando la spettroscopia a fessura). In modo alternativo si può procedere con la spettroscopia a campo integrale, in modo da ricavare una mappa bidimensionale di dispersione di velocità e non solo un profilo radiale. Un confronto tra i risultati ottenuti dai due tipi di spettroscopia è presentato in figura~\ref{fig:spettroscopia-fessura-campo-integrale}.

Un'altra caratteristica interessante delle galassie ellittiche è il loro schiacciamento: queste galassie infatti non hanno una rotazione significativa, sorge quindi la domanda di come facciano ad essere schiacciate. Questo fenomeno è in gran parte dovuta alla anisotropia orbitale (anisotropia della dispersione di velocità): questo significa che la dispersione di velocità non ha lo stesso valore in tutte le direzioni dello spazio. Se si ha una dispersione di velocità maggiore su un asse, avremo la galassia più allungata lungo tale direzione, ovvero la galassia risulta schiacciata nella direzione in cui si ha minore dispersione di velocità. La maggior parte delle ellittiche ha grande ellitticità pur con bassa velocità media di rotazione, come si può vedere in figura~\ref{fig:schiacciamento-galassie}.

\subsection{Materia oscura nelle galassie ellittiche} \label{sec:materia-oscura-ellittiche}
A questo punto possiamo chiederci, se è presente un'evidenza di materia oscura anche nelle galassie ellittiche come c’è nelle galassie a spirale? La risposta è che per le galassie ellittiche è molto complicato osservare prove sperimentali per una serie di motivi:
\begin{itemize}
    \item Le ellittiche non sono rotanti e non posso costruire una curva di rotazione come nelle spirali. Ricordiamo che nel caso delle spirali sono proprio queste curve che mettono in evidenza la presenza della materia oscura.
    \item Non conosciamo la struttura 3D (invece delle galassie a spirale sappiamo che sono "schiacciate", ossia a meno del bulge centrale le stelle nei bracci si trovano tutte sullo stesso piano) e l’unica informazione a disposizione è il profilo di brillanza superficiale.
    \item Non sappiamo la forma 3D della distribuzione anisotropa delle velocità, potrebbero esserci delle differenze significative a seconda della "profondità".
    \item Mentre nelle galassie a spirale il gas freddo (H neutro) ha grande estensione radiale e ci permette tramite l'effetto Doppler di misurare la curva di rotazione a grandi distante, nelle ellittiche non ci sono “tracciatori” di massa (o buca di potenziale) a grande distanza dal centro.
\end{itemize} 

\begin{figure}
    \centering
    \includegraphics[width=0.6\textwidth]{immagini/gravitational-lensing.png}
    \caption{Il fenomeno del lensing gravitazionale può mostrarsi in due modi: possiamo osservare immagini multiple di una sorgente di background (sopra) o può comparire un anello attorno alla massa centrale (sotto).}
    \label{fig:gravitational-lensing}
\end{figure}

È molto difficile quindi ottenere dati sperimentali: un modo per stimare la massa è quello di usare il gas caldo diffuso, che emette in raggi X, presente in grande quantità nelle galassie ellittiche. Attraverso un’assunzione di equilibrio idrostatico, che è ragionevole per le ellittiche, possiamo stimare la massa della galassia (vedi capitolo ammassi di galassie).

Il metodo più utilizzato per stimare la massa di galassie ellittiche però è sfruttare il segnale che si ottiene dallo strong gravitational lensing. Si tratta di un fenomeno per cui la massa distorce lo spazio-tempo, di conseguenza i fotoni, deviandone la traiettoria. A causa di questo, osservando un oggetto dalla Terra, compaiono immagini multiple della sorgente di background o, addirittura, può comparire un anello attorno alla massa centrale, detto anello di Einstein, anch'esso dovuto alla distorsione della luce da parte della galassia (come si può vedere in figura~\ref{fig:gravitational-lensing}). Attraverso questo effetto è possibile stimare la massa dell’oggetto responsabile dell’effetto; tipicamente si trova che la massa della lente (ossia della galassia ellittica interposta che fa da lente) è più grande della massa che posso stimare dalla luce dell’ellittica (usando il profilo di luce posso stimare la sua massa). In figura~\ref{fig:materia-oscura-ellittiche} si può vedere come, soprattutto a grandi raggi, sia necessaria quindi una correzione lineare all'andamento della densità di massa: per questo motivo si può interpretare la differenza fra la massa calcolata dal profilo di luce e quella calcolata dal lensing gravitazionale come la necessità di tenere conto di una componente addizionale di massa, composto da materia non barionica (oscura). Questo "alone", sembra quindi circondare anche le galassie ellittiche.

\begin{figure}
    \centering
    \includegraphics[width = 0.8 \textwidth]{immagini/materia-oscura-ellittiche.png}
    \caption{Densità di massa in funzione del raggio: si può vedere come per raggi grandi sia necessaria una correzione alla stima della massa, imputabile alla presenza di un alone di materia oscura anche attorno alle galassie ellittiche.}
    \label{fig:materia-oscura-ellittiche}
\end{figure}

%\chapterimage{}
\chapter{Ammassi di galassie}

%\chapterimage{}
\chapter{Introduzione alla cosmologia}

%\chapter{Da fare}
%\section{Cose da fare}
\paragraph{Redshift}\label{sec:redshift}
Va aggiunto il paragrafo sul redshift e va corretta la REFERENZA A TALE PARAGRAFO ALL'INIZIO, VEDI GLI ERRORI

\appendix
\chapter{Appendici}
%\section{Dati numerici}\label{app:dati}
Nella tabella~\ref{apptab:sole} sono presenti le principali grandezze caratterizzanti il nostro Sole. Nella tabella ~\ref{apptab:vialattea} sono presenti le grandezze principali che caratterizzano la Via Lattea.

\begin{table}
\caption{Parametri principali del Sole}
\label{apptab:sole}
\centering
\begin{tabular}{ll}
\toprule
Grandezza & Ordine di grandezza \\
\midrule
Raggio          & $\sim \SI{6.7e10}{cm}$ \\
Massa           & $\sim \SI{2e33}{g}$    \\
Età             & $\sim \SI{1.4e17}{s}$  \\
Distanza da noi & $\sim \SI{1.5e13}{cm}$  \\
\bottomrule
\end{tabular}
\end{table}

\begin{table}
\caption{Parametri principali della Via Lattea}
\label{apptab:vialattea}
\centering
\begin{tabular}{ll}
\toprule
Grandezza & Ordine di grandezza \\
\midrule
Raggio               & $\sim \SI{8e22}{cm}$ \\
Massa                & $\sim \SI{e45}{g}$    \\
Età                  & $\sim \SI{3.8e17}{s}$  \\
Distanza centro-Sole & $\sim \SI{2.5e22}{cm}$  \\
\bottomrule
\end{tabular}
\end{table}
%\section{Unità di misura}\label{app:unità}
Nella tabella~\ref{apptab:unità} si trovano le principali unità di misura dell'astrofisica con utili conversioni.

\begin{table}
\caption{Unità di misura principali}
\label{apptab:unità}
\centering
\begin{tabular}{ll}
\toprule
Unità & Conversione \\
\midrule
DA SCRIVERE
\end{tabular}
\end{table}
%\section{Equazione del trasporto radiativo}\label{app:trasporto-radiativo}
Integriamo l'equazione~\eqref{eq:trasporto-radiativo3}, che per comodità riportiamo sotto:
\[
\frac{\ud I_\nu}{\ud \tau_\nu} = -I_\nu + S_\nu
\]
Separo le variabili e riscrivo l'equazione in una forma più comoda
\[
    \ud I_\nu = -I_\nu \ud \tau_\nu  + S_\nu \ud \tau_\nu
\]
\[
    \ud I_\nu + I_\nu \ud \tau_\nu  = S_\nu \ud \tau_\nu
\]
\[
    e^{\tau_\nu} (\ud I_nu + I_\nu \ud \tau_\nu)  = e^{\tau_\nu} (S_\nu \ud \tau_\nu)
\]
\[
    d(e^{\tau_\nu} I_\nu) = e^{\tau_\nu} \ud I_\nu + I_\nu e^{\tau_\nu} \ud \tau_\nu
\]
\[
    d(e^{\tau_\nu} I_\nu) = e^{\tau_\nu} S_\nu \ud \tau_\nu
\]
Ricordando che la profondità ottica è un integrale tra $0$ e $S$ e al centro è nullo, come limiti inferiori dell'integrazione scelgo $\tau_{\nu_0} = 0$ e $I_{\nu_0}$, corrispondente all'intensità iniziale, prima dell'interazione con la materia. Suppongo, inoltre, che la funzione sorgente $S_\nu$ non vari con $\tau$, ovvero non vari con la distanza.
\[
    e^{\tau_\nu} I_\nu - I_{\nu_0} = \int_0^{\tau_\nu} e^{\tau_\nu'} S_\nu \ud \tau_\nu' = S_\nu (e^{\tau_\nu} - 1)
\]
Moltiplico ambo i membri per $e^{-\tau_\nu}$
\[
    I_\nu (\tau_\nu) - I_{\nu_0} e^{-\tau_\nu} = S_\nu (1-e^{-\tau_\nu})
\]
da cui si ottiene la soluzione~\eqref{eq:soluzione-trasporto-radiativo}
\end{document}
