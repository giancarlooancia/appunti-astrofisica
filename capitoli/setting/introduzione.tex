\section{Introduzione}\label{sec:introduzione}
Come per ogni ambito della fisica, anche per l'\emph{astrofisica} è necessario avere in mente gli ordini di grandezza con cui si lavora e gli oggetti di studio. Nella tab.~\ref{tab:ordini-grandezza-sole-vialattea} sono riportate alcune grandezze riferite al nostro Sole e alla Via Lattea. A causa delle scale spaziali e temporali enormi rispetto a quella umana, risulta ovvio che non è possibile riprodurre le strutture cosmiche in laboratorio, andare su di esse per prendere delle misure dirette, oppure seguirne la loro evoluzione istante per istante. Dunque, è necessario uno studio \emph{indiretto}, che avvenga analizzando la radiazione che riceviamo, che proviene necessariamente dal passato. In prima approssimazione possiamo ritenere che sia stata emessa $\Delta t = D / c$ tempo fa, con $D$ distanza della sorgente e $c$ velocità della luce nel vuoto. Un altro limite delle osservazioni astrofisiche consiste nel non avere accesso alla terza dimensione, ovvero alla profondità rispetto alla linea di vista. Inoltre, è necessario un \emph{approccio statistico} per comprendere i processi evoluti, cercando, in prima istanza, di raggruppare oggetti simili, i quali presumibilmente abbiamo la stessa età. Ovviamente l'astrofisica utilizza le \emph{leggi della fisica} per interpretare i dati osservativi e avanzare predizioni teoriche, tramite \emph{modelli} e \emph{simulazioni}.

\begin{table}
    \caption{Ordini di grandezza riferiti al Sole e alla Via Lattea.}
    \label{tab:ordini-grandezza-sole-vialattea}
    \centering
    \begin{tabular}{lll}
    \toprule
    Grandezza & Sole & Via Lattea  \\
    \midrule
    Raggio (\si{cm}) & $\sim \SI{6.7e10}{}$ & $\sim \SI{8e22}{}$ \\
    Massa (\si{g}) & $\sim \SI{2e33}{}$    & $\sim \SI{e45}{}$    \\
    Età (\si{s}) & $\sim \SI{1.4e17}{}$  & $\sim \SI{3.8e17}{}$  \\
    Distanza da noi (\si{cm}) & $\sim \SI{1.5e13}{}$  & $\sim \SI{2.5e22}{}$  \\
    \bottomrule
    \end{tabular}
\end{table}