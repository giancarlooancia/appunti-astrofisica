\section{Introduzione}\label{sec:introduzione}
Ci si può fare un'idea degli ordini di grandezza con cui ha a che fare l'astrofisica osservando la tabella~\ref{tab:parametri-principali-sole} e~\ref{tab:parametri-principali-via-lattea}. Quindi risulta ovvio che non è possibile riprodurre le strutture cosmiche in laboratorio, non è possibile andare sulle strutture cosmiche a fare misurazioni e non è neanche possibile seguirne la loro evoluzione temporale. Dunque, la \emph{scienza astrofisica} studia le strutture cosmiche in maniera \emph{indiretta}, in particolare misurando e analizzando la radiazione che raccogliamo con i telescopi e altri strumenti, radiazione proveniente dal \emph{passato} a causa delle grandi distanze cosmiche e del fatto che la velocità della luce è finita. In particolare, possiamo stimare che la radiazione che misuriamo è stata emessa $\Delta t = D / c$ tempo fa, con $D$ distanza della sorgente e $c$ velocità della luce nel vuoto. Inoltre, le misure astrofisiche \emph{non} hanno accesso alla terza dimensione, ovvero alla profondità lungo la linea di vista, ma sono limitate a \emph{proiezioni sul piano del cielo}. È necessario, infine, un \emph{approccio statistico} per comprendere i processi evolutivi degli oggetti cosmici, a causa del loro tempo di vita elevatissimo rispetto all'intervallo di tempo entro il quale riusciamo a prendere misure. Studiamo, allora, gruppi di oggetti simili, presumibilmente della stessa età, per cercare di capire come questi evolvano nel corso del tempo. Ovviamente l'astrofisica utilizza le \emph{leggi della fisica} per interpretare i dati osservativi e avanzare predizioni teoriche, tramite \emph{modelli} e \emph{simulazioni}. 

\begin{table}
\caption{Parametri principali del Sole}
\label{tab:parametri-principali-sole}
\centering
\begin{tabular}{ll}
\toprule
Grandezza & Ordine di grandezza \\
\midrule
Raggio          & $\sim \SI{6.7e10}{cm}$ \\
Massa           & $\sim \SI{2e33}{g}$    \\
Età             & $\sim \SI{1.4e17}{s}$  \\
Distanza da noi & $\sim \SI{1.5e13}{cm}$  \\
\bottomrule
\end{tabular}
\end{table}

\begin{table}
\caption{Parametri principali della Via Lattea}
\label{tab:parametri-principali-via-lattea}
\centering
\begin{tabular}{ll}
\toprule
Grandezza & Ordine di grandezza \\
\midrule
Raggio               & $\sim \SI{8e22}{cm}$ \\
Massa                & $\sim \SI{e45}{g}$    \\
Età                  & $\sim \SI{3.8e17}{s}$  \\
Distanza centro-Sole & $\sim \SI{2.5e22}{cm}$  \\
\bottomrule
\end{tabular}
\end{table}