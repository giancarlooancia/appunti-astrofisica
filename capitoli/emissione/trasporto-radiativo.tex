\section{Equazione del trasporto radiativo}\label{sec:intro-trasporto-radiativo}
Come anticipato nel paragrafo~\ref{sec:astrofisica-osservativa}, per poter interpretare in maniera corretta le misure di radiazione, è necessario studiare i meccanismi d'interazione tra la radiazione e la materia. Le cause principali di tale interazione sono l'\emph{assorbimento}, lo \emph{scattering}\footnote{Lo scattering è la deviazione rispetto alla direzione di propagazione originale.} e l'\emph{emissione}. Questi fenomeni variano a seconda della \emph{frequenza} della radiazione e questo è il motivo per cui osservare a frequenze diverse porta a informazioni diverse.

L'\emph{equazione del trasporto radiativo} descrive il trasporto di radiazione da parte dei fotoni. Non è l'unico meccanismo di trasporto ma è il più comune nell'universo. In particolare, l'equazione dice come varia l'intensità della radiazione a causa dell'interazione tra la radiazione e la materia. Prima di ricavare l'equazione è necessario introdurre alcune grandezze indispensabili per esprimere le energie.

\subsection{Intensità specifica o brillanza}\label{sec:intensità}
Dato un campo di radiazione, l'intensità specifica $I_\nu$ (fig.~\ref{fig:intensità-specifica}) lungo la direzione $\hat{k}$ in un punto $P$ del campo è la quantità di energia che, nell'intervallo di tempo $\ud t$ e nell'intervallo di frequenze $\ud \nu$, attraversa una superficie infinitesima perpendicolare alla direzione $\hat{k}$ ($\ud A_\perp$), entro un angolo solido elementare $\ud \Omega$:
\begin{equation}\label{eq:intensità}
    I_\nu = \frac{\ud E}{\ud t \ud \nu \ud A_\perp \ud \Omega} = \frac{\ud E}{\ud t \ud \nu \ud A \cos{\theta} \ud \Omega} \quad \bigl[\si{erg.s^{-1}.cm^{-2}.Hz^{-1}.sr^{-1}}\bigr]
\end{equation}
dove $\theta$ è l'angolo tra la normale a $\ud A$ e la direzione $\hat{k}$, ovvero si può scrivere $\ud A_\perp = \ud A \cos{\theta}$.

\begin{figure}
\centering
\includegraphics[width=0.25\textwidth]{immagini/intensita-specifica.png}
\caption{Definizione di intensità specifica. $\ud A$ è un'area infinitesima sulla superficie della stella, $P$ è il punto della superficie della stella, $\hat{k}$ è il versore rispetto al quale definisco l'intensità e $\ud A_\perp = \ud A \cos{\theta}$. La radiazione è portata da un fascio di fotoni, quindi devo considerare un tronco di angolo solido $\ud \Omega$, in cui $\ud A$ rappresenta la sezione del fascio.}
\label{fig:intensità-specifica}
\end{figure}

Si tratta di una proprietà \emph{intrinseca} del campo di radiazione, ovvero della sorgente, ed è definita per unità di frequenza o equivalentemente di lunghezza d'onda.

Si ricordi, inoltre, che il differenziale di angolo solido si scrive:
\begin{equation}\label{eq:angolo-solido}
    \ud \Omega = \sin{\theta} \ud \theta \ud \phi
\end{equation}

La dipendenza dell'intensità dalla frequenza (equiv. dalla lunghezza d'onda) si può rappresentare attraverso una \emph{distribuzione spettrale di energia (SED)}, come sarà visto successivamente.

\subsection{Luminosità}\label{sec:luminosità}
La \emph{luminosità bolometrica} ($L$) di una sorgente è definita come la quantità di energia totale emessa dalla sorgente nell'unità di tempo. Si misura in $\si{erg.s^{-1}}$ oppure in \emph{luminosità solari}, dove $\si{\solarluminosity} = \SI{3.8e33}{erg.s^{-1}}$. Si tratta di una quantità \emph{intrinseca} della sorgente.

\begin{equation}\label{eq:def-luminosità-bolometrica}
    L = \frac{\ud E}{\ud t} \quad [\si{erg.s^{-1}}]
\end{equation}

La \emph{luminosità monocromatica} ($L_\nu$) è la luminosità per unità di frequenza (o lunghezza d'onda), cioè la quantità di energia totale emessa dalla sorgente nell'unità di tempo e nell'intervallo di frequenze compreso tra $\nu$ e $\nu + \ud \nu$:

\begin{equation}\label{eq:def-luminosità-monocromatica}
    L_\nu = \frac{\ud E}{\ud t \ud \nu} \quad [\si{erg.s^{-1}.Hz^{-1}}], \qquad L_\lambda = \frac{\ud E}{\ud t \ud \lambda} \quad [\si{erg.s^{-1}.cm^{-1}}]
\end{equation}

È semplice ricavare la relazione che lega $L_\nu$ e $L_\lambda$, considerando che $\nu = c / \lambda$. Infatti, a un aumento di $\lambda$ corrisponde una diminuzione di $\nu$, ovvero $L_\nu \ud \nu = - L_\lambda \ud \lambda$ e inoltre, si ha $\ud \nu / \ud \lambda = - c / \lambda^2$. Combinando le due espressioni si ottiene:
\begin{equation}\label{eq:luminosità-monocromatica}
    L_\lambda = \frac{c}{\lambda^2} L_\nu
\end{equation}

Inoltre, si può ottenere la luminosità bolometrica a partire da quella monocromatica integrando su tutte le frequenze (o lunghezze d'onda):
\begin{equation}\label{eq:luminosità-bolometrica}
    L = \int_0^\infty L_\nu \ud \nu = \int_0^\infty L_\lambda \ud \lambda
\end{equation}

Ci si può ora chiedere quale sia la relazione tra l'intensità specifica e la luminosità.

\subsection{Relazione tra intensità e luminosità}\label{sec:relazione-intensità-luminosità}
Consideriamo un campo di radiazione isotropo, per cui $I_\nu$ (cfr. par.~\ref{sec:intensità}) è uguale in ogni direzione $\hat{k}$. Questo significa che se considero una superficie infinitesima $\ud A_\perp$\footnote{Il simbolo $\perp$ indica che si sta considerando una superficie tangente alla sfera, ovvero con normale $\hat{k}$, una qualunque direzione radiale.}, con normale $\hat{k}$, la quantità di radiazione entrante e uscente da tale superficie è la stessa per ogni superficie. La luminosità (cfr. par.~\ref{sec:luminosità}) della sorgente è la quantità di radiazione \emph{emessa} dalla sorgente nel tempo $\ud t$, cioè la quantità di radiazione che \emph{esce} dalla sorgente stessa, ovvero la quantità di radiazione che \emph{esce da ciascuna superficie $\ud A_\perp$, integrata sulla superficie totale della sorgente}.

Per ottenere la quantità di radiazione che \emph{esce} da ciascuna superficie infinitesima $\ud A_\perp$, che chiameremo $\ud L_\nu / \ud A_\perp$, bisogna integrare l'intensità della sorgente sull'angolo solido della semisfera \emph{uscente} (cfr. eq~\eqref{eq:intensità} ed eq.~\eqref{eq:def-luminosità-monocromatica}), ovvero, in un sistema di riferimento con asse polare $\hat{k}$ (fig.~\ref{fig:emisfera-uscente}), per $\theta \in [0, \pi/2]$ e $\phi \in [0, 2\pi]$:

\begin{equation}\label{eq:differenziale-luminosità}
    \frac{\ud L_\nu}{\ud A_\perp} = I_\nu \iint \ud \Omega_\textup{uscente} = I_\nu \int_{0}^{\frac{\pi}{2}} \ud \theta \sin{\theta} \int_{0}^{2\pi} \ud \phi
\end{equation}
dove $I_\nu$ è stato portato fuori dagli integrali perché non dipende dagli angoli, essendo la sorgente isotropa.

\begin{figure}
\centering
\includegraphics[width=0.2\textwidth]{immagini/emisfera-uscente.png}
\caption{Calcolo della quantità di radiazione \emph{uscente} da una superficie infinitesima $\ud A_\perp$ posta sulla superficie di una stella. La radiazione uscente è quella che attraversa la semisfera uscente che ha come cerchio massimo la circonferenza di $\ud A_\perp$. Pertanto, in un sistema di riferimento con asse polare $\hat{k}$, bisogna integrare nell'angolo solido sotteso da $\theta \in [0, \pi/2]$ e $\phi \in [0, 2 \pi]$. Per ottenere tutta la radiazione uscente dalla stella, integro in $\ud A_\perp$.}
\label{fig:emisfera-uscente}
\end{figure}

Per ottenere la luminosità monocromatica totale della sorgente (cfr. eq.~\eqref{eq:def-luminosità-monocromatica}), bisogna integrare la~\eqref{eq:differenziale-luminosità} su tutta la superficie $\ud A_\perp$, ovvero integriamo per un'area che va da $0$ a $4\pi R^2$. Ricordando che, in generale $\ud A_\perp = \ud A \cos{\theta}$, come visto in precedenza\footnote{È importante non dimenticare il fattore $\cos{\theta}$, infatti bisogna integrare \emph{sia} in angolo solido \emph{sia} rispetto all'area normale a $\hat{k}$, quindi nel doppio integrale bisogna proiettare l'area generica lungo $\hat{k}$.}, si ha:
\[
    L_\nu = \int_{0}^{4\pi R^2} \ud A_\perp \frac{\ud L_\nu}{\ud A_\perp} = \int_{0}^{4\pi R^2} \ud A \cos{\theta} \frac{\ud L_\nu}{\ud A_\perp} = I_\nu \int_0^{4\pi R^2} \ud A  \int_0^{\frac{\pi}{2}} \ud \theta \sin{\theta} \cos{\theta} \int_0^{2\pi} \ud \phi
\]
Sviluppando i conti si ottiene un'espressione per la luminosità monocromatica:
\begin{equation}\label{eq:luminosità-monocromatica-intensità}
    L_\nu = 4 \pi R^2 \pi I_\nu = 4 {\pi}^2 R^2 I_\nu
\end{equation}
e integrando su tutte le frequenze (eq.~\eqref{eq:luminosità-bolometrica}) si ottiene la luminosità bolometrica:
\begin{equation}\label{eq:luminosità-intensità}
    L = 4 \pi R^2 \pi I = 4 {\pi}^2 R^2 I
\end{equation}

\subsection{L'equazione del trasporto radiativo e le sue soluzioni}\label{sec:trasporto-radiativo}\label{sec:soluzioni-trasporto-radiativo}

\begin{figure}
\centering
\includegraphics[width=0.7\textwidth]{immagini/trasporto-radiativo.png}
\caption{Equazione del trasporto radiativo. Si studia la variazione di intensità della radiazione emessa dalla sorgente a seguito dell'interazione con il materiale interposto, che avviene attraverso assorbimenti ed emissioni. Lo scattering viene computato nell'assorbimento, poiché l'effetto è quello di deviare la radiazione provocando una diminuzione dell'intensità misurata.}
\label{fig:trasporto-radiativo}
\end{figure}

Come già anticipato nel par.~\ref{sec:intro-trasporto-radiativo}, l'\emph{equazione del trasporto radiativo} descrive la variazione dell'intensità della radiazione a causa dell'interazione tra la radiazione stessa e la materia. Infatti, prima di arrivare all'osservatore, la radiazione attraversa della materia, interagendo attraverso assorbimento ed emissione (fig.~\ref{fig:trasporto-radiativo}). L'equazione si presenta così:
\begin{equation}\label{eq:trasporto-radiativo}
    \ud I_\nu = - k_\nu \rho I_\nu \ud S + j_\nu \ud S
\end{equation}
dove appaiono i seguenti termini:
\begin{description}
    \item[$\ud I_\nu$] variazione dell'intensità specifica.
    \item[$k_\nu$] coefficiente di assorbimento per unità di massa od opacità ($\si{cm^2.g^{-1}}$). L'opacità ha le dimensioni di una sezione d'urto per unità di massa ed esprime quanta radiazione viene assorbita per unità di massa e per lunghezza attraversata.
    \item[$\rho$] densità di massa del materiale attraversato.
    \item[$I_\nu$] intensità in input.
    \item[$\ud S$] dimensione infinitesima del materiale attraversato.
    \item[$j_\nu$] coefficiente di emissione [\si{erg.s^{-1}.cm^{-3}.Hz^{-1}.sr^{-1}}].
\end{description}

In genere si definisce:
\begin{equation}
\alpha_\nu = k_\nu \rho
\end{equation}
e l'equazione~\eqref{eq:trasporto-radiativo} assume la seguente forma:
\begin{equation}\label{eq:trasporto-radiativo2}
    \frac{\ud I_\nu}{\ud S} = -\alpha_\nu I_\nu + j_\nu
\end{equation}

La si può risolvere facilmente in due casi limite. Nel caso si \emph{pura emissione}, corrispondente a $\alpha_\nu = 0$ ($k_\nu = 0$ - opacità nulla) si ha:
\[
    \frac{\ud I_\nu}{\ud S} = j_\nu \qquad I_\nu = I_\nu (0) + \int \ud S j_\nu
\]
In questo caso, l'aumento di intensita è pari al coefficiente di emissione integrato lungo la linea di vista.

Nel caso di \emph{puro assorbimento}, corrispondente a $j_\nu = 0$, si ha:
\[
\frac{\ud I_\nu}{\ud S} = -\alpha_\nu I_\nu \quad 
I_\nu (S) = I_\nu (0) e^{-\int \alpha_\nu \ud S}
\]
In questo caso, l'intensità cala come l'esponenziale del coefficiente di assorbimento integrato lungo la linea di vista.

È conveniente introdurre la \emph{profondità ottica} (sarà approfondita successivamente, cfr.~\ref{sec:profondità-ottica}):
\begin{equation}\label{eq:profondità-ottica}
        \tau_\nu = \int k_\nu \rho \ud S = \int \alpha_\nu \ud S
\end{equation}
dove si integra lungo il cammino $\ud S$. La profondità ottica in generale dipende anche dalla frequenza ed è un numero puro. 

È conveniente introdurre anche la \emph{funzione sorgente}:
\begin{equation}\label{eq:funzione-sorgente}
    S_\nu = \frac{j_\nu}{k_\nu \rho} = \frac{j_\nu}{\alpha_\nu}
\end{equation}
Essa esprime il rapporto tra il coefficiente di emissione e quello di assorbimento in termini di $\alpha_\nu$ e ha le stesse unità di misura di una intensità specifica.

Con queste due nuove variabili, l'equazione~\eqref{eq:trasporto-radiativo2} assume una forma particolarmente semplice:
\begin{equation}\label{eq:trasporto-radiativo3}
    \frac{\ud I_\nu}{\ud \tau_\nu} = -I_\nu + S_\nu
\end{equation}
Integriamola, separando le variabili e riscrivendola in una forma più comoda:
\[
    \ud I_\nu = -I_\nu \ud \tau_\nu  + S_\nu \ud \tau_\nu
\]
\[
    \ud I_\nu + I_\nu \ud \tau_\nu  = S_\nu \ud \tau_\nu
\]
\[
    e^{\tau_\nu} (\ud I_\nu + I_\nu \ud \tau_\nu)  = e^{\tau_\nu} (S_\nu \ud \tau_\nu)
\]
\[
    d(e^{\tau_\nu} I_\nu) = e^{\tau_\nu} S_\nu \ud \tau_\nu
\]
Ricordando (cfr.~\ref{sec:profondità-ottica}) che la profondità ottica è un integrale tra $0$ e $S$ e al centro è nulla, come limiti inferiori dell'integrazione scelgo $\tau_{\nu_0} = 0$ e $I_{\nu_0}$, corrispondente all'intensità iniziale, prima dell'interazione con la materia. Suppongo, inoltre, che la funzione sorgente $S_\nu$ non varii con $\tau$, ovvero con la distanza.
\[
    e^{\tau_\nu} I_\nu - I_{\nu_0} = \int_0^{\tau_\nu} e^{\tau_\nu'} S_\nu \ud \tau_\nu' = S_\nu (e^{\tau_\nu} - 1)
\]
Moltiplico ambo i membri per $e^{-\tau_\nu}$
\[
    I_\nu - I_{\nu_0} e^{-\tau_\nu} = S_\nu (1-e^{-\tau_\nu})
\]
da cui si ottiene:
\begin{equation}\label{eq:soluzione-trasporto-radiativo}
    I_\nu = I_{\nu_0} e^{-\tau_\nu} + S_\nu (1- e^{-\tau_\nu})
\end{equation}
Il primo termine è l'intensità iniziale $I_{\nu_0}$(prima dell'interazione con la materia) attenuata da un fattore esponenziale (si ricordi che $\tau \in [1,+\infty]$), mentre il secondo termine \emph{non} dipende dall'intensità iniziale (quella emessa dalla sorgente), ma riguarda solamente il materiale interposto\footnote{Il materiale interposto può anche essere l'atmosfera della stella stessa che si osserva, oppure potrebbe essere l'atmosfera della Terra e così via.}. Trascurando il primo termine, questo secondo termine, che rappresenta l'emissione e l'assorbimento, può anche rappresentare una auto--emissione e un auto--assorbimento nel caso in cui non ci sia una stella "dietro".

\subsection{Trascuro la sorgente di background}\label{sec:trascuro-sorgente-trasporto-radiativo}
Se nell'equazione~\eqref{eq:soluzione-trasporto-radiativo} trascuro la sorgente di background, assume la forma:
\begin{equation}\label{eq:soluzione-no-sorgente-trasporto-radiativo}
    I_\nu = S_\nu (1- e^{-\tau_\nu})
\end{equation}
Questo corrisponde, ad esempio, a osservare il mezzo interposto da una direzione perpendicolare rispetto alla stella, come mostrato in fig.~\ref{fig:trasporto-radiativo-perpendicolare}. Il primo termine, $S_\nu$, rappresenta l'emissione, mentre il secondo termine, $S_\nu e^{-\tau_\nu}$, rappresenta l'auto--assorbimento. Si può studiare l'eq.~\eqref{eq:soluzione-no-sorgente-trasporto-radiativo} considerando due situazioni limite:
\begin{description}
    \item[Regime otticamente spesso] la profondità ottica è molto grande, $\tau_\nu \gg 1$, quindi l'intensità coincide con la funzione sorgente: $I_\nu = S_\nu$
    \item[Regime otticamente sottile] la profondità ottica è molto piccola, $\tau_\nu \ll 1$, quindi l'intensità risulta $I_\nu = S_\nu \tau_\nu$
\end{description}

\begin{figure}
\centering
\includegraphics[width=0.3\textwidth]{immagini/trasporto-radiativo-perpendicolare.png}
\caption{Trascuro la sorgente di background nella soluzione dell'equazione del trasporto radiativo.}
\label{fig:trasporto-radiativo-perpendicolare}
\end{figure}

Ci si può chiedere come sia fatta la funzione sorgente $S_\nu$, definita tramite l'eq.~\eqref{eq:funzione-sorgente}. Essa è diversa a seconda del tipo di radiazione, cioè del processo responsabile del fenomeno di radiazione osservato. Tipicamente i processi di emissione vengono raggruppati in processi termici e processi non termici.
\begin{description}
    \item[Radiazione termica] è emessa da un corpo in equilibrio termico (o termodinamico), in cui tra materia e radiazione avvengono continui processi di emissione e assorbimento di fotoni tali da rendere il tasso di assorbimento uguale al tasso di emissione. Esempi di processi termici sono:
    \begin{itemize}
        \item Radiazione di corpo nero (stelle)
        \item Bremsstrahlung (gas negli ammassi di galassie)
    \end{itemize}
    
    \item[Radiazione non termica] non sono dovuti a processi di emissione e assorbimento, ma, ad esempio, a moto di cariche elettriche oppure a un effetto di scattering. Esempio sono:
    \begin{itemize}
        \item Sincrotrone (pulsar)
        \item Compton (effetto Sinyaev-Zel'dovich)
    \end{itemize}
\end{description}

\subsection{Radiazione termica}
Se, nell'approssimazione del paragrafo~\ref{sec:trascuro-sorgente-trasporto-radiativo} considero solamente processi termici, ovvero processi dovuti a continue emissioni e assorbimenti, la funzione sorgente coincide con la \emph{funzione di Planck} (eq.~\eqref{eq:corpo-nero}):
\[
    S_\nu = B_{BB} (T) = \frac{2 h}{c^2} \frac{\nu^3}{e^{\frac{h \nu}{k_B T}} - 1}
\]
In questo caso, i due regimi di approssimazione si riducono alle seguenti espressioni:
\begin{description}
    \item[regime otticamente spesso ($\tau_\nu \gg 1$)] $I_\nu = B_{BB}$
    \item[regime otticamente sottile ($\tau_\nu \ll 1$)] $I_\nu = B_{BB} \tau_\nu$
\end{description}
È evidente che, se il regime è \emph{otticamente spesso}, la radiazione termica è \emph{radiazione di corpo nero}.
