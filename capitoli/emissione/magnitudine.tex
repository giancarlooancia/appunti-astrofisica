\section{Magnitudine}\label{sec:magnitudine}
\subsection{Magnitudine apparente}\label{sec:magnitudine-apparente}
La \emph{magnitudine apparente} è una misura del flusso ricevuto dallo strumento. È una grandezza introdotta per la prima volta dall'astronomo greco \emph{Ipparco} e varia tra $m=1$, che corrisponde all'oggetto più brillante del cielo, e $m=6$, che corrisponde alle stelle più deboli che si possono osservare a occhio nudo. Si faccia attenzione al fatto che \emph{maggiore è la magnitudine, più debole è la stella}. In particolare, la scala di magnitudini segue la \emph{risposta logaritmica} dell'occhio umano alla luminosità delle stelle che osserviamo sulla volta celeste.

\emph{Norman Robert Pogson} (1856) notò che una magnitudine di $m=6$ corrisponde a una stella $\sim 100$ volte più debole di una con magnitudine $m=1$. Si può quindi osservare che una differenza di $5$ magnitudini corrisponde a un rapporto tra i flussi di $100$:
\[
    m_1=1, \,  m_2=6 \implies F_1 = 100 F_2
\]
\[
    m_2 - m_1 = 5 \implies \frac{F_1}{F_2} = 100 = 100^{\frac{m_2 - m_1}{5}}
\]
Quindi risulta:
\begin{equation}\label{eq:magnitudine-apparente}
    \frac{F_1}{F_2} = 100^{\frac{m_2 - m_1}{5}}
\end{equation}
da ciò segue che una differenza di $1$ magnitudine corrisponde a un rapporto tra i flussi pari a $100^{1 / 5} \sim 2.5$. Quindi, ad esempio, una stella di magnitudine $m=1$ \emph{appare} $2.5$ volte più brillante di una stella di magnitudine $m=2$ e $2.5^5 \sim 100$ volte più brillante di una stella con magnitudine $m=6$. Si parla dunque di magnitudini \emph{apparenti}.

\subsection{Legame tra magnitudine apparente e flusso}\label{sec:relazione-magnitudine-apparente-flusso}
Prendendo il logaritmo (sia $\log$ il logaritmo in base $10$) di entrambi i membri dell'equazione~\eqref{eq:magnitudine-apparente}, si trova:
\[
    \log\frac{F_1}{F_2} = \frac{m_2 - m_1}{5} \log(100) = \frac{2}{5} (m_2 - m_1) = 0.4 (m_2 - m_1)
\]
Da cui segue:
\begin{equation}\label{eq:relazione-magnitudine-apparente-flusso}
    m_1 - m_2 = -2.5 \log\frac{F_1}{F_2}
\end{equation}
Ovvero, considerando i flussi monocromatici:
\begin{equation}\label{eq:relazione-magnitudine-apparente-flusso-monocromatico}
    m_1\nu - m_2\nu = -2.5 \log\frac{F_1\nu}{F_2\nu}
\end{equation}

Misurando il flusso di una sorgente, \emph{non} conosco la sua magnitudine, poiché, come ovvio dall'equazione~\eqref{eq:relazione-magnitudine-apparente-flusso}, posso solamente conoscere la magnitudine \emph{rispetto a un'altra sorgente}. È dunque necessario assegnare un valore di magnitudine apparente fisso attraverso un \emph{sistema fotometrico}. Un esempio è il \emph{sistema fotometrico di Vega}, secondo il quale impongo che la stella Vega abbia magnitudine $m=0$ a tutte le frequenze, sicché la relazione tra magnitudine e flusso diventeranno:
\[
    m_\nu = - 2.5 \log\frac{F_nu}{F_0\nu}
\]
dove $m_\nu$ e $F_\nu$ sono rispettivamente la magnitudine e il flusso della sorgente a una data frequenza $\nu$, mentre $F_0\nu$ è il flusso della stella Vega a quella data frequenza. Per farsi un'idea degli ordini di grandezza nel sistema di Vega si faccia riferimento alla tabella~\ref{tab:sistema-fotometrico-vega}.

\begin{table}
\caption{Magnitudini apparenti nel Sistema Fotometrico di Vega. Il Sole, essendo la stella a noi più vicina, è quella che ci \emph{appare} più luminosa, infatti ha la magnitudine assoluta più piccola possibile, ovvero più negativa possibile in questo caso. Il flusso dipende sia dalla luminosità della sorgente sia dalla distanza della sorgente dall'osservatore.}
\label{tab:sistema-fotometrico-vega}
\centering
\begin{tabular}{lS}
\toprule
Oggetto & {Magnitudine apparente} \\
\midrule
Sole         & -26.8 \\
Luna piena   & -12.6 \\
Venere       & -4.7 \\
Giove, Marte & -2.8 \\
Sirio & -1.5 \\
Vega & 0.0 \\
Uranio & 5.5 \\
Più deboli visibili a occhio & 6.0 \\
Nettuno & 7.7 \\
Plutone & 13.0 \\
\bottomrule
\end{tabular}
\end{table}

\subsection{Magnitudine assoluta}\label{sec:magnitudine-assoluta}
Si definisce \emph{magnitudine assoluta} la magnitudine apparente che avrebbe una sorgente se fosse a una distanza di $\SI{10}{parsec}$ ($\SI{1}{pc} = \SI{3.086e18}{cm}$):
\begin{equation}\label{eq:magnitudine-assoluta}
    M = m_{\SI{10}{pc}}
\end{equation}
Se la magnitudine apparente \emph{non} dà informazioni sulla brillantezza intrinseca della sorgente, essendo dipendente dalla distanza della sorgente stessa, la magnitudine è legata alla luminosità intrinseca della sorgente.

Si può ricavare la relazione che sussiste tra la magnitudine assoluta e la luminosità utilizzando l'equazione~\eqref{eq:relazione-magnitudine-apparente-flusso}
\[
    M_1 - M_2 = m_{1, \SI{10}{pc}} - m_{2, \SI{10}{pc}} = -2.5 \log\frac{F_1 (\SI{10}{pc})}{F_2 (\SI{10}{pc})}
\]
e utilizzando l'equazione~\eqref{eq:flusso} per scrivere
\[
    \frac{F_1 (\SI{10}{pc})}{F_2 (\SI{10}{pc})} = \frac{L_1}{L_2}
\]
Da ciò segue che:
\begin{equation}\label{eq:magnitudine-assoluta-luminosità}
    M_1 - M_2 = -2.5 \log\frac{L_1}{L_2}
\end{equation}

\subsection{Modulo di distanza}
Conoscendo la magnitudine apparente e la magnitudine assoluta di una sorgente, è possibili stimare la sua distanza attraverso il così detto \emph{modulo di distanza}. Troviamo la relazione tra $m$ e $M$ utilizzando l'equazione~\eqref{eq:relazione-magnitudine-apparente-flusso}
\[
    m - M = m - m_{\SI{10}{pc}} = -2.5 \log\frac{F(d)}{F(\SI{10}{pc})}
\]
e usando l'equazione~\eqref{eq:flusso} con $r=d$
\[
    \frac{F(d)}{F(\SI{10}{pc})} = \frac{10^2}{d^2}
\]
mettendo insieme si trova
\[
    m - M = -5 \log\frac{10}{d} = -5 + 5 \log d
\]
Da cui l'espressione per il \emph{modulo di distanza}:
\begin{equation}\label{eq:modulo-distanza}
    m - M = -5 + 5 \log(d_{\si{pc}})
\end{equation}
dove con il pedice si è sottolineato che le distanze sono espresse in \emph{parsec}.

Siccome siamo in grado di misurare la distanza del Sole e possiamo misurare la sua magnitudine apparente a varie lunghezze d'onda, utilizzando l'equazione~\eqref{eq:modulo-distanza} è possibile ricavare le magnitudini assolute del sole a varie lunghezze d'onda, ad esempio si ha (B, V e K sono tre differenti filtri fotometrici):
\[
    M_\ensuremath{\textup{bol}\odot} = 4.75 \quad M_\ensuremath{\textup{B}\odot} = 5.48 \quad M_\ensuremath{\textup{V}\odot} = 4.83 \quad M_\ensuremath{\textup{K}\odot} = 3.31
\]
Queste sono utilizzate come riferimento per esprimere la luminosità e la magnitudine assoluta per ogni altra sorgente. Infatti, richiamando l'equazione~\eqref{eq:magnitudine-assoluta-luminosità} e imponendo $M_2 = \si{\solarmass}$ e $L_2 = \si{\solarluminosity}$, si ottiene
\begin{equation}\label{eq:magnitudine-assoluta-luminosità-sole}
    M_1 - \si{\solarmass} = -2.5 \log\frac{L_1}{\si{\solarluminosity}}
\end{equation}