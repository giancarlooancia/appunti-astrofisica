\section{Flusso}\label{sec:flusso}
L'intensità (par-~\ref{sec:intensità}) e la luminosità (par-~\ref{sec:luminosità}) sono quantità \emph{intrinseche} della sorgente, ma \emph{non} sono osservabili. Infatti, più una sorgente è lontana da noi, più ci appare debole, ma la sua luminosità non cambia con la distanza, poiché l'energia emessa ogni secondo è sempre la stessa. Ciò che osserviamo è il \emph{flusso}, definito come la luminosità per unità di area: l'energia per unità di tempo per unità di area che arriva nel nostro strumento. Ovviamente il flusso dipende dalla distanza: diminuisce all'aumentare della distanza dalla sorgente.

Ipotizzando una emissione isotropa, il flusso misurato a una distanza $r$ da una sorgente che emette con una luminosità $L$ è dato da:
\begin{equation}\label{eq:flusso}
    F(r) = \frac{L}{4 \pi r^2}
\end{equation}
Come nei casi già visti, è possibile definire un \emph{flusso monocromatico}, per unità di frequenza, $F_\nu$, e trovare il flusso totale come:
\[
    F = \int_0^\infty F_\nu \ud \nu
\]
Il flusso e la luminosità delle sorgenti sono spesso espressi in termini di \emph{magnitudini}.