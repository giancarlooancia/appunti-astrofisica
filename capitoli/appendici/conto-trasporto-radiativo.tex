\section{Equazione del trasporto radiativo}\label{app:trasporto-radiativo}
Integriamo l'equazione~\eqref{eq:trasporto-radiativo3}, che per comodità riportiamo sotto:
\[
\frac{\ud I_\nu}{\ud \tau_\nu} = -I_\nu + S_\nu
\]
Separo le variabili e riscrivo l'equazione in una forma più comoda
\[
    \ud I_\nu = -I_\nu \ud \tau_\nu  + S_\nu \ud \tau_\nu
\]
\[
    \ud I_\nu + I_\nu \ud \tau_\nu  = S_\nu \ud \tau_\nu
\]
\[
    e^{\tau_\nu} (\ud I_nu + I_\nu \ud \tau_\nu)  = e^{\tau_\nu} (S_\nu \ud \tau_\nu)
\]
\[
    d(e^{\tau_\nu} I_\nu) = e^{\tau_\nu} \ud I_\nu + I_\nu e^{\tau_\nu} \ud \tau_\nu
\]
\[
    d(e^{\tau_\nu} I_\nu) = e^{\tau_\nu} S_\nu \ud \tau_\nu
\]
Ricordando che la profondità ottica è un integrale tra $0$ e $S$ e al centro è nullo, come limiti inferiori dell'integrazione scelgo $\tau_{\nu_0} = 0$ e $I_{\nu_0}$, corrispondente all'intensità iniziale, prima dell'interazione con la materia. Suppongo, inoltre, che la funzione sorgente $S_\nu$ non vari con $\tau$, ovvero non vari con la distanza.
\[
    e^{\tau_\nu} I_\nu - I_{\nu_0} = \int_0^{\tau_\nu} e^{\tau_\nu'} S_\nu \ud \tau_\nu' = S_\nu (e^{\tau_\nu} - 1)
\]
Moltiplico ambo i membri per $e^{-\tau_\nu}$
\[
    I_\nu (\tau_\nu) - I_{\nu_0} e^{-\tau_\nu} = S_\nu (1-e^{-\tau_\nu})
\]
da cui si ottiene la soluzione~\eqref{eq:soluzione-trasporto-radiativo}