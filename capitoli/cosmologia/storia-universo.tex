\section{(Breve) storia dell'Universo}\label{sec:storia-universo}
\subsection{Primi secondi}\label{primi-secondi}
Assumendo che il tempo abbia avuto origine con il Big Bang e che quindi questo sia il momento che possiamo assumere $t_0=0$, nel brevissimo intervallo che intercorre tra $t_0$ ed il tempo di Planck $t_p$ non è possibile avanzare alcuna predizione sulla natura delle leggi che governavano l'universo. Questo è il limite della conoscenza fisica.

Quello che si assume è che in origine, a causa delle elevatissime energie, tutte le interazioni fondamentali erano unificate in una sola, separandosi poi al diminuire della densità dell'universo. Ad un tempo dell'ordine di grandezza del tempo di Planck avviene una scissione tra la forza gravitazionale e le altre interazioni (nucleare forte ed elettrodebole). Dopo circa $\SI{e-35}{s}$ dal Big Bang la forza elettrodebole e quella nucleare forte si spaiano, rilasciando un'enorme quantità di energia causando una rapidissima espansione: \textit[Inflazione].