\section{(Breve) storia dell'universo}\label{sec:storia-universo}
\subsection{Inflazione}\label{inflazione}
Assumendo che il tempo abbia avuto origine con il Big Bang e che quindi questo sia il momento che possiamo assumere $t_0=0$, nel brevissimo intervallo che intercorre tra $t_0$ ed il tempo di Planck $t_p$ non è possibile avanzare alcuna predizione sulla natura delle leggi che governavano l'universo. Questo è il limite della conoscenza fisica.

Quello che si assume è che in origine, a causa delle elevatissime energie, tutte le interazioni fondamentali fossero unificate in una sola, separandosi poi al diminuire della densità e dell'energia nell'universo. Ad un tempo dell'ordine di grandezza del tempo di Planck si ipotizza avvenire una scissione tra la forza gravitazionale e le altre interazioni (nucleare forte ed elettrodebole). Dopo circa $\SI{e-35}{s}$ dal Big Bang la forza elettrodebole e quella nucleare forte si spaiano, rilasciando un'enorme quantità di energia che causa una rapidissima espansione: \textit[Inflazione]. Questa è necessaria per spiegare l'omogeneità e l'isotropia dell'universo (problema dell'orizzonte).

Il CMB mostra un cielo estremamente omogeneo in tutte le direzioni, implicando che tutte le regioni dell'universo sarebbero dovute essere casualmente connesse, a prescindere dalla distanza, e quindi comunicanti in un qualche istante della vita dell'universo. Si tratta di punti a distanza anche molto maggiore rispetto a $ct_U$, con $t_U$ il tempo dal Big Bang ad oggi, e quindi non casualmente connessi, dal momento che $c$ è una velocità limite e nulla viaggia più veloce di essa. Non posso, perciò, essere entrati in contatto. La soluzione a questo problema è introdurre l'inflazione, ovvero, ad un tempo che va da $\sim \SI{e-35}{s}$ a $\sim \SI{e-32}{s}$, lo spazio tempo ha attraversato un periodo di espansione esponenziale, aumentando le proprie dimensioni di un fattore $\sim 10^{50}$. La soluzione del problema dell'orizzonte, rieiede quindi nell'assunzione che zone dell'universo che ora sono estremamente distanti, prima dell'inflazione erano abbastanza vicine da essere casualmente connesse.

\subsection{Fasi della vita dell'universo}\label{sec:fasi-universo}
In origine l'universo era in uno stato di plasma estremamente caldo, pieno di fotoni estremamente energetici e con continua creazione di coppie particella-antiparticella. Era presente un equilibrio tra produzione e annichilimento di queste, grazie al principio di equivalenza massa energia. 

All'aumentare dell'età dell'universo, però, aumentava anche la lunghezza d'onda dei fotoni $\lambda$, implicando una diminuzione della loro energia $E_\gamma$ e quindi della possibilità di generare nuove particelle. Tutta la materia che veniva creata si era completamente annichilita lasciando solo un piccolissimo residuo di particelle ($1$ su $10^9$ coppie), cioè protoni, neutroni ed elettroni di cui è costituito la materia. 

Dopo appena un minuto dal Big Bang la temperatura è diminuita fino a $\SI{e9}{K}$, alla quale l'energia dei fotoni è abbastanza bassa da permettere la formazione e la sostentazione dei primi nuclei atomici e man mano che la temperatura scendeva si sono cominciati a produrre i primi elementi: Deuterio \ce{^2 H}, Elio \ce{He} e alcune tracce di Litio \ce{Li}. Il fatto che in natura non esistano elementi stabili con numero atomico $5$ o $8$ ha ostacolato la formazione di elementi più pesanti di \ce{He^4} durante il Big Bang (non è stato possibile generare elementi più pesanti di $A=8$ a causa del breve tempo trascorso e della bassa temperatura raggiunta). Tutti gli elementi più pesanti sono stati risultato della fusione nelle stelle o della loro esplosione in supernove.

Tutto questo è avvenuto in un periodo che prende il nome di \textit{Epoca della Radiazione}, durata circa $\SI{e4}{yr}$ dopo il Big Bang, in cui la densità di massa di pressione di radiazione sovrastava tutte le altre. A questa fase è poi seguita l'\textit{Epoca della Materia}. Dopo l'esplosione della singolarità, l'universo era permeato da una distribuzione di materia e radiazione che, dall'uguaglianza energia massa, è possibile riassumere definendo anche una densità di massa-energia. A causa dell'espansione dello spazio-tempo, entrambe queste densità diminuiscono all'aumentare del tempo, seguendo, come visto nella sezione precedente, un andamento differente:
\begin{equation}\label{eq:densita-massa}
    \rho_m = \rho_{m,0} = {(1+z)}^3
\end{equation}
\begin{equation}\label{eq:densita-radiazione}
    \rho_{rad}=\rho_{rad, 0} {(1+z)}^4
\end{equation}
Ricostruendo l'andamento della densità all'interno dell'universo dai suoi istanti primordiali, conoscendo il valore che questa assume oggi, ad un istante del della sua evoluzione è avvenuta un uguaglianza tra i due tipi di densità $\rho_m = \rho_{rad}$. In questo istante, se si eguagliano (\refeq{eq:densita-massa}) e (\refeq{eq:densita-radiazione}), si ottiene il valore redshift $z_{eq}$ al quale queste si bilanciavano.
\[
    \rho_{m,0} = {(1+z)}^3 = \rho_{rad, 0} {(1+z)}^4
\]
\[
    z_{eq} = \frac{\rho_{m,0}}{\rho_{rad, 0}} - 1
\]
Un rapido studio di funzione mostra che $\rho_m$ decresce più lentamente rispetto a $\rho_{rad}$, prima del punto di equivalenza, $\rho_{rad}$ dominava su $\rho_{m}$. In questa epoca (epoca della radiazione), la radiazione è in equilibrio termico con la materia, a causa delle continue interazioni con gli elettroni, generando fenomeni di scattering, osservando uno spettro identico a quello di un corpo nero alla temperatura a cui si trova la materia. Ci si trova quindi in un universo spesso, opaco alla radiazione.

Quando le temperature scendono sotto i $\SI{4000}{K}$ il numero di fotoni ad alta energia diminuisce abbastanza da permettere la ricombinazione di protoni e neutroni nei nuclei, che assieme aegli elettroni, formando i primi atomi. Dalla fisica nucleare si è a conoscenza del fatto che in realtà gli atomi di Idrogeno si formano a temperature al di sotto di $\sim \SI{10000}{K}$, ciò nonostante l'energia dei fotoni sarebbe troppo elevata, causando una ionizzazione immediata. Questa parte della vita dell'universo viene detta \textit{Epoca della Ricombinazione} e si trova ad un redshift $z_{rec} \sim 1100$ ($\sim \SI{e5.5}{yr})$ dal Big Bang. Lo scattering, in questa fase, diventa trascurabile ed i fotoni smettono d'interagire con la materia, viaggiando indisturbati. Si ottiene, quindi, un universo trasparente in cui la materia non è più accoppiata alla radiazione, in particolare la radiazione perde progressivamente energia, mantenendo però lo spettro di un corpo nero. Il CMB, perciò, non è altro che questa radiazione raffreddata, che mostra come apparivano le ultime superfici soggette a scattering.

Poiché l'universo era opaco fino a $z_{rec} \sim 1100$, non è possibile osservare più indietro dell'epoca della ricombinazione, l'epoca della radiazione è quindi inaccessibile attraverso lo studio della radiazione. Il CMB è quindi la fotografia più antica che è possibile osservare del nostro universo.

Dopo la ricombinazione, l'universo è composto da Idrogeno neutro e l'unica fonte di radiazione è la radiazione cosmica di fondo. Non è quindi presente alcuna fonte luminosa, motivo per cui questa fase viene denominata con il termine \textit{Epoca Buia}.

L'universo oggi, però, appare quasi completamente ionizzato, perciò deve essere apparsa una fonte di raggi UV o X nel periodo che va tra $z_{rec} \sim 1100$ e $z = 0$, che fosse in grado di ionizzare la materia presente. Questo è probabile sia avvenuto tra $z \sim 10$ e $z \sim 6$, ma la fonte di questa energia è ancora un'incognita. I candidati più probabili sono, però, la prima stella di popolazione III, la radiazione emessa dall'esplosione della prima stella o quella proveniente da un QSO, è infatti questo il periodo in cui si sono formate le prime stelle e galassie. L'attivazione della formazione di queste strutture cosmiche è ancora frutto di speculazioni, ma si pensa essere dovuta alla presenza d'instabilità gravitazionali o fluttuazioni di distribuzione di materia, dovuta alla locale non omogeneità del fluido che hanno poi portato all'aggregazione di gas, come conseguenza dell'inflazione. Dal momento che la radiazione era legata alla materia prima della ricombinazione, perturbazioni in temperature erano equivalenti a perturbazione di densità di materia e dall'osservazione del CMB le fluttuazioni di temperatura erano molto presenti nell'epoca della radiazione.

A un certo punto dell'epoca della materia, infine, la materia oscura deve essersi slegata dalla radiazione (ipotizzando l'esistenza di una strana interazione tra radiazione e materia oscura, la cui natura è del tutto sconosciuta) generando delle fluttuazioni di densità di massa, provocando delle instabilità gravitazionali. Queste fungono da buche di potenziale per la materia barionica, che nel frattempo si è slegata dalla radiazione, formando le prime galassie e stelle. L'accrescimento delle strutture attraverso instabilità gravitazionali è descritto attraverso l'approccio di Jeans, visto nella sezione~\ref{sec:massa-jeans}, modificato in modo tale da applicarlo all'espansione di un fluido (applicabile per fluttuazioni inferiori alla densità media: $\Delta \rho \ll \bar{\rho}$). La massa di Jeans necessaria dipende dalla natura della materia oscura, in particolare per la \textit{Cold Dark Matter}, ovvero per materia oscura che al momento della scissione dalla radiazione non ha energie relativistiche, $M_J \sim 10^5 \si{\solarmass}$. Questa è la massa più probabile che le prime strutture cosmiche avessero al momento della loro formazione.

Il modo in cui l'universo si è evoluto segue in andamento detto \textit{Bottom-up} o \textit{Gerarchico}, formando prima le strutture più piccole e meno massive, per poi costruire quelle più grandi attraverso l'aggregazione di queste più piccole. Era stato ipotizzato anche un andamento \textit{Top-down}, associato ad una \textit{Hot Dark Matter}, ma si è rivelata un'ipotesi di meno successo.