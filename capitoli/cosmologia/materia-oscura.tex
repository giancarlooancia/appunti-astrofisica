\section{Materia Oscura}\label{sec:materia-oscura}
\subsection{Evoluzione della densità}\label{sec:evoluazione-densita}
Mettendo insieme l'equazione~\ref{eq:friedman} di Friedman, l'equazione~\ref{eq:second-friedman} di accelerazione e l'equazione~\ref{eq:equazione-stato} di stato, mantenendo l'assunzione che $w$ sia costante nel tempo, si ottiene che:
\[
    \rho = \rho_0 (1+z)^{3(1+w)} = \rho_0 R(t)^{-3(1+w)}
\]
Si osserva che la densità dell'universo sta diminuendo all'aumentare del tempo, al fine di studiare meglio questo andamento scomponiamola nella componente di materia e la componente di radiazione. La prima, che tiene, per l'appunto, conto della materia barionica e della cold dark matter (CDO) ed è di tipo non relativistica ($P<<\rho c^1$), assumendo $w = 0 \Rightarrow P=0$, sarà
\[
    \rho_m = \rho_{m,0}(1+z)^3
\]
La seconda, invece, tiene conto della radiazione, la quale avrà una dinamica relativistica, con $w = 1/3$ e di conseguenza
\[
    \rho_{rad} = \rho_{rad, 0}(1+z)^4
\]
Si osserva quindi che entrambe le densità fossero estremamente elevate in passato, ma con il tempo sono decresciute con andamenti differenti, in particolare la densità di materia decresce più velocemente di quella di radiazione. Per questo motivo ggi, la densità di materia è circa $1000$ volte maggiore della densità di radiazione $(\rho_{m,0} \sim \rho_{rad, 0}$).

Un'altra componente dell'universo si pensa essere l'energia oscura, ovvero una forma di energia ancora sconosciuta, ma che avrebbe come effetto l'accelerazione dell'espansione dello spazio tempo ($\ddot{R}(t)>0$). Un caso particolare sarebbe quello dell'energia del vuoto, la quale possiede un legame molto stretto con la costante cosmologica ed è caratterizzata da un'equazione di stato con $w = -1$. In particolare se l'energia oscura fosse l'energia del vuoto, allora questa sarebbe ben descritta da un valore $\rho_{\Lambda}$ costante durante tutta l'evoluzione dell'universo.

Gli effetti di un'espansione in accelerazione può essere visto considerando un universo dominato da materia, assumendo quindi che
\[
    \rho_{m} = \rho_{m, 0}{(1+z)}^3 = \rho_{m,0}{R(t)}^{-3}
\]
mentre dall'equazione di Friedman 