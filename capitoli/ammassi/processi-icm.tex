\section{Processi relativi a ICM}
Affrontiamo ora i processi di raffreddamento, di riscaldamento e di formazione dell’intra-cluster medium.

\subsection{Raffreddamento}
A causa dell’emissione di radiazione tramite il fenomeno di Bremsstrahlung (cap.~\ref{sec:bremsstrahlung}) l’ICM si raffredda e i tempi caratteristici di tale raffreddamento seguono dalla seguente relazione:
\begin{equation}
    t_\textup{cool} \approx \num{6e3}\,\frac{T^{\frac{1}{2}}}{n_e}\:\si{anni}
\end{equation}

\begin{description}
    \item ove $n_e$ rappresenta la densità degli elettroni nel gas.
\end{description} 

Inserendo dei plausibili valori per $n_e = \SI{e-3}{cm^{-3}}$ e $T = \SI{e8}{K}$ si ottiene un tempo caratteristico $t_\textup{cool}\sim\SI{6e10}{anni}$ e questo comporta un raffreddamento in media non rilevante.
Tuttavia, il raffreddamento dell'ICM può risultare rimarchevole: si pensa che la densità di elettroni $n_e$ al centro del gas sia maggiore rispetto ad altre regioni, questo porta a tempi di raffreddamento minori e dunque ci si aspetta di osservare flussi di gas freddo, i quali potrebbero spiegare alcuni fenomeni relativi alla formazione della BCG (cap.~\ref{sec:brightest-cluster-galaxy}). Fino ad oggi non vi sono prove dirette di tale fenomeno.

\subsubsection{Brightest Cluster Galaxy}\label{sec:brightest-cluster-galaxy}
Ogni ammasso galattico possiede una galassia più luminosa di tutte le altre che viene detta Brightest Cluster Galaxy o BCG.
Tali galassie si trovano solitamente al centro del relativo ammasso e nella maggior parte dei casi sono galassie ellittiche giganti con una brillanza superficiale molto estesa e un alone stellare molto debole.
Le origini e i processi di formazione di tali galassie rimangono ancora un mistero, ma vi sono delle possibili ipotesi:

\begin{enumerate}
    \item I flussi freddi di ICM, dovuti all’elevata densità elettronica della regione centrale del gas, potrebbero spiegare in parte il fenomeno
    \item La formazione di BCG potrebbe essere legata a rapide fusioni di galassie massive durante la fase iniziale della loro formazione (scenario più accreditato)
    \item Un'ultima ipotesi considera la fusione di galassie nel centro dell’ammasso dovuta ai fenomeni di frizione dinamica e di \emph{tidal stripping}, ovvero quando una galassia più grande strappa materiale stellare ad una galassia più piccola, durante la fase evolutiva dell’ammasso
\end{enumerate}

\subsection{Riscaldamento}
Cerchiamo ora di spiegare il perché dell'elevata temperatura dell'ICM.
Una prima stima di tale temperatura è ottenibile grazie alla seguente relazione, valida nella buca di potenziale dell'ammasso:

\begin{equation}
    \frac{3}{2} \frac{k}{\mu} \frac{T}{H} = -\phi =\sigma ^2
\end{equation}

\noindent Il primo membro corrisponde all'energia interna per unità di massa delle particelle di gas.

\begin{description}
    \item $\phi$ il potenziale gravitazionale dell'ammasso per unità di massa
    \item $\sigma^2$ è la dispersione di velocità dell'ammasso
\end{description}

La temperatura osservata tramite misurazioni è dell'ordine di $T \approx 10^7-\SI{e8}{K} $ compatibile con la temperatura che si ottiene supponendo il gas in equilibrio viriale e sostituendo $\sigma \sim \SI{1000}{km.s^{-1}}$.
Tuttavia, i processi di raffreddamento del gas sono attivi ed è dunque necessario tenere in considerazione ulteriori meccanismi:

\begin{itemize}
    \item AGN feedback (ref.~\ref{co-evoluzione-galassie-agn})
    \item Moto del gas durante i processi di merge 
    \item Riscaldamento dovuto all'"attrito" conseguente al moto delle galassie 
    \item Esplosione di supernove
\end{itemize}

\subsection{Formazione}
Ricordiamo che la metallicità dell'ICM ($Z\sim 0.3 Z_{\odot}$) indica che l'origine del gas non è di tipo primordiale, vi sono tre principali ipotesi per spiegare ciò:

\begin{enumerate}
    \item Si tratta di un gas primordiale arricchito da popolazioni stellari pre-galattiche (Pop $III$ ref.~\ref{pop-stellare-via-lattea}).
    \item Il gas è stato espulso dalle galassie tramite esplosione di supernove
    \item Il gas è stato strappato (non espulso) dalle galassie tramite processi dovuti a:
    \begin{itemize}
        \item pressione d'ariete (\emph{ram pressure}): particolarmente efficiente nel centro dell'ammasso
        \item evaporazione: il calore fluisce dalle regioni di gas caldo a quelle più fredde
        \item vere e proprie collisioni che avvengono fra galassie dell'ammasso
    \end{itemize}
\end{enumerate}

L'ipotesi $3$ spiegherebbe inoltre la sovrabbondanza di galassie ellittiche, lenticolari e spirali anemiche (forma a spirale, ma con bracci poco luminosi) negli ammassi rispetto a zone dell'universo poco dense. Dove le collisioni non possono avvenire per mancanza di materiale.
