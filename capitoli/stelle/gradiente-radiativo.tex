\section[Gradiente radiativo]{Gradiente radiativo e criterio di Schwarzschild}\label{sec:gradiente-radiativo}
L'equazione del \emph{gradiente radiativo} fornisce il profilo radiale delle variazioni di $T$ all'interno della stella. Si può scrivere:
\begin{equation}\label{eq:gradiente-radiativo}
    \dfrac{\ud T}{\ud r}\Big|_\textup{rad} = - \dfrac{3 \kappa \rho}{4 \pi r^2} \dfrac{L(r)}{4 a c T^3}
\end{equation}
Essa mostra che l'opacità e il flusso radiativo determinano quanto rapidamente $T$ varia con $r$.

Utilizzando questa equazione è possibile determinare quando c'è convezione nella stella, utilizzando il seguente \emph{criterio di Schwarzscild}:
\begin{equation}\label{eq:criterio-schwarzscild}
    \text{se} \quad \abs*{\dfrac{\ud T}{\ud r}}_\textup{rad} > \abs*{\dfrac{\ud T}{\ud r}}_\textup{ad} \implies \text{c'è convezione.}
\end{equation}
in particolare, quando il criterio è verificato avverrà la convezione.

\subsection{Ricavare l'equazione}
Per trovare il gradiente radiativo prima di tutto ricaviamo il gradiente di pressione di radiazione, derivando rispetto a $r$ l'espressione~\eqref{eq:pressione-radiazione}. Si ottiene:
\[
\dfrac{\ud P_\textup{rad}}{\ud r} = \frac{4}{3} a T^3 \dfrac{\ud T}{\ud r}
\] 
d'altra parte, il gradiente di radiazione dipende anche dall'\emph{opacità} e dal \emph{flusso} di radiazione. Si ricorda che l'opacità $\kappa$ era stata introdotta nell'eq.~\eqref{eq:trasporto-radiativo} del par.~\ref{sec:soluzioni-trasporto-radiativo}, mentre il flusso, $F_\textup{rad}$, è definito dall'eq.~\eqref{eq:flusso}. In particolare, senza dare ulteriori specificazioni, possiamo esprimere il gradiente della pressione di radiazione come:
\[
\dfrac{\ud P_\textup{rad}}{\ud r} = -\dfrac{\kappa \rho}{c} F_\textup{rad}
\]
e mettendo insieme le utlime due relazioni si ottiene:
\[
\dfrac{\ud T}{\ud r}\Big|_\textup{rad} = - \dfrac{3}{4ac} \dfrac{\kappa \rho}{T^3} F_\textup{rad}
\]
con
\[
F_\textup{rad} = \dfrac{L_r}{4 \pi r^2}
\]
da cui segue immediatamente la~\eqref{eq:gradiente-radiativo}. Come già detto, l'equazione fornisce il profilo radiale delle variazioni di $T$ con $r$ e mostra come l'opacità $\kappa$ e il flusso radiativo $F_\textup{rad}$ influiscono sulla rapidità di variazione di $T$ con $r$.

Il gradiente di temperatura ha un impatto cruciale sul meccanismo preponderante di trasporto di energia all'interno della struttura stellare. Introduciamo brevemente i meccanismi di trasporto di energia.

\subsection{Meccanismi di trasporto di energia}
I tre meccanismi di trasporto in una stella sono il \emph{trasporto radiativo}, \emph{convettivo} e \emph{conduttivo}. Di seguito ne sono elencate le principali proprietà:
\begin{description}
    \item[trasporto conduttivo] I principali responsabili di questo meccanismo sono gli \emph{elettroni} ed è efficiente solo se il gas è \emph{degenere}. Infatti, in condizione di non degenerazione, ovvero se il gas è perfetto, il libero cammino medio è molto piccolo e un elettrone cede subito energia. Come visto, nel caso di un gas degenere (par.~\ref{sec:principio-indeterminazione}) le celle dello spazio delle fasi sono impacchettati in modo che i livelli energetici più bassi sono pieni, quindi gli elettroni percorrono una grande distanza prima di cedere energia (la quale deve essere di un ordine pari al primo livello energetico libero, che tendenzialmente sarà alto). In questo caso il trasporto conduttivo è efficiente e in questo modo è possibile trasportare l'energia dall'interno verso l'esterno. Tuttavia, questo meccanismo di trasporto non è quello preponderante.
    \item[trasporto radiativo] Esso è dovuto alla radiazione trasportata dai \emph{fotoni}.
    \item[trasporto convettivo] Con la convezione ho un rimescolamento del \emph{gas} e dunque di porzioni di gas con composizioni chimiche diversa. Siccome la struttura chimica è cruciale per stabilire la struttura stellare e la sua evoluzione, è importante capire se è in atto questo meccanismo di trasporto energetico.
\end{description}

\subsection{Criterio di Schwarzschild}
Come evidenziato nel paragrafo precedente, stabilire se sia in corso un meccanismo di convezione è importante per realizzare dei corretti modelli della struttura stellare e dunque per capire correttamente l'evoluzione di una stella. Come discriminante si può utilizzare il \emph{criterio di Schwarzschild} (eq.~\eqref{eq:criterio-schwarzscild}). Semplicemente si tratta di un confronto tra il \emph{gradiente radiativo} di temperatura e un valore di riferimento chiamato \emph{gradiente adiabatico}. In particolare, nelle regioni in cui il gradiente radiativo è maggiore del gradiente adiabatico, è in corso la convezione, come stabilito dalla~\eqref{eq:criterio-schwarzscild}. Il gradiente adiabatico dipende dai calori specifici del gas e si può scrivere:
\begin{equation}\label{eq:gradiente-adiabatico}
    \dfrac{\ud T}{\ud r}\Big|_\textup{ad} = \Bigl(1- \frac{1}{\gamma}\Bigr) \dfrac{T}{P} \dfrac{\ud P}{\ud r}
\end{equation}
dove
\[
\gamma =\frac{c_P}{c_V} = \frac{5}{3}
\]

\begin{figure}
\centering
\includegraphics[width=0.3\textwidth]{immagini/criterio-schwarzscild.png}
\caption{Raffigurazione del criterio di Schwarzscild. Se la bolla si sposta dalla posizione $1$ e $2$, può avvenire la convezione solamente se ${\rho_2}^* < \rho_2$. In caso contrario siamo in presenza di equilibrio stabile e la bolla viene respinta verso la posizione iniziale.}
\label{fig:criterio-schwarzscild}
\end{figure}

Per capire il criterio~\eqref{eq:criterio-schwarzscild} consideriamo (fig.~\ref{fig:criterio-schwarzscild}) una bolla di gas in una posizione $1$ a distanza $r$ rispetto al centro, di densità ${\rho_1}^*$ e pressione interna ${P_1}^*$. Siano $\rho_1$ e $P_1$ rispettivamente la densità e la pressione dell'ambiente circostante la bolla in quella posizione. Immagino che essa si sposti in una posizione $2$ a distanza $r + \ud r$ dal centro, avendo una nuova densità ${\rho_2}^*$ e una nuova pressione interna ${P_2}^*$. Analogamente a prima, siano $\rho_2$ e $P_2$ rispettivamente la densità e la pressione dell'ambiente circostante la bolla in quella posizione. Ho convezione nella stella solo se l'equilibrio è instabile, perché in caso di equilibrio stabile la bolla tenderebbe a tornare nella posizione originaria $1$ e non sarebbe possibile lo spostamento di porzioni di gas nella stella. La condizione di equilibrio dipende dal rapporto della densità interna della bolla nel secondo caso, ${\rho_2}^*$, e la densità del gas circostante, $\rho_2$. In particolare:
\begin{subequations}
\label{eq:relazioni-densità-schwarzscild}
\begin{align}
  \text{se} \quad {\rho_2}^* &> \rho_2 \implies \text{la bolla torna nella posizione iniziale $1$.} \\
  \text{se} \quad {\rho_2}^* &< \rho_2 \implies \text{la bolla continua ad andare verso l'alto.} 
\end{align}
\end{subequations}
ed è semplice tradurre le espressioni~\eqref{eq:relazioni-densità-schwarzscild} in termini dei gradienti di temperatura, ottenendo~\eqref{eq:criterio-schwarzscild}.

\subsection{Esempio per il gas perfetto}
\begin{figure}
\centering
\subfloat[][\emph{Caso in cui avviene la convezione: modulo del gradiente radiativo maggiore del gradiente adiabatico.} \label{fig:criterio-schwarzscild-gas-perfetto1}]
{\includegraphics[width=.35\textwidth]{immagini/criterio-schwarzscild-gas-perfetto1.png}} \qquad
\subfloat[][\emph{Caso in cui non avviene la convezione: modulo del gradiente radiativo minore del gradiente adiabatico.}\label{fig:criterio-schwarzscild-gas-perfetto2}]
{\includegraphics[width=.35\textwidth]{immagini/criterio-schwarzscild-gas-perfetto2.png}} 
\caption{Applicazione del criterio di Schwarzscild nel caso di un gas perfetto.}
\label{fig:criterio-schwarzscild-gas-perfetto}
\end{figure}

Per capire meglio le condizioni~\eqref{eq:relazioni-densità-schwarzscild} e il criterio di Schwarzscild~\eqref{eq:criterio-schwarzscild}, consideriamo il caso di un gas perfetto, in cui vale la legge~\eqref{eq:gas-ideale}. Si faccia riferimento alla figura~\ref{fig:criterio-schwarzscild-gas-perfetto}. Consideriamo un caso in cui è possibile la convezione e un caso in cui essa non è possibile

\paragraph{Convezione possibile}
Nel caso illustrato nella figura~\ref{fig:criterio-schwarzscild-gas-perfetto1} la convezione è possibile, infatti in quel caso si ha:
\[
abs*{\dfrac{\ud T}{\ud r}}_\textup{rad} > \abs*{\dfrac{\ud T}{\ud r}}_\textup{ad}
\]
si faccia attenzione al fatto che nel criterio di Schwarzscild~\eqref{eq:criterio-schwarzscild} i gradienti sono espressi in modulo, quindi bisogna guardare la pendenza in modulo delle rette. In particolare, la bolla di gas di sposta da $r_0$ fino a $r_0 + \ud r$ e nella posizione $r_0 + \ud r$ possiede una temperatura (curva rossa -- profilo adiabatico) maggiore dell'ambiente circostante (curva blu -- profilo radiativo). Per un gas perfetto, secondo la ~\eqref{eq:gas-ideale}, a fissata pressione, a una maggiore temperatura corrisponde una densità più bassa. Quindi la bolla possiede una densità più bassa dell'ambiente circostante e tende a continuare il suo moto verso l'esterno. È dunque possibile la convezione.

\paragraph{Convezione impossibile}
Nel caso illustrato nella figura~\ref{fig:criterio-schwarzscild-gas-perfetto2} la convezione \emph{non} è possibile, infatti in quel caso si ha:
\[
abs*{\dfrac{\ud T}{\ud r}}_\textup{rad} < \abs*{\dfrac{\ud T}{\ud r}}_\textup{ad}
\]
In particolare, la bolla di gas di sposta da $r_0$ fino a $r_0 + \ud r$ e nella posizione $r_0 + \ud r$ possiede una temperatura (curva rossa -- profilo adiabatico) minore dell'ambiente circostante (curva blu -- profilo radiativo). Per un gas perfetto, secondo la ~\eqref{eq:gas-ideale}, a fissata pressione, a una minore temperatura corrisponde una densità più alta. Quindi la bolla possiede una densità più alta dell'ambiente circostante e viene respinta verso la posizione iniziale a $r_0$.


Si faccia riferimento alla figura
Si faccia riferimento alla figura~\ref{fig:criterio-schwarzscild-gas-perfetto2}

\subsection{Gradiente di pressione e scala logaritmica}
Il criterio di Schwarzscild è spesso scritto usando il gradiente di temperatura riferito alla \emph{pressione} invece che al raggio e in scala logaritmica. Questo perché la temperatura è intimamente connessa con la pressione.
\begin{equation}
    \dfrac{\ud T}{\ud P} \dfrac{P}{T} = \dfrac{\ud \log T}{\ud \log P} \equiv \nabla
\end{equation}
L'impiego di $\nabla$ semplifica la formulazione del criterio. Ad esempio, possiamo riscrivere l'espressione del gradiente adiabatico~\eqref{eq:gradiente-adiabatico} nella seguente maniera:
\[
\Bigl(1- \frac{1}{\gamma}\Bigr) = \dfrac{P}{T} \dfrac{\ud r}{\ud P} \dfrac{\ud T}{\ud r}\Big|_\textup{ad} = \dfrac{P}{T} \dfrac{\ud T}{\ud P}\Big|_\textup{ad}
\]
da cui:
\begin{equation}
    \nabla_\textup{ad} = \Bigl(1- \frac{1}{\gamma}\Bigr)
\end{equation}
che è semplicemente un numero. In particolare, con $\gamma = 5/3$ si ottiene $\nabla_textup{ad} = 0.4$ ed è possibile riscrivere il criterio di Schwarzscild~\eqref{eq:criterio-schwarzscild} nella seguente materia:
\begin{equation}\label{eq:criterio-schwarzscild-nabla}
    \text{se} \quad \nabla_\textup{rad} > \nabla_\textup{ad} \implies \text{c'è convezione.}
\end{equation}
Si faccia riferimento alla figura~\ref{fig:criterio-schwarzscild-nabla} per un sunto del criterio.

\begin{figure}
\centering
\includegraphics[width=0.4\textwidth]{immagini/criterio-schwarzscild-nabla.png}
\caption{Raffigurazione del criterio di Schwarzscild utilizzando $\nabla$.}
\label{fig:criterio-schwarzscild-nabla}
\end{figure}