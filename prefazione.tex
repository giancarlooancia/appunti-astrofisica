{\noindent \bfseries \Huge \prefacename}
      % \begin{center}
            % \phantomsection \addcontentsline{toc}{chapter}{\prefacename} % enable this if you want to put the preface in the table of contents
            \thispagestyle{plain}
            \\
            \\
Il presente documento raccoglie gli appunti delle lezioni di astrofisica della professoressa Barbara Lanzoni, tenutesi presso l'Università di Bologna, nel corso di fisica dell'A.A 2022--2023. Nato da un'idea di Giancarlo Oancia, il lavoro è proseguito attraverso una stretta collaborazione con Simone Coli, Lorenzo Calandra Buonaura e Alessandro D'Amico. In particolare, è stato effettuato un lavoro iniziale preparativo con il fine di suddividere finemente il lavoro e accordarsi sulle convenzioni tipografiche e di programmazione da usare. Dopo una prima stesura è stato effettuato un lavoro di rilettura e correzione incrociato, in modo che ognuno leggesse la parte scritta dagli altri per identificare quanti più errori possibili. In aggiunta, una revisione finale dell'ideatore del progetto ha avuto lo scopo di unificare il codice nella maniera più accurata possibile.

In particolare, il lavoro si è articolato nella seguente maniera. Giancarlo Oancia si è occupato della stesura dei capitoli \emph{Setting the stage}~\ref{cap:setting-stage}, \emph{Meccanismi di emissione}~\ref{cap:meccanismi-emissione}, \emph{Struttura stellare}~\ref{cap:struttura-stellare} e \emph{Modello di atmosfera stellare}~\ref{cap:modello-atmosfera}. Simone Coli si è occupato dei capitoli \emph{Evoluzione Stellare}~\ref{cap:evoluzione-stellare}, \emph{Ammassi di stelle}~\ref{cap:ammassi-stelle} e \emph{Introduzione alla cosmologia}~\ref{cap:cosmologia}. Lorenzo Calandra Buonaura ha curato il capitolo \emph{Galassie}~\ref{cap:galassie}. Alessandro D'Amico il capitolo \emph{Ammassi di galassie}~\ref{cap:ammassi-galassie}.

Consapevoli che un lavoro di questo tipo, effettuato in un intervallo temporale ristretto a causa delle necessità della sessione di esami, è lungi dall'essere perfetto, è presente una mail a cui si invita di scrivere per riferire ogni possibile correzione: \emph{dispenseastrofisica@gmail.com}.

Ricordando che \emph{non} è possibile utilizzare questo materiale per scopo di lucro, si vuole evidenziare che tutto il codice \LaTeX del progetto e la versione pdf più aggiornata sono disponibili in un \emph{github} pubblico al link: \url{https://github.com/giancarlooancia/appunti-astrofisica}. Colgo l'occasione per ringraziare tutti gli studenti che hanno seguito il corso, i quali hanno contribuito a integrare gli appunti e ad evidenziare errori e concetti poco chiari espressi nel testo.

\begin{flushright}
    --Giancarlo Oancia.
\end{flushright}



        %\end{center}%
    {\vspace*{\stretch{5}}}
